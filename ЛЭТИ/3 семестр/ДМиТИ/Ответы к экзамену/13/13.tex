\documentclass[12pt]{article}
\usepackage[a4paper, portrait, margin=1cm, bottom=2cm]{geometry}
\usepackage{fontspec}
\usepackage[fleqn]{amsmath}
\usepackage{amssymb}
\usepackage{graphicx}
\usepackage{indentfirst}
\usepackage{polyglossia}
\usepackage[dvipsnames]{xcolor}
\usepackage{svg}

\renewcommand{\gcd}{\text{НОД}}

\setmainfont[Ligatures=TeX]{Times New Roman}
\setdefaultlanguage{russian}
\setotherlanguages{english}
\graphicspath{graphics}

\begin{document}

\setcounter{section}{12}
\section{Неприводимые многочлены. Поля Галуа}
\subsection{Неприводимые многочлены}
\subsubsection{Определение}
Пусть $K$ — поле. Тогда $P(x) \in K[x]$ неприводим, если не существует нетривиальный делитель $Q(x) \in K[x]$, такой, что его степень больше 0, меньше степени многочлена $P$ и $P(x)$ делится на $Q(x)$:
\[
      \nexists Q(x) \in K[x] : 0 < \deg Q < \deg P \text{ и } P(x) \vdots Q(x)
\]
\subsubsection{Определение}
$P(x)$ свободный от квадратов, если не существует $Q(x)$, такой, что $\deg Q >$ и $P(x) \vdots Q^2(x)$.
\subsubsection{Утверждение}
$
      \begin{aligned}
             & P(x) = Q^k(x)M(x)                                                                             \\
             & \gcd(Q(x), M(X)) = 1 \text{ и } Q(x) \text{ неприводим } \Rightarrow P'(x) = Q^{k - 1}(x)N(x) \\
             & \gcd(Q(x), N(x)) = 1
      \end{aligned}
$
\subsubsection{Доказательство}
$
      \begin{aligned}
             & P'(x) = kQ^{k - 1}(x)Q'(x)M(x) + Q^k(x)M'(x)                                             \\
             & P'(x) = Q^{k - 1}(x)(kQ'(x)M(x) + Q(x)M'(x)) \text{, где } kQ'(x)M(x) + Q(x)M'(x) = N(x) \\
             & kQ'(x)M(x) = N(x) - Q(x)M'(x)\ \vdots\ Q(x)                                              \\
             & kQ'(x) \ \vdots\ Q(x) \text{ — противоречие}                                             \\
             & \deg Q' = \deg Q - 1                                                                     \\
      \end{aligned}
$

Это работает при $\mathbb{Z}_k$ ($k \not \vdots$  характериситку поля).

\subsubsection{Следствие}
$P(x) = Q_1^{k_1}(x)...Q_m^{k_m}(x)$, $Q_i$ — неприводимы $\Rightarrow\ P'(x) = Q_1^{k_1-1}(x)...Q_m^{k_m - 1}Q(x)$, $Q$ взаимно прост с $Q_i$
\subsubsection{Критерий Эйзенштейн}
$
      \begin{aligned}
             & P(x) \in \mathbb{Z}[x]                                                                                               \\
             & P(x) = a_nx^n + ... + a_0                                                                                            \\
             & a_{n-1} \vdots p,\ a_{n - 2} \vdots p,\ ..., a_0 \vdots p,\ a_n \not \vdots p \Rightarrow \text{ неприводима над } Q
      \end{aligned}
$
\subsubsection{Доказательство}
Рассмотрим $P(x) = f(x)g(x)$.
\allowdisplaybreaks[4]
\begin{align*}
       & f(x) = b_mx^m + ... + b_0                                                              \\
       & g(x) = c_{n - m}x^{n - m} + ... + c_0                                                  \\
       & p(x) = a_nx^n + ... + a_0 = (b_mx^m + ... + b_0)\cdot(c_{n - m} x^{n - m} + ... + c_0) \\
       & a_0 = b_0c_0\ \vdots\ p \qquad b_0\ \vdots\ p \qquad c_0 \not\vdots\ \ p               \\
       & a_1 = b_0c_2 + b_1c_1 + b_2c_0\ \vdots\ p \Rightarrow b_2\ \vdots\ p                   \\
       & \vdots                                                                                 \\
       & a_m = b_0c_m + ... + b_mc_0\ \vdots\ p \Rightarrow b_{m - 1}\ \vdots\ p                \\
       & a_n = b_nc_{n - m}\ \vdots\ p \text{ — противоречие}                                   \\
\end{align*}
\subsection{Поля Галуа}
\subsubsection{Определение}
Конечное поле, или поле Галуа в общей алгебре — поле, состоящее из конечного числа элементов. Обозначается $\mathbb{F}_q$ или $\mathbf{GF}(q)$ или $<\mathbf{GF}(q), +, *>$, где $q = |\mathbf{GF}(q)|$ — порядок поля. Порядком поля называется количество входящих в него элементов. Пример: $\mathbb{Z}_p,\ p \in \mathbb{P}$.
\subsubsection{Определение}
Характериситка поля $F$ — наименьшее $n$, такое, что $\forall a \in F$ выполняется следующее равенство:
\[
      \underbrace{a + a + ... + a}_{n\text{ раз}} = 0
\]

Если такого $n$ не существует, то $n$ считается равным $0$.

\subsubsection{Лемма}
Характериситка поля — простое или $0$.
\subsubsection{Доказательство}
Рассмотрим $n = \alpha \beta$.
\[
      \underbrace{1 + 1 + ... + 1}_{n \text{ раз}} = 0
\]
\[
      \underbrace{\underbrace{1 + 1 + ... + 1}_{\alpha \text{ раз}} + \underbrace{1 + 1 + ... + 1}_{\alpha \text{ раз}} + ... + \underbrace{1 + 1 + ... + 1}_{\alpha \text{ раз}}}_{\beta \text{ раз}} = 0
\]
\[
      \underbrace{\alpha + \alpha + ... + \alpha}_{\beta \text{ раз}} = 0 \qquad \Rightarrow \text{ характеристика — } \mathbb{P}
\]
\subsubsection{Свойства}
$f(x) \in \mathbb{Z}_p[x]$ ($f(x)$ принадлежит множеству многочленов с целыми коэффициентами по модулю $p$)  $\mathbb{Z}_p[x] / f(x)$ (кольцо вычетов многочленов с целыми коэффициентами по модулю $f(x)$)

$
      \displaystyle
      \left.\begin{aligned}
             & g_1(x) \equiv_{f(x)} h_1(x) \\
             & g_2(x) \equiv_{f(x)} h_2(x) \\
      \end{aligned}\right| \Rightarrow \begin{aligned}
             & g_1 + g_2 \equiv_{f(x)} h_1 + h_2 \\
             & g_1g_2 \equiv_{f(x)} h_1h_2       \\
      \end{aligned}
$
\subsubsection{Теорема}
$\mathbb{Z}_p[x] / f(x)\ \Leftrightarrow\ f(x)$ неприводим над $\mathbb{Z}_p$.

$\deg f = m$    $\mathbf{GF}(p^m)$
\subsubsection{Доказательство}
\textbf{Прямое.} Рассмотрим $f(x) = g(x)h(x)$ \\
$g(x)h(x) \equiv_{f(x)} 0 \qquad | \cdot g^{-1}(x) $ \\
$h(x) \equiv_{f(x)} 0 |\Rightarrow f(x) = 0$

\textbf{От обратного.} $\forall g(x) \neq 0$ $\deg g < \deg p$ \\
$\gcd(g(x), f(x)) = 1\ \Rightarrow$ по расширенному алгоритму Евклида:\\
$a(x)g(x) + b(x)f(x) = 1$ \\
$a(x)g(x) \equiv_{f(x)} 1\ \Rightarrow\ a = g^{-1} \ \Rightarrow\ \mathbb{Z}_p$ — поле

\subsubsection{Связь с линейным пространством}
Поле Галуа образует линейное (векторное) пространство. Его аксиомы:
\begin{itemize}
      \item $(a + b) + c = a + (b + c)$
      \item $\exists 0 : a + 0 = a$
      \item $\forall a \  \exists(-a) : a + (-a) = 0$
      \item $a + b = b + a$
      \item $\alpha(a + b) = \alpha a + \alpha b$
      \item $(\alpha + \beta)a = \alpha a + \beta a$
      \item $\alpha(\beta a) = (\alpha \beta) a$
      \item $\exists 1 : 1 \cdot a = a$
\end{itemize}

$\alpha, \beta \in \mathbb{Z}_p$, $F$ — линейное пространство. $m = \dim F$.

\subsubsection{Определение}
$\alpha$ — примитивный элемент, если $\forall b \neq 0 \in F$ \qquad $b = \alpha^i$.
\end{document}
