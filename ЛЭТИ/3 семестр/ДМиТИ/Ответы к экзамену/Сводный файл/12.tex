\documentclass[12pt]{article}
\usepackage[a4paper, portrait, margin=1cm, bottom=2cm]{geometry}
\usepackage{fontspec}
\usepackage[fleqn]{amsmath}
\usepackage{amssymb}
\usepackage{graphicx}
\usepackage{indentfirst}
\usepackage{polyglossia}
\usepackage[dvipsnames]{xcolor}
\usepackage{svg}

\setmainfont[Ligatures=TeX]{Linux Libertine O}
\setdefaultlanguage{russian}
\setotherlanguages{english}
\graphicspath{graphics}

\begin{document}
\setcounter{section}{11}

\section{(Билет 12) Разложение многочленов на свободные от квадратов множетели.}
\subsection{Определение:}
\noindent Пусть K - поле
  \\$P(x)\in K[x]$  \textbf{неприводим} if не $\exists$ нетривиальный делитель: не $\exists$ $Q(x) \in K[x]: 0 < deg(Q) < deg(P)$ и $ P(x) \vdots Q(x) $;
  \\$ P(x) \vdots  $ С 
  \\$ P(x) \vdots cP(x)$
  \\$x^2 - 2 $ над $\mathbb{Q}$
  \\$\sqsupset\mid x^2-2 \vdots x - a$
  \\$a^2-2=0$
  \\$a^2=2$
  \\$x^2-2=(x-\sqrt2)(x+\sqrt2)$
  \\То есть
  \\$x^2+1$ над $\mathbb{Q}$ \textbf{неприводим}!
  \\над $\mathbb{R}$ \textbf{неприводим}!
  \\$x^2+1=(x-i)(x+i)$ над $\mathbb{C}$
  \\$x^2+1=(x^2+1)$ над $\mathbb{Z}$
\subsection{Определение: $P(x)$ свободный от квадратов }
    \noindent не $\exists$ $ Q(x)$: $deg(Q) > 0$
    \\$ P(x) \vdots Q(x)^2 $ 
    \\$ P(x) = {P_1(x)}{P_2(x)^2}{P_3(x)^3}$...${P_n(x)^n}$
    \\${P_1(x)}$,...,${P_n(x)}$ попарно взаимно/пр. и свободны от квадратов
\subsection{Утверждение: $P(x) = Q(x)^kM(x)$}
\noindent $NOD(Q(x),M(x)) = 1$ и $Q(x)$ неприводим => $P'(x) = Q(x)^{k-1}N(x)$
\\$NOD(Q(x),N(x)) = 1$
\\ Доказательство: 
\\$P'(x) = kQ(x)^{x-1}Q'(x)M(x)+Q(x)M'(x)$
\\$P'(x) = Q(x)^{x-1}(kQ'(x)M(x)+Q(x)M'(x))$
\\$kQ'(x)M(x)=N(x)- Q(x)M'(x) \vdots Q(x)$
\\$kQ'(x)\vdots Q(x)$ ?! Противоречие
\\Это работает при $Z_k$ k характ поля, p - характ. k if $\forall a\in K a+a+...+a=0 $
\\$K$-поле характеристики ноль
\subsection{Следствие:}
\noindent $P(x)=Q_1(x)^{k_1}$$... Q_m(x)^{k_n}$,$Q_i$- неприводимы => $P'(x)=Q_1(x)^{k_1-1}$$... Q_m(x)^{k_n-1}$,$Q$ вз./пр. с $Q_i$
\\Доказательство:
\\$P'(x)=\sum\limits _{i=1}^{m} K_iQ_1(x)^{k_1}$$...Q_m(x)^{k_n}$$Q'_i(x)=Q_1(x)^{k_1-1}$$...Q_m(x)^{k_m-1}$$\sum\limits _{i=1}^{n}\frac{k_iQ_1(x)...Q_m(x)}{Q_i(x)}Q'_i(x)$
\\$NOD(Q_1,Q_i)\neq 1$
\\$NOD(Q_1,Q_i) = Q_i$
\\$Q \vdots Q_i$
\\$Q(x)= k_1Q_2Q_3...Q_mQ'_1+k_2Q_1Q_3...Q_mQ'_2+...+...Q_{m-1}Q'_m\vdots Q_i$
\\$k_iQ'_i\vdots Q_i$
\\$Q'_i\vdots Q_i =>$ Противоречие 
\subsection{Алгоритм разложения на свободные от  квадратов множ.:}
\noindent На языке программирования питон: 
\\$j=1$
\\$while$ $deg P>0$
\\$p' = diff(P)$
\\$S=NOD(P,P')$
\\$r = \frac{P}{S}$
\\$t = NOD(r,P')$
\\$P_j=\frac{r}{t}$
\\$P:=\frac{P}{r}$
\\$j++$
\\$return$ $P_1P^2_2...P_{j-1}$
\\
\\Где:
\\
\\$P=P_1P^2_2P^3_3...P^n_n$
\\$P'=P_2P^2_3...P_n^{n-1}$
\\$NOD(P,P')=P_2P_3^2...P_n^{n-1}$
\\$r=\frac{P}{NOD(P,P')}=P_1P_2...P_n$
\\$NOD(r,P')=P_2...P_n$
\\$\frac{r}{NOD(r,P')}=P_1$
\\
\\Пример:
\\$P(x)=x^3+x^2-x-1$
\\$P'(x)=3x^2+2x-1$
\\$NOD(P(x),P'(x))=x+1=S(x)$
\\$r(x)= P/S=x^2-1$
\\$NOD(r,P')=x+1=t(x)$
\\$P_1=x-1$
\\new $P=x+1$
\\$P'=1$
\\$S=1$
\\$r = x+1$
\\$t = 1$
\\$P_2=x+1$
\\$P(x) = P_1(x)P_2(x)^2P_3(x)^3...P_n(x)^n$
\\$P_1(x),...,P_n(x)$ попарно вз./пр. и свободные от квадратов множетели
\subsection{$P(x) \in\mathbb{Z} P(x)\vdots x - \frac{p}{q}\\NOD(p,q)=1 => a_n \vdots q $, где $p(v)=a_n(x^n)+...+a_0$}
\subsection{Теорема Безу}
\noindent $P(x) \vdots x - \frac{p}{q} => p(\frac{p}{q} = 0)$
\\Доказательство: 
\\$\frac{a_nP^n}{q_n}+\frac{a_{n-1}P^{n-1}}{q^{n-1}}+\frac{a_1P}{q}+a_0=0$
\\$a_nP^n+a_{n-1}P^{n-1}q+...+a_nPq^{n-1}+a_0q^n=0$
\\$a_np^n \vdots q => a_n \vdots q$
\\$a_0p^n \vdots p => a_0 \vdots q$
\subsection{Критерий Эзерштейна}
\noindent $P(x) \in \mathbb{Z}[x]$
\\$P(x) = a_nx^n+...+a_0$
\\$a_{n-1}\vdots P, a_{n-2} \vdots P,...,a_0\vdots P$ и $a_n$не$\vdots P=>$ неприводима над $Q$ 
\\Доказательство: 
\\Пусть $P(x) = f(x)g(x)$
\\$f(x) = b_mx^m+...+b_0$
\\$g(x) = C_{n-m}x^{n-m}+...+C_0$
\\$P(x) = a_nx^n+...+a_0=(b_mx^n+...+b_0)(C_{n-m}x^{n-m}+...+C_0)$
\\$a_0=b_0C_0 \qquad b_0 \vdots P$
\\$. \hspace{6em}C_0$ не$ \vdots P$
\\$a_1=b_0C_2+b_1C_1+b_2C_0 => b_2\vdots P $
\\$a_m=b_nC_m+...+b_mC_0 => b_{n-1}\vdots P $
\end{document}
