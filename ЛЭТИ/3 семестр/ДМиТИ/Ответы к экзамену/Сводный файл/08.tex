\documentclass[12pt]{article}
\usepackage[a4paper, portrait, margin=1cm, bottom=2cm]{geometry}
\usepackage{fontspec}
\usepackage[fleqn]{amsmath}
\usepackage{amssymb}
\usepackage{graphicx}
\usepackage{indentfirst}
\usepackage{polyglossia}
\usepackage[dvipsnames]{xcolor}
\usepackage{svg}

\usepackage{booktabs}

\setmainfont[Ligatures=TeX]{Linux Libertine O}
\setdefaultlanguage{russian}
\setotherlanguages{english}
\graphicspath{graphics}

\begin{document}

\setcounter{section}{7}

\section{Функция Эйлера и её свойства}

\subsection{Определение} Функция Эйлера $\varphi(n)$ ставит в соответствие каждому натуральному $n$ количество чисел, меньших $n$ и взаимно простых с $n$. Будем полагать $\varphi(1)=1$.

\subsection{Свойства} 
\begin{enumerate}
    \item $p\in P:\varphi(p)=p-1,$ - Функция Эйлера для простого числа
    \item $p\in P:k\in N:\varphi(p^{k})=p^{k}-p^{k-1},$ - Функция Эйлера для простого числа в степени
    \item $\sqsupset$НОД$(a,b)=1=>\varphi(ab)=\varphi(a)\varphi(b)$, при $a,b\in \mathbb{N}$ - мультипликативность функции Эйлера
\end{enumerate}

\subsection{Утверждение} Для простого $p$ значение функции Эйлера задаётся формулой: \[\varphi(p)=p-1,\] которая следует из определения. Если $p$ - простое, то все числа, меньшие $p$, взаимно просты с ним, а их ровно $p-1$ штук.
\par Для вычисления функции Эйлера от степени простого числа используют следующую формулу: \[\varphi(p^{n})=p^{n}-p^{n-1}.\] 
\par \textit{Доказательство.} Подсчитаем количество чисел от $1$ до $p^{n}$, которые не взаимно просты с $p^{n}$. Все они, очевидно, кратны $p$, то есть, имеют вид: $p,2p,3p,\dots,p^{n-1}p$. Всего таких чисел $p^{n-1}$. Поэтому количество чисел, взаимно простых с $p^{n}$, равно $p^{n}-p^{n-1}$.

\subsection{Следствие} Если НОД$(a,b)=1$, тогда $\varphi(ab)=\varphi(a)\varphi(b)$.
\par \textit{Доказательство.} В полной системе вычетов по модулю $a$ существует $\varphi(a)$ значений $x$, таких, что НОД$(a,x)=1$. Также и для полной системы вычетов по модулю $b$ существует $\varphi(b)$ значений $y$, таких, что НОД$(b,y)=1$. Следовательно, всего имеется $\varphi(a)\varphi(b)$ значений $z$, взаимно простых с $ab$. Но значения $z$ образуют полную систему вычетов по модулю $ab$, и чисел, взаимно простых с $ab$, в ней $\varphi(ab)$.

\subsubsection{Пример} Для иллюстрации доказательства следствия составлена таблица 1 величин (x,y) при $a=4$ и $b=5$. Возможные значения для x - числа $0,1,2,3$, возможные значения для y - числа $0,1,2,3,4$. Из них для $x$ имеется два значения (1 и 3) взаимно простых с a (так как $\varphi(4)=2$). Соответственно для $y$ также есть четыре значения (1,2,3 и 4) взаимно простых с b (так как $\varphi(5)=4$). Эти значения помещены в кружочки, как и соответствующие им значения $z=ay+bx$.
\par Выделенные значения $z$ дают 8 чисел, меньших 20 и взаимно простых с ним, Таким образом: \[\varphi(20)=\varphi(4)\varphi(5)=2*4=8.\]

\begin{table}[h!]
\centering
    \begin{tabular}{|c|c|c|c|c|c|}
    \toprule  
    \multicolumn{1}{|c|}{} & \multicolumn{5}{c|}{\textbf{y}} \\
    \cmidrule(){2-6}
    \textbf{x} & {0} & \textcircled{1} & \textcircled{2} & \textcircled{3} & \textcircled{4}\\
    \midrule
    0 & 0 & 4 & 8 & 12 & 16\\
    \textcircled{1} & 5 & \textcircled{9} & \textcircled{13} & \textcircled{17} & \textcircled{1}\\
    2 & 10 & 14 & 18 & 2 & 6\\
    \textcircled{3} & 15 & \textcircled{19} & \textcircled{3} & \textcircled{7} & \textcircled{11}\\
    \bottomrule
\end{tabular}
\caption{\textbf{Доказательство мультипликативности}}
\end{table}

\subsection{Следствие} Всякое натуральное число n>1 представляется в виде: \[n=p^{\alpha_{1}}_{1}*\dots*p^{\alpha_{k}}_{k},\] где $p_{1}<\dots<p_{k}$ - простые числа, $\alpha_{1}<\dots<\alpha_{k}$ - натуральные числа.
\par Тогда $\varphi(n)=p^{\alpha_{1}-1}_{1}(p_{1}-1)*p^{\alpha_{2}-1}_{2}(p_{2}-1)*\dots=n\biggl(1-\cfrac{1}{p_{1}}\biggl)*\biggl(1-\cfrac{1}{p_{2}}\biggl)$
\subsubsection{Пример} Для доказательства следствия приведён пример вычисления:

\begin{enumerate}
    \item $\varphi(49)=\varphi(7^{2})=7^{2}-7=42,$
    \item $\varphi(30)=\varphi(2*3*5)=\varphi(2)\varphi(3)\varphi(5)=(2-1)(3-1)(5-1)=8,$ 
    \item $\varphi(60)=60\biggl(1-\cfrac{1}{2}\biggl)*\biggl(1-\cfrac{1}{3}\biggl)*\biggl(1-\cfrac{1}{5}\biggl)=16.$
\end{enumerate}

\subsection{Следствие} Функция Эйлера $\varphi(n)$ принимает только чётные значения при $n>2$. Причём, если $n$ имеет $k$ различных нечётных простых делителей, то $2^{k}\mid\varphi(n)$.
\par \textit{Доказательство.} Если $\exists p>2$ и $p$ - простое число, тогда \[\varphi(n)\vdots(p-1)\vdots2.\] Тогда если $n=2^{k}$, то \[\varphi(n)=2^{k}\biggl(1-\cfrac{1}{2}\biggl)=2^{k-1}\vdots2.\]

\subsection{Теорема Формула Гаусса}
\subsection{Определение} Пусть d пробегает все делители числа m. Тогда \[m=\sum_{d\mid m}\varphi(d).\]
\par \textit{Доказательство.} $\sqsupset p$ - простое число. Тогда \[\varphi(p^{n})=p^{n-1}(p-1).\] Таким образом \[1+\varphi(p)+\varphi(p^{2})+\dots+\varphi(p^{\alpha})=1+(p-1)+\dots+(p^{\alpha}-p^{\alpha-1})=p^{\alpha}\] Пусть $m=p^{\alpha_{1}}_{1}*\dots*p^{\alpha_{k}}_{k}$, тогда \[\prod^{k}_{i=1}(1+\varphi(p_{i})+\dots+\varphi(p^{\alpha_{i}}_{i}))=\prod^{k}_{i=1}p^{\alpha_{i}}_{i}=m.\] Исходя из этого \[\prod^{k}_{i=1}\sum^{\alpha_{i}}_{j=0}\varphi(p^{j}_{i})=\sum_{\substack{
    0\leq j_{1}\leq \alpha_{1} \\
    \vdots \\
    0\leq j_{k}\leq \alpha_{k}
}}\varphi(p^{j_{1}}_{1})*\varphi(p^{j_{2}}_{2})*\dots*\varphi(p^{j_{k}}_{k})=\sum_{\substack{
    0\leq j_{1}\leq \alpha_{1} \\
    \vdots \\
    0\leq j_{k}\leq \alpha_{k}
}}\varphi(p^{j_{1}}_{1})*\dots*p^{j_{k}}_{k})=\sum_{d\mid m}\varphi(d)\]
\subsubsection{Пример} $\displaystyle\sum_{d\mid 30}\varphi(d)=\varphi(1)+\varphi(2)+\varphi(3)+\varphi(5)+\varphi(6)+\varphi(10)+\varphi(15)+\varphi(30)=1+1+2+4+2+4+8+8=30$

\subsection{Теорема Эйлера}
Пусть НОД$(a,m)=1$, тогда $a^{\varphi(m)}\equiv_{m}1.$
\par \textit{Доказательство.} Пусть НОД$(a,m)=1.$ Тогда классов вычетов взаимно простых с $m$ будет $\varphi(m)$. Пусть $\{x_{1},x_{2},\dots,x_{\varphi(m)}\}$ - представители классов, то $\{ax_{1},ax_{2},\dots,ax_{\varphi(m)}\}$ будут также взаимно просты с $m$. Также, если $ax_{i}\equiv_{m} ax_{j}$, то $x_{i}\equiv_{m} x_{j}.$ Следовательно, числа $ax_{i}$ - также представители классов вычетов, взаимно простых с $m$. Тогда каждое $ax_{i}$ сравнимо с одним и только одним $a_{j}$. \[x_{i}*\dots*x_{\varphi(m)}\equiv_{m} ax_{i}*\dots*ax_{\varphi(m)}.\] После сокращения получаем нужное сравнение: \[1\equiv_{m} a^{\varphi(m)}.\]

\subsection{Следствие: Малая теорема Ферма}
Если $p$ - простое число, то $a^{p}\equiv_{p}a$
\par \textit{Доказательство.} 
\par Если НОД$(a,p)=1$, то $a^{\varphi(p)}\equiv_{p}1$, $a^{p-1}\equiv_{p}1$, $a^{p}\equiv_{p}a\equiv_{p}0$. 
\par Если $a\vdots p$, то $a\equiv_{p}0$, $a^{p}\equiv_{p}a\equiv_{p}0$.

\subsection{Следствие}
Если p - простое число, то $(a+b)^{p}\equiv_{p}a^{p}+b^{p}$
\par \textit{Доказательство.} $(a+b)^{p}\equiv_{p}a+b\equiv_{p}a^{p}+b^{p}$

\subsection{Следствие}
Если
\begin{table}[h!]
    \begin{tabular}{r|c}
        $a\equiv_{m}b$ \\
        $c\equiv_{\varphi(m)}d$ & $=>a^{c}\equiv_{m}b^{d}$ \\
        НОД$(a,m)=1$
    \end{tabular}
\end{table}
\par \textit{Доказательство.} $a^{c}\equiv_{m}b^{c}\equiv_{m}b^{k\varphi(m)+d}=(b^{\varphi(m)})^{k}*b^{d}\equiv_{m}b^{d}$
\par $c\equiv_{\varphi(m)}d=>c=k\varphi(m)+d$
\par НОД$(a,m)=1=>$ НОД$(b,m)=1$
\par НОД$(a,m)=$ НОД$(a-\widetilde{k}m,m)$

\subsubsection{Пример}
$25^{11^{35}}\equiv_{34}(-9)^{3}=81*(-9)\equiv_{34}-13*9=-117\equiv_{34}19$
\par НОД$(25,34)=1$
\par $11^{35}\equiv_{16}11^{3}\equiv_{16}(-5)^{3}=-125\equiv_{16}3$
\par $\varphi(34)=34\biggl(1-\cfrac{1}{2}\biggl)*\biggl(1-\cfrac{1}{17}\biggl)=16$
\par НОД$(11,16)=1$
\par $\varphi(16)=8$
\par $35\equiv_{8}3$

\end{document}

