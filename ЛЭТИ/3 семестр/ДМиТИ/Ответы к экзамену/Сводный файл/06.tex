\documentclass[12pt]{article}
\usepackage[a4paper, portrait, margin=1cm, bottom=2cm]{geometry}
\usepackage{fontspec}
\usepackage[fleqn]{amsmath}
\usepackage{amssymb}
\usepackage{graphicx}
\usepackage{indentfirst}
\usepackage{polyglossia}
\usepackage[dvipsnames]{xcolor}
\usepackage{svg}

\setmainfont[Ligatures=TeX]{Linux Libertine O}
\setdefaultlanguage{russian}
\setotherlanguages{english}
\graphicspath{graphics}

\begin{document}

\setcounter{section}{5}
\section{Кольца вычетов. Полная система вычетов. Теорема о кольцах вычетов по простому модулю}
\subsection{Кольца вычетов}
$<x,+,*>$
\begin{equation*}
    \text{Кольцо}
    \begin{cases}
        \text{Абелева группа}
        \begin{cases}
            \text{Группа}
            \begin{cases}
                \text{Полугруппа}
                \begin{cases}
                    1^{\circ} \quad (a+b)+c = a+(b+c)
                \end{cases}
                \\
                2^{\circ} \quad \exists 0: a+0 = a
                \\
                3^{\circ} \quad \forall a \quad\exists (-a): a + (-a) = 0
            \end{cases}
            \\
            4^{\circ} \quad a + b = b + a
        \end{cases}
        \\
        5^{\circ} \quad \begin{cases}
                            (a+b)*c = (a*c) + (b*c)
                            \\
                            c*(a+b) = (c*a) + (c*b)
                        \end{cases}
    \end{cases}
\end{equation*}

\subsubsection{Свойства колец}
\begin{itemize}
    \item Кольцо ассоциативно: $(a*b)*c = a*(b*c)$
    \item Кольцо коммутативно: $a*b = b*a$
    \item Кольцо с единицей: $\exists 1: 1*a = a$
    \item Область целостности: $\exists a\ne 0, b \ne 0 \Rightarrow a*b \ne 0$
\end{itemize}

\subsubsection{Примеры колец}
\begin{itemize}
    \item Кольцо целых чисел $\mathbb{Z}$, кольцо рациональных чисел $\mathbb{Q}$, кольцо вещественных чисел $\mathbb{R}$
    \item Кольцо $\mathbb{Z}[i]$ целых гауссовых чисел вида $a + bi$, где $a,b \in \mathbb{Z}$
    \item Кольцо $\mathbb{Z}[\sqrt{2}]$ вещественных чисел вида $a + b\sqrt{2}$ c целыми $a,b$
\end{itemize}

\subsubsection{Поле}
Поле - коммутативное ассоциативное кольцо с единицей в котором $\forall a \ne 0 \quad \exists a^{-1}: a * (a^{-1}) = 1$

\subsubsection{Множество классов вычетов}
Множество классов вычетов (обозначают $\mathbb{Z}_m$) является ассоциативным коммутативным кольцом с единицей
\subsubsection{Примеры полей}
\begin{itemize}
    \item Числовые поля $\mathbb{Q}$, $\mathbb{R}$
    \item Поле $\mathbb{Q}[i]$ рациональных чисел вида $a + bi$, где $a,b \in \mathbb{Q}$
    \item Поле $\mathbb{Z}[\sqrt{2}]$ вещественных чисел вида $a + b\sqrt{2}$ c рациональными $a,b$
\end{itemize}

\subsection{Полная система вычетов}

Классом вычетов по модулю m называют множество чисел с одинаковым остатком при делении на m
\subsubsection{Определение}
Если взять по одному представителю из каждого класса вычетов, то эти m чисел образуют полную систему вычетов по модулю m

\subsubsection{Примеры простейших полных систем вычетов}
\begin{itemize}
    \item $\{0, 1, 2, \dots, m-1\}$ наименьшие положительные вычеты
    \item $\{0, -1, -2, \dots, -(m-1)\}$ наименьшие отрицательные вычеты
    \item для произвольного $a \in \mathbb{Z} \quad\{a, a+1, a+2, \dots, a+(m-1)\}$
    \item если $D(a,m) = 1$, то $\{0, a, 2a, \dots, (m-1)a\}$
\end{itemize}

\subsection{Теорема о кольцах вычетов по простому модулю}

Китайская теорема об остатках утверждает, что система сравнений с попарно взаимно простыми модулями
$m_1, m_2, \dots, m_k$:\\
\begin{equation*}
    \begin{cases}
        x \equiv c_1  (\text{mod } m_1) \\
        x \equiv c_2 (\text{mod } m_2)  \\
        \dots                           \\
        x \equiv c_k (\text{mod } m_k)
    \end{cases}
\end{equation*}
всегда разрешима и имеет единственное решение по модулю ($m_1m_2\dots m_k$)\\
\\
Другими словами, китайская теорема об остатках утверждает, что кольцо вычетов по модулю произведения нескольких попарно взаимно простых чисел является прямым произведением соответствующих множителям колец вычетов

\subsubsection{Теорема}
Если модуль $m$ - составное число, то $\mathbb{Z}_m$ не является полем\\
\\
Доказательство\\
\\
Пусть \quad $m = p_1p_2$, \qquad $1 < p_1, p_2 < m$\\
\\
Будем считать, что $P_1, P_2 \in \mathbb{Z}_m$ - классы вычетов, которым принадлежат $p_1, p_2$. Тогда:\\
\\
\quad $P_1P_2 = 0$, \quad $P_1, P_2 \ne 0$ \quad и элементы $P_1, P_2$ необратимы.\\
\\
Следовательно, $\mathbb{Z}_m$ - не поле

\end{document}