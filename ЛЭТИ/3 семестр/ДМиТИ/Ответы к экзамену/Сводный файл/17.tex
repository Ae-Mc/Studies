\documentclass[12pt]{article}
\usepackage[utf8x]{inputenc}
\usepackage[russian,english]{babel}
\usepackage[a4paper, portrait, margin=1cm, bottom=2cm]{geometry}
\usepackage{fontspec}
\usepackage[fleqn]{amsmath}
\usepackage{amssymb}
\usepackage{graphicx}
\usepackage{indentfirst}
\usepackage{polyglossia}
\usepackage[dvipsnames]{xcolor}
\usepackage{svg}
\usepackage{blkarray}

\setmainfont[Ligatures=TeX]{Times New Roman}

\begin{document}

\section{Алгоритм Берлекемпа}
$\exists F$ - многочлен, свободный от квадратов. Разложим $F = f_1 * f_2 * f_3, ..., f_s (f_i$ - неприводим над $\mathbb{Z}_p )$
\subsection {Алгоритм:}
Составить матрицу A, которая имеет вид:

\[
A = 
\begin{blockarray}{cccccc}
x^0 & x^{1*p} & x^{2*p} & ... & x^{(deg(f) - 1)*p} \\
\begin{block}{(ccccc)c}
  . & . & . & . & . & x^0 \\
  . & . & . & . & . & x^1 \\
  . & . & . & . & . & x^2 \\
  . & . & . & . & . & ... \\
  . & . & . & . & . & x^{deg(f)-1} \\
\end{block}
\end{blockarray}
 \]
 
Каждый столбец матрицы соответствует векторному разложению многочленов из множества \\
$\{x^0, x^{1*p}, x^{2*p}, ..., x^{(deg(f)-1)*p}\}$ \hspace{0,5cm} (1)\\
в базисе $\{x^0, x^1, x^2, ..., x^{deg(f)-1}\}$\\\\
Для получения векторного представления каждого из многочленов множества (1) необходимо привести их по модулю f.\\
$x_0 \equiv_{f(x)} x^0\\
x_1 \equiv_{f(x)}...\\
x_2 \equiv_{f(x)}...$\\

Далее необходимо найти собственные векторы матрицы A. Для этого получим матрицу $B = A - E$, где E - единичная матрица. Далее нужно привести матрицу к ступенчатому виду методом Гаусса или же любым другим методом и найти ранг матрицы.\\\\
Если $rank B = 1$, то f - неразложим над $\mathbb{Z}_p$.\\
Если $1 < rank B$  и $ < deg(f)-1$, то f - разложим над $\mathbb{Z}_p$\\\\
Далее решим уравнение для нахождения собственных векторов $h_i$:\\
$B*h_i= \begin{pmatrix}
                      0 \\
                      0 \\
                      ... \\
                      0
                  \end{pmatrix}$\\
Количество подходящих векторов h можно найти по формуле:\\
Кол-во $h_i = deg f - rank B$\\
$h_1$ всегда имеет вид $h_1= \begin{pmatrix}
                      1 \\
                      0 \\
                      ... \\
                      0
                  \end{pmatrix}$\\
$h_2, h_3$ и т.д. необходимо найти аналитическим методом либо другим методом для нахождения собственных векторов матрицы.\\

После нахождения всех векторов $h_i$ приведем их к виду многочлена.\\
\textit{Пример:}\\
$h_1= \begin{pmatrix}
                      1 \\
                      0 \\
                      0 \\
                      1 \\
                      0
                  \end{pmatrix} = 1*x^0 + 0 * x^1 + 0 * x^2 + 1 * x^3 + 0 * x^4 = x^3 + 1$\\\\

Итоговое разложение будет иметь вид:\\
$f(x) = \underset{c \in \mathbb{Z}_p}{\prod}$ НОД $(f(x), h_i - c)$\\

Примечание: $h_1$ в формуле разложения не рассматривается, т.к. НОД = 1 или $f(x)$ нас не интересует.

\subsection{Пример}
$f(x) = x^7 + x^6 + x^5 + x^4 + x^3 + x + 1$ над $\mathbb{Z}_2$\\\\
\textbf{Шаг 1}: Проверить, свободен ли f(x) от квадратов\\
\textbf{Шаг 2}: Составить матрицу A.\\
Для этого приведем элементы из множества $\{x^0, x^2, x^4, x^6, x^8, x^{10}, x^{12}\}$ по модулю $f(x)$.\\
$x^0 \equiv_{f(x)} x^0\\
x^2 \equiv_{f(x)} x^2\\
x^4 \equiv_{f(x)} x^4\\
x^6 \equiv_{f(x)} x^6\\
x^8 \equiv_{f(x)} x^7 + x^6 + x^5 + x^4 + x^2 + x \equiv x^3 + x^2 + 1\\
x^{10} \equiv_{f(x)} x^8 * x^2 \equiv x^2 * (x^3 + x^2 + 1) \equiv x^5 + x^4 + x^2\\
x^{12} \equiv_{f(x)} x^{10} * x^2 \equiv x^7 + x^6 + x^4 \equiv x^5 + x^3 + x + 1\\$
Тогда A имеет вид:

\[
A = 
\begin{blockarray}{cccccccc}
x^0 & x^2 & x^4 & x^6 & x^8 & x^{10} & x^{12}\\
\begin{block}{(ccccccc)c}
  1 & 0 & 0 & 0 & 1 & 0 & 1 & x^0\\
  0 & 0 & 0 & 0 & 0 & 0 & 1 & x^1\\
  0 & 1 & 0 & 0 & 1 & 1 & 0 & x^2\\
  0 & 0 & 0 & 0 & 1 & 0 & 1 & x^3\\
  0 & 0 & 1 & 0 & 0 & 1 & 0 & x^4\\
  0 & 0 & 0 & 0 & 0 & 1 & 1 & x^5\\
  0 & 0 & 0 & 1 & 0 & 0 & 0 & x^6\\
\end{block}
\end{blockarray}
 \]\\
 \textbf{Шаг 3}: Найти собственные векторы матрицы A\\
 
\[
B = A - E = 
\begin{blockarray}{cccccccc}
x^0 & x^2 & x^4 & x^6 & x^8 & x^{10} & x^{12}\\
\begin{block}{(ccccccc)c}
  0 & 0 & 0 & 0 & 1 & 0 & 1 & x^0\\
  0 & 1 & 0 & 0 & 0 & 0 & 1 & x^1\\
  0 & 1 & 1 & 0 & 1 & 1 & 0 & x^2\\
  0 & 0 & 0 & 1 & 1 & 0 & 1 & x^3\\
  0 & 0 & 1 & 0 & 1 & 1 & 0 & x^4\\
  0 & 0 & 0 & 0 & 0 & 0 & 1 & x^5\\
  0 & 0 & 0 & 1 & 0 & 0 & 1 & x^6\\
\end{block}
\end{blockarray}
 \]\\
 Приведенная матрица B будет иметь вид:\\
 \[
B = 
\begin{blockarray}{cccccccc}
x^0 & x^2 & x^4 & x^6 & x^8 & x^{10} & x^{12}\\
\begin{block}{(ccccccc)c}
  0 & 0 & 0 & 0 & 0 & 0 & 0 & x^0\\
  0 & 1 & 0 & 0 & 0 & 0 & 0 & x^1\\
  0 & 0 & 1 & 0 & 0 & 1 & 0 & x^2\\
  0 & 0 & 0 & 0 & 1 & 0 & 0 & x^3\\
  0 & 0 & 0 & 0 & 0 & 0 & 0 & x^4\\
  0 & 0 & 0 & 0 & 0 & 0 & 1 & x^5\\
  0 & 0 & 0 & 1 & 0 & 0 & 0 & x^6\\
\end{block}
\end{blockarray}
 \]\\
 $rank B = 5\\$
 Решим уравнение: $B * h_i= \begin{pmatrix}
                      0 \\
                      0 \\
                      0 \\
                      0 \\
                      0 \\
                      0 \\
                      0
                  \end{pmatrix}$\\\\
 Кол-во $h_i = deg(f) - rank B = 7 - 5 = 2$\\\\
$h_1= \begin{pmatrix}
                      1 \\
                      0 \\
                      0 \\
                      0 \\
                      0 \\
                      0 \\
                      0
                  \end{pmatrix}$
$h_2= \begin{pmatrix}
                      0 \\
                      0 \\
                      1 \\
                      0 \\
                      0 \\
                      1 \\
                      0
                  \end{pmatrix}$\\\\
$h_1 = 1$\\
$h_2 = x^5 + x^2$\\
\textbf{Шаг 4}: Найдем итоговое разложение\\\\
$f(x) = \underset{c \in \mathbb{Z}_2}{\prod}$ НОД $(f(x), h_i - c)$\\
НОД $(f(x), x^5 + x^2 - 0)$ = $x^2 + x + 1$\\
НОД $(f(x), x^5 + x^2 - 1)$ = $x^5 + x^2 + 1$\\\\
Ответ: $f(x) = (x^2 + x + 1)(x^5 + x^2 + 1)$
\end{document}