\documentclass[14pt]{article}
\usepackage[utf8]{inputenc}
\usepackage[a4paper, portrait, margin=1cm, bottom=2cm]{geometry}
\usepackage{fontspec}
\usepackage[fleqn]{amsmath}
\usepackage{amssymb}
\usepackage{graphicx}
\usepackage{indentfirst}
\usepackage{polyglossia}
\usepackage[dvipsnames]{xcolor}
\usepackage{svg}

\setmainfont[Ligatures=TeX]{Linux Libertine O}
\setdefaultlanguage{russian}

\title{Матан}
\author{Валера Чумак}
\date{December 2023}

\begin{document}

\section{Система шифрования RSA}

\textbf{Определение 1}: Функция $f$ называется односторонней, если для любого $x$ существует эффективный алгоритм вычисления $f(x)$, но не существует эффективного алгоритма решения уравнения $f(x) = a$.

\textbf{Определение 2}: Функция $f_k(x)$ называется функцией с секретом, если для любого $k$ и $x$ существует эффективный алгоритм вычисления $f_k(x)$, такой что $f_k(x) = a$, но не существует эффективного алгоритма решения уравнения $f_k(x) = a$. Однако, если значение $k$ известно, то существует эффективный алгоритм решения уравнения $f_k(x) = a$.

RSA, разработанная Райвестом, Шамиром и Адлеманом, определяется функцией $f(x) = x^e \mod m$, где $m$ - произведение двух больших простых чисел $p$ и $q$. Для выбора открытого ключа $e$ необходимо выбрать число, взаимно простое с функцией Эйлера $\varphi(m) = (p-1)(q-1)$. Закрытый ключ $d$ находится из уравнения $e \cdot d \equiv 1 \mod \varphi(m)$.

Для шифрования сообщения $x$ отправитель (Алиса) использует открытый ключ $(m,e)$ получателя (Боба) и преобразует сообщение в зашифрованное сообщение $c$ с помощью формулы $c \equiv x^e \mod m$. Зашифрованное сообщение $c$ отправляется Бобу.

Для расшифровки сообщения Боб использует свой закрытый ключ $d$ и преобразует зашифрованное сообщение $c$ обратно в исходное сообщение $x$ с помощью формулы $x \equiv c^d \mod m$.

RSA также может использоваться для создания электронных подписей. Электронная подпись используется для подтверждения подлинности и целостности данных. Она создается путем хеширования сообщения и шифрования полученного хеша закрытым ключом отправителя. Получатель может проверить подлинность сообщения, расшифровав подпись с помощью открытого ключа отправителя и сравнив полученный хеш с хешем исходного сообщения.
\end{document}
