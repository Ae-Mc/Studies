\documentclass[12pt]{article}
\usepackage[a4paper, portrait, margin=1cm, bottom=2cm]{geometry}
\usepackage{fontspec}
\usepackage[fleqn]{amsmath}
\usepackage{amssymb}
\usepackage{graphicx}
\usepackage{indentfirst}
\usepackage{polyglossia}
\usepackage{blkarray}
\usepackage[dvipsnames]{xcolor}
\usepackage{svg}

\setmainfont[Ligatures=TeX]{Times New Roman}
\setdefaultlanguage{russian}
\setotherlanguages{english}
\graphicspath{graphics}

\begin{document}
\section{Коды Рида-Соломона:}
Рассмотрим алгоритм на примере:\par
Поле $GF(16)$ порождается присоединением к $GF(2)$ корня $a$ многочлена $P(x)$, $P(x) = x^{4}+x^{3}+1$\par
Пусть порождающий много член имеет вид $G(x) = (x-a)(x-a^2)(x-a^3)(x-a^4)$, тогда количество\par
ошибок многочлена которое исправит код Рида-Соломона $2t = deg(G(x)) \Rightarrow t = 4/2 = 2.$\par
Принятое сообщение $S(x) = a^{9}x^14 + a^5x^{13} + a^{12}x^{12} + a^{10}x^{11} + a^7x^9 + a^5x^8 + a^7x^7 + a^{13}x^6 + a^3x^5 +$\par $+ a^{11}x^4 + a^3x^3 + a^{10}x^2 + a^8x + a$\par
Вырази все $a$ пока они не зациклятся, т.е. $a_{n} = a_{0}$\par
$a^{0} = a^{0} = 1$\par
$a^{1} = a$\par
$a^{2} = a^2$\par
$a^{3} = a^3$\par
$a^{4} = a^3+1$\par
$a^{5} = a*a^{4} = a^{4}+a = a^3 + a +1$\par
$a^{6} = a^{3} + a^{2} + a + 1$\par
$a^{7} = a^2 + a + 1$\par
$a^{8} = a^3 + a^2 + a$\par
$a^{9} = a^2 + 1$\par
$a^{10} = a^3 + a$\par
$a^{11} = a^3 + a^2 +1$\par
$a^{12} = a + 1$\par
$a^{13} = a^2 + a$\par
$a^{14} = a^3 + a^2$\par
$a^{15} = a^4 + a^3 = 1$\par
И так как у нас поле $GF(16)$ то $a^{15} = a^{0} = 1$\par
Из порождающего многочлена выразим корни уравнения и подставим их в принятое сообщение\par S(x).\par
$S_1(a) = a^4$\par
$S_2(a^2) = 0$\par
$S_3(a^3) = a^2$\par
$S_4(a^4) = a^2$\par
Теперь строим систему уравнений вида
\begin{gather}
    \begin{pmatrix}
        S_{n}     & S_{n+1} & {...} & S_{n+t-1}  \\
        S_{n+1}   & S_{n+2} & {...} & S_{n+t}    \\
        {.}       & {.}     & {.}   & {.}        \\
        S_{n+t-1} & S_{n+t} & {...} & S_{n+2t-2}
    \end{pmatrix}
    \begin{pmatrix}
        \lambda_{t}   \\
        \lambda_{t-1} \\
        {.}           \\
        \lambda_{1}
    \end{pmatrix}
    =
    \begin{pmatrix}
        S_{n+t}   \\
        S_{n+t+1} \\
        {.}       \\
        S_{n+2t-2}
    \end{pmatrix}
\end{gather}
Где $n=1, t=2$, подставим значения в матрицу и получим:
\begin{gather}
    \begin{pmatrix}
        S_{1} & S_{2} \\
        S_{2} & S_{3}
    \end{pmatrix}
    \begin{pmatrix}
        \lambda_{2} \\
        \lambda_{1}
    \end{pmatrix}
    =
    \begin{pmatrix}
        S_{3} \\
        S_{4}
    \end{pmatrix}
\end{gather}

\begin{gather}
    \begin{pmatrix}
        a^{4} & 0     \\
        0     & a^{2}
    \end{pmatrix}
    \begin{pmatrix}
        \lambda_{2} \\
        \lambda_{1}
    \end{pmatrix}
    =
    \begin{pmatrix}
        a^2 \\
        a^2
    \end{pmatrix}
\end{gather}
Решив систему мы получим что $\lambda_{2} = a^{13}, \lambda_{1} = 1$.\par
$L(x) = 1 + \lambda_{1}x + \lambda_{2}x^2 + ... + \lambda_{t}x^{t}$.
Подставим значения от $1$ до $a^{15}$ и найдем такие значения $a$ что $L(a^{n}) = 0$.
В нашем случае такими значениями являются $a^3$ и $a^{14}$.
Найдем обратные к этим значениям $\gamma_{1} = \frac{a^{15}}{a^{3}} = a^{12}, \gamma_{2} = a$.
\begin{gather}
    \begin{pmatrix}
        {\gamma_{1}}^{s}     & {\gamma_{2}}^{s}     & {...} & {\gamma_{t}}^{s}     \\
        {\gamma_{1}}^{s+1}   & {\gamma_{2}}^{s+1}   & {...} & {\gamma_{t}}^{s+1}   \\
        {.}                  & {.}                  & {.}   & {.}                  \\
        {\gamma_{1}}^{s+t-1} & {\gamma_{2}}^{s+t-1} & {...} & {\gamma_{t}}^{s+t-1}
    \end{pmatrix}
    \begin{pmatrix}
        e_{1} \\
        e_{2} \\
        {.}   \\
        e_{t}
    \end{pmatrix}
    =
    \begin{pmatrix}
        S_{n}   \\
        S_{n+1} \\
        {.}     \\
        S_{n+t-1}
    \end{pmatrix}
\end{gather}
Подставим значения и получим:
\begin{gather}
    \begin{pmatrix}
        a^{12} & a     \\
        a^{24} & a^{2}
    \end{pmatrix}
    \begin{pmatrix}
        e_{1} \\
        e_{2}
    \end{pmatrix}
    =
    \begin{pmatrix}
        a^{4} \\
        0
    \end{pmatrix}
\end{gather}
Используя метод Крамера найдем $e_{1}$ и $e_{2}$:
\begin{gather}
    e_{1}
    =
    \frac{
        \begin{vmatrix}
            a^{4} & a     \\
            0     & a^{2}
        \end{vmatrix}
    }
    {
        \begin{vmatrix}
            a^{12} & a     \\
            a^{24} & a^{2}
        \end{vmatrix}
    }
    =
    \frac{a^{6}}{a^{13}}
    =
    a^{8}
\end{gather}

\begin{gather}
    e_{2}
    =
    \frac{
        \begin{vmatrix}
            a^{12} & a^4 \\
            a^{24} & 0
        \end{vmatrix}
    }
    {
        \begin{vmatrix}
            a^{12} & a     \\
            a^{24} & a^{2}
        \end{vmatrix}
    }
    =
    \frac{-a^{28}}{a^{13}}
    =
    a^{15}
    =
    1
\end{gather}
Найдем многочлен ошибок $E(x) = e_{1}x^{i_{1}} + e_{2}x^{i_{2}} + e_{3}x^{i_{3}} + ... + e_{t}x^{i_{t}}$, $\gamma_{1} = a^{i_{1}}, ... \gamma_{t} = a^{i_{t}} $, $E(x) = a^{8}x^{12} + x$.
Теперь найдем правильный код $V(x) + E(x)$, в нашем случае правильный код $V(x) + E(x) = a^{9}x^{14} + a^{5}x^{13} + a^{11}x^{12} + a^{10}x^{11} + a^{7}x^{9} + a^{5}x^{8} + a^{7}x^{7} + a^{13}x^{6} + a^{3}x^{5} + a^{11}x^{4} + a^{3}x^{3} + a^{10}x^{2} + a^{6}x^{1} + a$ далее используя схему Горнера находим исходное сообщение $A(x)$.
\end{document}
