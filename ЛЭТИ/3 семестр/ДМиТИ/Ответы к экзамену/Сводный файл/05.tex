\documentclass[12pt]{article}
\usepackage[a4paper, portrait, margin=1cm, bottom=2cm]{geometry}
\usepackage{fontspec}
\usepackage[fleqn]{amsmath}
\usepackage{amssymb}
\usepackage{graphicx}
\usepackage{indentfirst}
\usepackage{polyglossia}
\usepackage[dvipsnames]{xcolor}
\usepackage{svg}

\setmainfont[Ligatures=TeX]{Linux Libertine O}
\setdefaultlanguage{russian}
\setotherlanguages{english}
\graphicspath{graphics}

\begin{document}

\section{Простые числа. Основная теорема арифметики.}
\subsection{Определение}
$p \in \mathbb{N}$ простое, если есть 2 натуральных делителя.
\subsection{Определение}
$a \in \mathbb{N}$ составное, если есть $>2$ натуральных делителей
\subsection{Утверждение}
$p$ - простое, $ab$ $\vdots$ $p$ $\Rightarrow$ $a$ $\vdots$ $p$ или $b$ $\vdots$ $p$
\subsubsection{Доказательство}
Пусть $a$ не делится на $p$, НОД($a, p$) = 1 $\Rightarrow$ $b$ $\vdots$ $p$
\subsection{Утверждение}
НОД($a, b, c$) = 1 <=> НОД($a, b$) = 1  и НОД($a, c$) = 1 
\subsubsection{Доказательство}
$|=>/$ НОД($a, b$) = $d$ \par
$a$ $\vdots$ $d$ \par
$bc$ $\vdots$ $b$ $\vdots$ $d$ \par
НОД($a, b, c$) $\vdots$ $d$ \par
$1$ $\vdots$ $d$ \par
$d$ = $1$\par
$|<=|$ НОД($a, b, c) = d$\par
$a$ $\vdots$ $d$ $\vdots$ $f$ \qquad  $a$ $\vdots$ $d$ $\vdots$ $\frac{d}{f}$ \par
$bc$ $\vdots$ $d$ \par
НОД($b, d) = f$ \qquad  $c$ $\vdots$ $\frac{d}{f}$\par
$b$ $\vdots$ $f$ \par
НОД($a, b) \vdots f$ \par
$1$ $\vdots$ $f$ \qquad $\frac{d}{f}$ = $1$\par
$f = 1$ \qquad $d = 1$ \par
\subsection{Утверждение}
НОД($a, b$) = 1 \par
$a$ $\vdots$ $b$, $a$ $\vdots$ $с$ $\Rightarrow$ $a$ $\vdots$ $bc$
\subsubsection{Доказательство}
$a$ $\vdots$ НОК($b, c) = \frac{bc}{gcd(a, b)} = bc$ (gcd = НОД)
\subsection{Утверждение}
НОД(b, c) = 1 => НОД(a, bc) = НОД(a, b)НОД(a, c)
\subsubsection{Доказательсто}
НОД(НОД(a, b), НОД(a, c)) = d \par
$b$ $\vdots$ НОД$(a, b)$ $\vdots$ $d$ \par
$c$ $\vdots$ НОД$(a, c)$ $\vdots$ $d$ \par
НОД$(b, c)$ $\vdots$ $d$ \par
$d = 1$\par
$bc$ $\vdots$ НОД$(a, b)$, $bc$ $\vdots$ НОД$(a, c)$ $\Rightarrow$ $bc$ $\vdots$ НОД$(a, b)$НОД$(a, c)$ \par
$f-$ОД$(a, bc)$ \qquad
$a$ $\vdots$ $f$ \qquad
$bc$ $\vdots$ $f$ \par
НОД$(a, b)$ $=$ $g$, НОД$(\frac{b}{g}, \frac{f}{g})$ $=$ $1$ \qquad $c$ $\vdots$ $\frac{f}{g}$\par
$a$ $\vdots$ $f$ $\vdots$ $g$ \qquad $b$ $\vdots$ $g$ \qquad НОД$(a, b)$ $\vdots$ $g$\par
$a$ $\vdots$ $f$ $\vdots$ $\frac{f}{g}$, $c$ $\vdots$ $\frac{f}{g}\Rightarrow$ НОД(a, c) $\vdots$ $\frac{f}{g}$\par
НОД$(a, b)$НОД$(a, c)$ $\vdots$ $f$
\subsection{Утверждение}
$\forall a \in \mathbb{N},\ a > 1 \Rightarrow \exists \ p$ простое $a\ \vdots \ p$
\subsubsection{Доказательство}
$1,d_1,..., d_k$ - делители a \par
$d_1$ простое $d_1 \ \vdots \ f$ \qquad $a \ \vdots \ d_1 \ \vdots \ f$ \par
\subsection{Утверждение(основная теорема арифметики)}
$\forall a \in \mathbb{N}, \ a>1, \qquad a = p_1^{\alpha_1} ... p_k^{\alpha_k}, \ \ \ \ \alpha_i = 0$
\subsubsection{Доказательство}
$a = a \qquad a=bc \qquad$ $a=bc=p_1^{\beta_1}...p_1^{\beta_1}p_3^{\beta_3}...p_k^{\beta_k}$ \par $a=p_1^{\alpha_k}...p_k^{\alpha_k} = q_1^{\beta_1}...q_l^{\beta_l}$ \par
$p_1 = q_1a=p_2^{\alpha_2}...p_n^{\alpha_n}=q_1^{\beta_1-2}...q_l^{\beta_l}$
\subsection{Утверждение}
$\alpha=p_1^{\gamma_1}...p_k^{\gamma_k} \qquad \beta=$ $p_1^{\delta_1}...p_k^{\delta_k}$\par
$\gamma_i \geq 0 \qquad \delta_i \geq 0 \qquad p_i -$ простое \par
НОД$(\alpha, \beta) =$
$ p_1^{min(\gamma_1, \delta_1)}...p_k^{min(\gamma_k, \delta_k)}$ \par
НОК$(\alpha, \beta)=\prod_{i=1}^kp_i^{max(\gamma_i, \delta_l)}$
\subsubsection{Доказательство}
$] d = \prod_{i=1}^k p_i^{min(\gamma_i, \gamma_i)}$ \par
$\alpha \ \vdots \ d, \  \beta \ \vdots \ d, \ ]d'-$ОД$(\alpha, \beta)$ \par
$\alpha \ \vdots \ q_1^{\epsilon_1}...q_s^{\epsilon_s}$ \par
$]| q_1 \not= p_1, \ q_1\not= p_2,...,q_1\not=p$ \par
$p_1^{\gamma_1}...p_k^{\gamma_k} \ \vdots \ q_1,$ НОД$(p_1^\gamma, q_1) = 1$
$\Rightarrow \ p_2^{\gamma_2}...p_k^{\gamma_k} \ \vdots \ q_1 \Rightarrow$
$p_k^{\gamma_k}\ \vdots \ q_1$ \par
$d' = p_1^{\epsilon_1}...p_k^{\epsilon_k}$ \par
$]| \epsilon_1>\gamma_1 \ \ \ \ \alpha \ \vdots \ d' \ \vdots \ p_1^{\epsilon_1} \ \ \ \ p_1^{\gamma_1} \ \vdots \ p_1^\epsilon$ \par
$p_1^{\gamma_1 - \epsilon} \in \mathbb{Z} \ \ \ \ \gamma_1-\epsilon \geq 0 \ \ \ \ \epsilon_1 \leq \gamma_1 \ \ \ \ \epsilon_1 \leq \delta_1 \ \ \ \ \epsilon_1 \leq min(\gamma_1, \delta_1) \ \ \ \ d' \leq d$ \par
НОД$(\alpha, \beta)$НОК$(\alpha, \beta) = \alpha\beta$ \par
НОК$(\alpha, \beta) = \frac{\alpha\beta}{gcd(\alpha, \beta)}=\prod_{i=1}^k \frac{p_i^{\gamma_i}p_i^{\delta_i}}{min(\gamma_i, \delta_i)}=\prod_{i = 1}^kp_i^{max(\gamma_i, \delta_i)}$
\subsubsection{Пример}
$24 = 2^3*3*5^0$ \par $90 = 2 * 3^2 *5$ \par
НОД$(24, 90) = 2^{min(3, 1)}3 ^{min(1, 2)}5^{min(0, 1)} = 2^1*3^1*5^0=6$
\end{document}