\documentclass[12pt]{article}
\usepackage{polyglossia}
\setdefaultlanguage{russian}
\setotherlanguages{english}
\usepackage{fontspec}
\setmainfont{Times New Roman}
\usepackage{amsmath, amssymb}
\usepackage{graphicx}

\begin{document}

\section{Линейные сравнения}
\textbf{Теорема:}
Если $\gcd(a, b) = d$, то уравнение $a*x\equiv b \pmod{m}$ имеет решение тогда и только тогда, когда $b \vdots d$.\\
\textbf{Доказательство} ---------------------------------------------------------

\vspace{0.3cm} %

$\Rightarrow$ $ax \equiv b \pmod{m}$, то $ax - b \vdots m \vdots d$, так как $ax \vdots d$ и $b \vdots d$.

$\Leftarrow$$\frac{a}{d}x \equiv \frac{b}{d} \pmod{\frac{m}{d}}$

$\gcd\left(\frac{a}{d}, \frac{b}{d}\right) = 1$ и $\left(\frac{a}{d}\right)^{\phi\left(\frac{m}{d}\right)} \equiv 1 \pmod{\frac{m}{d}}$.

\vspace{0.5cm} %
 $\left(\frac{a}{d}\right)^{\phi\left(\frac{m}{d}\right)} x \equiv \left(\frac{a}{d}\right)^{\phi\left(\frac{m}{d}\right) - 1} \frac{b}{d} \pmod{m}$

\vspace{0.5cm} %
$x \equiv \left(\frac{a}{d}\right)^{\phi\left(\frac{m}{d}\right) - 1} \frac{b}{d} \pmod{\frac{m}{d}}$

\vspace{0.5cm} %
$\left(\frac{a}{d}\right)^{\phi\left(\frac{m}{d}\right) - 1} \cdot \frac{b}{d} \equiv d \cdot \left(\frac{a}{d}\right)^{\phi\left(\frac{m}{d}\right)} \cdot \frac{b}{d} \pmod{m}= d\left[k\frac{m}{d} + 1\right]\frac{b}{d} = [km + d]\frac{b}{d} \equiv d\frac{b}{d} \pmod{m} = b$.

\vspace{0.3cm} %
--------------------------------------------------------------------------------

\vspace{0.5cm} %
$1)
ax \equiv b \pmod{m} \\
\gcd(a, m) = 1 $! решение$ \Rightarrow x \equiv a^{\phi(m)-1}b \pmod{m}
$
\vspace{0.5cm} %

$2)
\gcd(a, m) = d ;   b \vdots d  \\
\frac{a}{d} x \equiv \frac{b}{d} \pmod{\frac{m}{d}} \\
x \equiv \left(\frac{a}{d}\right)^{\phi\left(\frac{m}{d}\right) - 1} \frac{b}{d} \pmod{\frac{m}{d}}, \text{ где } \left(\frac{a}{d}\right)^{\phi\left(\frac{m}{d}\right) - 1} \frac{b}{d} = x_0
$
\vspace{0.5cm} %

$d \text{ реш:}\left[ \begin{array}{l}
x \equiv x_0 \pmod{m} \\
x \equiv x_0 + \frac{m}{d} \pmod{m} \\
x \equiv x_0 + (d-1)\frac{m}{d} \pmod{m}
\end{array} \right.
$
\vspace{0.5cm} %

$3)
\gcd(a, m) = d ; b \not\vdots d $   => $\emptyset  \\$

$4)
ax \equiv b \pmod{m} \\$
$ax - b \vdots m$\\
$ax - b = my$\\
$b =ax- my$\\
$x=x_0+t\frac{m}{d}$\\

\textbf{Утв:}
$p$-простое$ , a<p $\\
Решение  $x\equiv \frac{C_p^a}{p}*b*(-1)^k \pmod{m}$,где k = a -1\\
\textbf{Доказательство:}\\ 
$ C_p^a = \frac{p!}{a!(p-a)!}=\frac{p(p-1)....(p-a+1)}{1*2...a} $\\
$\frac{C_p^a}{p}= =\frac{p(p-1)....(p-a+1)}{1*2...(a-1)a}$\\
$(p-1)(p-2)...(p-a+1)\equiv (-1)(-2)....(-a+1) = (-1)^k*1*2....*(a-1)$,где k=a-1\\
$a\frac{C_p^a}{p}\equiv (-1)^k \pmod{p}$,где k=a-1\\
$ab\frac{C_p^a}{p} * (-1)^k\equiv b \pmod{p} $,где k=a-1\\
\\
\textbf{Теорема Критерий Вильсона}
\\
$p$-простое $<=> (p-1)! \equiv -1 \pmod{p} $\\
\textbf{Доказательство:}\\ 
$|=>|  (p-1)!= 1(p-1) $\\
$a^2 \equiv 1 \pmod{p}$\\
$a^2 -1 \vdots p$\\
$(a-1)(a+1) \vdots p$\\
\\
$\left[ \begin{array}{l}
a-1  \equiv p\\
a+1  \equiv p\\
\end{array} \right.
$
\\$\left[ \begin{array}{l}
a \equiv 1 \pmod{p}\\
a \equiv -1 \pmod{p}\\
\end{array} \right.
$
$|<=| p $- не простое $p = mn$\\
$(p-1)!\vdots m $\\
$(p-1)!\equiv -1 \pmod{p}$\\
$(p-1)!+1\vdots m\vdots p $\\


\\
\textbf{Теорема Китайскаяя теорема об остатках}
\\
$m_1,..,m_n $- попарно взаимно простые \\

$
d \text{ имеет ! реш:}\left\{
\begin{aligned}
x \equiv C_1 \pmod{m1}\\
..............................\\
..............................\\
x \equiv C_n \pmod{m_n}\\.
\end{aligned}
\right.
$\\
$x\equiv C\pmod{M} $, где$M = m_1,...m_n$\\

\textbf{Доказательство:}\\ 
$M_i=\frac{M}{m_i}$\\
\\
$\gcd(M_i, m_i) = 1 $\\
\\
$x*M_i\equiv 1\pmod{m_i} => x\equiv a_i\pmod{m_i} $\\
\\
$x = \sum_{i=1}^{n} M_i*a_i*C_i\equiv M_i*a_i*C_i\pmod{m_i}\equiv C_i\pmod{m_i} $\\
$|!| x \not\equiv y \pmod{M} $\\
$x\equiv C_i\pmod{m_i}\equiv y\pmod{m_i}$\\
$x-y\vdots m_i => x-y\vdots M$\\
$x = (C_1,...C_n)$-китайский код числа X\\
$y = (d_1,..d_n)$\\
$x+y = (C_1+d_1,...C_n+d_n)$


\end{document}


