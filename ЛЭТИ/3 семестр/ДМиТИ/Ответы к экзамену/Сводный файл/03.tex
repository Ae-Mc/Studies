\documentclass[12pt]{article}
\usepackage[a4paper, portrait, margin=1cm, bottom=2cm]{geometry}
\usepackage{fontspec}
\usepackage[fleqn]{amsmath}
\usepackage{amssymb}
\usepackage{graphicx}
\usepackage{indentfirst}
\usepackage{polyglossia}
\usepackage[dvipsnames]{xcolor}
\usepackage{svg}

\setmainfont[Ligatures=TeX]{Linux Libertine O}
\setdefaultlanguage{russian}
\setotherlanguages{english}
\graphicspath{graphics}

\begin{document}

\section{Континуанта}
\subsection{Определение}
Введенные понятия цепной дроби и подходящих дробей оказываются
очень полезными для анализа работы алгоритма Евклида. Дадим необходимые обозначения. Рассмотрим трехдиагональный определитель:\par
$\begin{pmatrix}
                  q_0 & 1 & 0 & 0 & ... &0 & 0 \\
                  -1 & q_1 & 1 & 0 & ...& 0 & 0 \\
                  0 & -1 & q_2 & 1 & ... &0 & 0\\
                  ..... & ..... & ..... & ..... & ..... & ..... & .....\\
                  0 & 0 & 0 & 0 & ... & q_{n-2} & 1\\
                  0 & 0 & 0 & 0 & ... & -1 & q_{n-1}
                  \end{pmatrix}$
= $K_n(q_0, q_1, ..., q_{n-1})$

Определитель называют континуантой $n$-го порядка или индекса.


Континуанта индекса $n$ есть многочлен $K_n$($x_1$, ..., $x_n$) определяемый рекуррентным соотношением:

    $K_-1$ = 0,   $K_0$ = 1\par
Разложим континуанту n-го порядка по последнему столбцу:\par
    $K_n$($x_1$, ..., $x_n$) = $x_n K_{n-1}$($x_1$, ..., $x_{n-1}$) + $K_{n-2}$($x_1$, ..., $x_{n-2}$)\par
Cоотношение очень напоминает рекуррентные соотношения для числителей и знаменателей подходящих дробей. Это не случайно и две следующие леммы подтверждают предположение о связи континуант и цепных дробей.
\subsection{Лемма 1}
\textbf{Континуанта $K_n(q_0, q_1, . . . , q_{n−1})$ равна сумме всевозможных произведений элементов $q_0, q_1, . . . , q_{n−1}$, одно из которых содержит
все эти элементы, а другие получаются из него выбрасыванием одной или
нескольких пар сомножителей с соседними номерами (если выброшены все
сомножители, то считаем, что осталась 1).}\par
$Докозательство$. Индукция по $n$. База индукции:\par
$K_1(q_0) = q_0, K_2(q_0, q_1) = $$\begin{pmatrix}
                  q_0 & 1 \\
                  -1 & q_1
                  \end{pmatrix}$
$= q_0q_1 + 1$\par
и утверждение леммы справедливо для континуант первого и второго порядков.
Шаг индукции. Пусть утверждение леммы справедливо для континуант ($n$−2)-го и ($n$−1)-го порядков. Применив разложение , получим
требуемое. \par

Пример\par
$K_6(q_0, q_1, q_2, q_3, q_4, q_5) = q_0 q_1 q_2 q_3 q_4 q_5 + q_2 q_3 q_4 q_5 + q_0 q_3 q_4 q_5 + q_0 q_1 q_4 q_5 + q_0 q_1 q_2 q_5 + q_0 q_1 q_2 q_3 + q_4 q_5 + q_2 q_5 + q_0 q_5 + q_2 q_3 + q_0 q_3 + q_0 q_1 + 1$\par
Явная связь континуант и цепных дробей впервые была установлена
Эйлером.
\subsection{Лемма Эйлера} 
Справедливо тождество\par
$[q_0;q_1,...,q_{n-1}] = \frac{K_n(q_0, q_1, . . . , q_{n−1})}{K_{n-1}(q_1, q_2, . . . , q_{n−1})}$\par

Доказательство. Индукция по n. База индукции:\par
$[q_0;q_1] = q_0 + \frac{1}{q_1} = \frac{q_0q_1 + 1}{q_1} = \frac{K_2(q_0,q_1)}{K_1(q_1)}$\par
Шаг индукции. Пусть тождество верно для дробей с $n$−1 звеном включительно. Представим $n$-звенную дробь $(q_0, q_1, . . . , q_{n−1})$ дробью с $n$−1 звеном, где последнее звено имеет вид $q_{n−2} + \frac{1}{q_{n-1}}$. Применив к этой дроби индукционное предположение, с учетом разложения , имеем\par
$[q_0; q_1, . . . , q_{n−1}] = [q_0, q_1, . . . , q_{n−2} + \frac{1}{q_{n-1}}] = \frac{K_{n-1}(q_0, q_1, . . . , (q_{n-2}+\frac{1}{q_{n-1})})}{K_{n-2}(q_1, q_2, . . . , (q_{n-2}+\frac{1}{q_{n-1})})} = \frac{(\frac{1}{q_{n-1}}) K_{n-2}(q_0,q_1,...,q_{n-3})+K_{n-3}(q_0,q_1,...,q_{n-4})}{(\frac{1}{q_{n-1}}) K_{n-3}(q_1,q_2,...,q_{n-3})+K_{n-4}(q_1,q_2,...,q_{n-4})} = \frac{K_{n-1}(q_0,q_1,...,q_{n-2}) + \frac{K_{n-2}(q_0,q_1,...,q_{n-3})}{q_{n-1}}}{K_{n-2}(q_1,q_2,...,q_{n-2}) + \frac{K_{n-3}(q_1,q_2,...,q_{n-3})}{q_{n-1}}} = \frac{K_{n}(q_0,q_1,...,q_{n-1})}{K_{n-1}(q_1,q_2,...,q_{n-1})}$\par
Перейдем к анализу алгоритма Евклида. Нас будет интересовать наихудший случай — когда алгоритм Евклида работает особенно долго. Сформулируем вопрос точнее: для каких двух наименьших чисел надо применить алгоритм Евклида, чтобы он работал в точности заданное число шагов? Ответ на этот вопрос дает следующая теорема.

\section{Теоремма Ламе}
Пусть $n$ — произвольное натуральное число, и $a > b > 0$ такие, что алгоритму Евклида для обработки $a$ и $b$ необходимо выполнить точно n шагов (делений с остатком), причем a — наименьшее натуральное число с таким свойством. Тогда\par
$fi = $ф так как сайт шлёт нахуй с римскими буквами \par
$a = fi_{n+2}, b = fi_{n+1}$\par
где $fi_k$ — $k-e$ число Фибоначчи.

\textit{Доказательство}. Разложим a/b в цепную дробь. Согласно лемме Эйлера получаем,\par

$\frac{a}{b} = [q_0; q_1,...,q_{n-1}]=\frac{K_{n}(q_0,q_1,...,q_{n-1})}{K_{n-1}(q_1,q_2,...,q_{n-1})}$\par
где $q_0; q_1,...,q_{n-1}$ — неполные частные из алгоритма Евклида. По условию теоремы, их ровно n. Принимая во внимание несократимость подходящих
дробей становится очевидно, что континуанты $q_0; q_1,...,q_{n-1}$ и $q_1; q_2,...,q_{n-1}$ взаимно просты. Пусть
D(a, b) = d. Тогда\par

\begin{equation}
    \begin{cases}
    a = K_{n}(q_0,q_1,...,q_{n-1})
    \\
    b = K_{n-1}(q_1,q_2,...,q_{n-1})
    \end{cases}
\end{equation}

В силу единственности разложения в цепную дробь, в
случае a > b > 0 справедливы неравенства $q_0,q_1,...,q_{n-2} > 1, q_{n−1} > 2$.
Очевидно, что d > 1. По лемме 1, континуанта есть многочлен с неотрицательными коэффициентами от переменных $q_0,q_1,...$. Его минимальное
значение очевидно достигается при $q_0 = q_1 = · · · = q_{n−2} = 1, q_{n−1} = 2$. Положив d = 1 и подставив эти значения $q_i$, получим требуемое.

\end{document}