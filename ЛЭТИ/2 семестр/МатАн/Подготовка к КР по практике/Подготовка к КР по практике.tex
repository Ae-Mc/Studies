\documentclass[12pt]{article}
\usepackage[fleqn]{amsmath}
\usepackage{amssymb}
\usepackage{enumerate}
\usepackage{enumitem}
\usepackage{fontspec}
\usepackage[a4paper,margin=1cm,bottom=2cm]{geometry}
\usepackage{indentfirst}
\usepackage{mathtools}
\usepackage{multicol}
\usepackage{polyglossia}

\setmainfont[Ligatures=TeX]{Linux Libertine}
\setdefaultlanguage{russian}
\setotherlanguages{english}

\title{Подготовка к контрольной работе на получение 4-5 по МатАнализу}
\author{Студент группы 2305 Макурин Александр}
\date{3 июня 2023}

\begin{document}
\maketitle

\section{Определение неопределённого интеграла}
Неопределённым интегралом от функции $f(x)$ называется семейство первообразных $\{F(x) + C\}$. Первообразной функции $f(x)$ называется такая функция $F(x)$, что $F'(x) = f(x)$.

\section{Свойства неопределённого интеграла}
\begin{enumerate}
    \item $(\int{f(x)dx})' = f(x)$
    \item $\int{f'(x)dx} = f(x) + C$
    \item $\int Af(x)dx = A\int f(x)dx$, $A$ — const.
    \item $\int (f(x) \pm g(x))dx = \int f(x)dx \pm \int g(x)dx$
\end{enumerate}

\section{Формулы для основных неопределённых интегралов}
\setlength{\columnsep}{1cm}
\begin{multicols}{2}
    \begin{enumerate}
        \item $\displaystyle \int x^ndx = \dfrac{x^{n + 1}}{n + 1} + C$, $n \neq -1$
        \item $\displaystyle \int \dfrac{dx}{x} = \ln{|x|} + C$
        \item $\displaystyle \int \ln{x}dx = x\ln{x} - x + C$
        \item $\displaystyle \int e^xdx = e^x + C$
        \item $\displaystyle \int a^xdx = \dfrac{a^x}{\ln{a}} + C$
        \item $\displaystyle \int \cos{x}dx = \sin{x} + C$
        \item $\displaystyle \int \sin{x}dx = - \cos{x} + C$
        \item $\displaystyle \int \dfrac{dx}{\cos^2{x}} = \tan{x} + C$
        \item $\displaystyle \int \dfrac{dx}{\sin^2{x}} = -\cot{x} + C$
        \item $\displaystyle \int \dfrac{dx}{x^2 + a^2} = \dfrac{1}{a}\arctan{\dfrac{x}{a}} + C$
        \item $\displaystyle \int \dfrac{dx}{\sqrt{a^2 - x^2}} = \arcsin{\dfrac{x}{a}} + C$
        \item $\displaystyle \int \cosh{x}dx = \sinh{x} + C$
        \item $\displaystyle \int \sinh{x}dx = \cosh{x} + C$
        \item $\displaystyle \int \dfrac{dx}{\cosh^2{x}} = \tanh{x} + C$
        \item $\displaystyle \int \dfrac{dx}{\sinh^2{x}} = -\coth{x} + C$
        \item $\displaystyle \int \dfrac{dx}{\sqrt{x^2 \pm a^2}} = \ln{|x + \sqrt{x^2 \pm a^2}|} + C$
        \item $\displaystyle \int \dfrac{dx}{x^2 - a^2} = \dfrac{1}{2a}\ln{\left|\dfrac{x - a}{x + a}\right|} + C$
    \end{enumerate}
\end{multicols}

\section{Достаточные условия интегрируемости}
Функция $f(x)$ интегрируема на отрезке $[a,b]$, если она непрерывна на этом отрезке.

\section{Формула замены переменной в неопределённом интеграле (строгая формулировка)}
Пусть $g(t)$ непрерывна и дифференцируема, $g(t) = x$. Тогда:
\[
    \int f(x) dx = \left[\begin{array}{c}
            g(t) = x \\
            d(g(t)) = dx
        \end{array}\right] = \int f(g(t)) d(g(t)) = \int f(g(t))g'(t)dt
\]

\section{Формула интегрирования по частям в неопределённом интеграле (строгая
  формулировка). Примеры}
Обозначим:
\[
    u = U(x); \ v = V(x)
\]
\[
    (uv)' = u'v + uv'
\]
\[
    \int (uv)'dx = \int u'vdx + \int uv'dx \xRightarrow[u'dx = du]{v'dx = dv} uv + C = \int v du + \int u dv
\]

Так как константы будут и в результатах двух оставшихся интегралов, $C$ можно сократить. В результате получим:

\[
    \int udv = uv - \int vdu
\]

Примеры:
\begin{itemize}
    \item $\displaystyle \int xe^{x}dx = \left[\begin{array}{ll}
                      u = x        & du = dx              \\
                      dv = e^{x}dx & v = \int e^xdx = e^x
                  \end{array}\right] = xe^x - \int e^xdx = e^x(x - 1) + C$
\end{itemize}

\section{Вывод рекурсивной формулы для интеграла $\int \dfrac{dx}{(x^2 + a^2)^n}$  с помощью метода интегрирования по частям}
\[
    I_n = \int \dfrac{dx}{(x^2 + a^2)^n}
\]
\[
    \int \dfrac{dx}{(x^2 + a^2)^n} = \dfrac{1}{a^2} \int \dfrac{x^2 + a^2 - x^2}{(x^2 + a^2)^n}dx
    = \dfrac{1}{a^2}\left(\int \dfrac{dx}{(x^2 + a^2)^{n-1}} - \int \dfrac{x^2dx}{(x^2 + a^2)^n}\right)
    = \dfrac{1}{a^2}\left(I_{n-1} - \int \dfrac{x^2dx}{(x^2 + a^2)^n}\right)
\]
\[
    \begin{aligned}
         & \dfrac{1}{2}\int \dfrac{xd(x^2 + a^2)}{(x^2 + a^2)^n} =
        \left[\begin{array}{ll}
                      u = x                                    & du = dx                                   \\
                      dv = \dfrac{d(x^2 + a^2)}{(x^2 + a^2)^n} & v = \dfrac{1}{(x^2 + a^2)^{n - 1}(1 - n)} \\
                  \end{array}\right] =                                        \\
         & = \dfrac{x}{2(x^2 + a^2)^{n-1}(1-n)} - \dfrac{1}{2(1 - n)}\int \dfrac{dx}{(x^2 + a^2)^{n - 1}} = \dfrac{x}{2(x^2 + a^2)^{n-1}(1-n)} - \dfrac{I_{n-1}}{2 - 2n}
    \end{aligned}
\]
\[
    \begin{aligned}
        \int \dfrac{dx}{(x^2 + a^2)^n} & = \dfrac{1}{a^2}\left(I_{n-1} - \dfrac{I_{n-1}}{2n-2} + \dfrac{x}{(x^2 + a^2)^{n-1}(2n-2)}\right) = \\
                                       & = \dfrac{1}{a^2}\left(I_{n-1}\dfrac{2n - 3}{2n - 2} + \dfrac{x}{(x^2 + a^2)^{n-1}(2n-2)}\right)
    \end{aligned}
\]
\[
    I_n = \dfrac{1}{a^2(2n - 2)}\left(I_{n-1}(2n - 3) + \dfrac{x}{(x^2 + a^2)^{n-1}}\right)
\]

\section{Зацикливающиеся интегралы типа $\int e^{ax}\sin{bx}dx$, $\int e^{ax}\cos{bx}dx$ (формулы и вывод)}
\[
    \begin{aligned}
        \int e^{ax}\sin{bx}dx & =
        \left[\begin{array}{ll}
                      u = \sin{bx}  & du = b\cos{bx}dx       \\
                      dv = e^{ax}dx & v = \dfrac{1}{a}e^{ax}
                  \end{array}\right]
        = \dfrac{1}{a}e^{ax}\sin{bx} - \dfrac{b}{a}\int e^{ax}\cos{bx}dx =                                                           \\
                              & =
        \left[\begin{array}{ll}
                      u = \cos{bx}  & du = -b\sin{bx}dx      \\
                      dv = e^{ax}dx & v = \dfrac{1}{a}e^{ax}
                  \end{array}\right]                                                                                 \\
                              & = \dfrac{1}{a}e^{ax}\sin{bx} - \dfrac{b}{a^2}e^{ax}\cos{bx} - \dfrac{b^2}{a^2} \int e^{ax}\sin{bx}dx
    \end{aligned}
\]
\[
    \int{e^{ax}\sin{bx}dx} = \dfrac{e^{ax}}{a^2 + b^2}\left(a\sin{bx} - b\cos{bx}\right)
\]

\[
    \begin{aligned}
        \int e^{ax}\cos{bx}dx & =
        \left[\begin{array}{ll}
                      u = \cos{bx}  & du = -b\sin{bx}dx      \\
                      dv = e^{ax}dx & v = \dfrac{1}{a}e^{ax}
                  \end{array}\right]
        = \dfrac{1}{a}e^{ax}\cos{bx} + \dfrac{b}{a}\int e^{ax}\sin{bx}dx =                                                           \\
                              & =
        \left[\begin{array}{ll}
                      u = \sin{bx}  & du = b\cos{bx}dx       \\
                      dv = e^{ax}dx & v = \dfrac{1}{a}e^{ax}
                  \end{array}\right] =                                                                                 \\
                              & = \dfrac{1}{a}e^{ax}\cos{bx} + \dfrac{b}{a^2}e^{ax}\sin{bx} - \dfrac{b^2}{a^2} \int e^{ax}\cos{bx}dx
    \end{aligned}
\]
\[
    \int{e^{ax}\cos{bx}dx} = \dfrac{e^{ax}}{a^2 + b^2}\left(a\cos{bx} + b\sin{bx}\right)
\]

\section{Интегрирование рациональных функций. Типы элементарных интегралов. Сведение общего случая к элементарным (в общем виде)}
Все интегралы рациональных функций посредством разложения на простейшие сводятся к 4 типам элементарных интегралов:
\begin{enumerate}[label=\Roman*]
    \item $\displaystyle \int \dfrac{A}{x - a}$
    \item $\displaystyle \int \dfrac{A}{(x - a)^n}$
    \item $\displaystyle \int \dfrac{Bx + C}{x^2 + bx + c}$
    \item $\displaystyle \int \dfrac{Bx + C}{(x^2 + bx + c)^n}$
\end{enumerate}

Разложение на простейшие — рациональная дробь $\dfrac{P(x)}{Q(x)}$ представляется в виде $\displaystyle G(x) + \sum^n_{i=1}\dfrac{P_i(x)}{Q_i(x)}$, где $G(x)$ — многочлен, выделенный из $\dfrac{P(x)}{Q(x)}$, а $\dfrac{P_i(x)}{Q_i(x)}$ — простейшие дроби.

Дробь называется простейшей, если её знаменатель представляет собой степень некоторого неприводимого многочлена, а числителем является многочлен степени меньше, чем степень неприводимого многочлена.

Простейшие дроби находятся по методу неопределённых коэффициентов.

Любой многочлен можно свести к виду:
\[
    Q(x) = (Q_1(x))^{m_1}(Q_2(x))^{m_2}...(Q_n(x))^{m_n}
\]
где $Q_i(x)$ — неприводимые многочлены.

Тогда \textit{правильная} рациональная дробь $\dfrac{F(x)}{Z(x)}$, где $Z(x)$ раскладывается в $\displaystyle \prod_{i=1}^n(Z_i(x))^{m_i}$ и может быть представлена как:
\begin{equation}
    \dfrac{F(x)}{Z(x)} = \sum^n_{i=1} \sum^{m_i}_{j=1} \dfrac{H_{ij}(x)}{(Z_i(x))^{j}} \label{eq:1}
\end{equation}
где
\[
    H_{ij}(x) = \begin{cases}
        A \text{, если $Z_i = ax - b$} \\
        Ax + B \text{, если $Z_i = ax^2 + bx + c$}
    \end{cases}
\]

Соответственно, разложение на простейшие сводится к поиску всех коэффициентов $a_{ik}$. Так как для равенства двух многочленов необходимо, чтобы коэффициенты при соответствующих степенях совпадали, можно домножить обе стороны уравнения \eqref{eq:1} на $Z(x)$ и получить:

\begin{equation}
    F(x) = \sum^n_{i=1} \sum^{m_i}_{j = 1} H_{ij}(x)\prod_{k \neq i}^{n}(Z_k(x))^{m_k}
\end{equation}

\section{Метод интегрирования иррациональных функций (общая идея). Конкретные подстановки (с доказательством рационализации исходного интеграла после выполненной замены)}
Основные виды иррациональных выражений и соответствующие подстановки (замены):
\begin{enumerate}
    \item $R\left(x, \sqrt[n]{\dfrac{ax + b}{cx + d}}\right)$ \\
          Замена $t = \sqrt[n]{\dfrac{ax + b}{cx + d}}$
    \item $R\left(x, \left(\dfrac{ax + b}{cx + d}\right)^\dfrac{1}{S_1}, \left(\dfrac{ax + b}{cx + d}\right)^\dfrac{1}{S_2}, ..., \left(\dfrac{ax + b}{cx + d}\right)^\dfrac{1}{S_n}\right)$ \\
          Замена $t = \sqrt[\lambda]{\dfrac{ax + b}{cx + d}}$, где $\lambda$ — общее кратное $S_1, S_2, ..., S_n$.
    \item $x^m(a + bx^n)^p$, где $m, n, p \in \mathbb{Q}$.
          Тогда возможны варианты:
          \begin{enumerate}[label*=\arabic*]
              \item $p \in \mathbb{Z} \Rightarrow t = \sqrt[\lambda]{x}$, где $\lambda$ — общее кратное знаменателей $m, n$.
              \item $p \notin \mathbb{Z}, \dfrac{m + 1}{n} \in \mathbb{Z} \Rightarrow t = \sqrt[\lambda]{a + bx^n}$, где $\lambda$ — знаменатель $p$.
              \item $p \notin \mathbb{Z}, \dfrac{m + 1}{n} \notin \mathbb{Z}, p + \dfrac{m + 1}{n} \in \mathbb{Z} \Rightarrow t = \sqrt[\lambda]{ax^{-n} + b}$, где $\lambda$ — знаменатель $p$.
              \item Иначе, интеграл не берущийся.
          \end{enumerate}
\end{enumerate}

\section{Тригонометрические подстановки в иррациональных интегралах (интегралы типа $\int \sqrt{x^2 - a^2}dx$)}
\[
    \begin{aligned}
         & x = \dfrac{a}{\sin t}                                                                   \\
         & dx = -\dfrac{\cos t}{\sin^2t}dt                                                         \\
         & \sqrt{x^2 - a^2} = \sqrt{\dfrac{a^2 - a^2\sin^2 t}{\sin^2 t}} = \dfrac{a\cos t}{\sin t}
    \end{aligned}
\]

\section{Интегрирование тригонометрических функций (общая идея). Возможные подстановки с обоснованием их применения}
Общая идея состоит в применении тригонометрических преобразований. Либо, использование тригонометрических замен:
\begin{enumerate}
    \item В самом общем виде ($\int R(\sin x, \cos x)dx$) используется универсальная тригонометрическая подстановка
          \begin{align*}
              t      & = \tan \dfrac{x}{2}        \\
              \sin x & = \dfrac{2t}{1 + t^2}      \\
              \cos x & = \dfrac{1 - t^2}{1 + t^2} \\
              dx     & = \dfrac{2}{1 + t^2}dt
          \end{align*}
    \item Если $R(-\sin x, \cos x) = -R(\sin x, \cos x)$, тогда можно взять более удобную подстановку
          \begin{align*}
              \cos x & = t                           \\
              \sin x & = \sqrt{1 - t^2}              \\
              dx     & = -\dfrac{dt}{\sqrt{1 - t^2}}
          \end{align*}
    \item Если $R(\sin x, -\cos x) = -R(\sin x, \cos x)$, тогда можно взять более удобную подстановку
          \begin{align*}
              \sin x & = t                          \\
              \cos x & = \sqrt{1 - t^2}             \\
              dx     & = \dfrac{dt}{\sqrt{1 - t^2}}
          \end{align*}
    \item Если $R(-\sin x, -\cos x) = R(\sin x, \cos x)$, тогда можно взять более удобную подстановку
          \begin{align*}
              t      & = \tan x                    \\
              \cos x & = \dfrac{1}{\sqrt{t^2 + 1}} \\
              \sin x & = \dfrac{t}{\sqrt{t^2 + 1}} \\
              dt     & =  (1 - \tan^2 x)dx         \\
              dx     & = \dfrac{dt}{1 - t^2}
          \end{align*}
\end{enumerate}

\section{Неберущиеся интегралы}
Неберущийся интеграл — интеграл, который не выражается через элементарные функции.

Интеграл $\int f(x)dx = F(x) + C$ — неберущийся, если $F(x)$ не относится к элементарным функциям.

\section{Определение определённого интеграла}

$f(x)$ — непрерывна на $[a, b]$, $f(x) \geq 0$. Разобъём $[a, b]$ на $n$ частей. $a = x_0 < x_1 < x_2 < ... < x_{n - 1} < x_n = b$. $[x_{i - 1}, x_i], i = 1, ..., n$. Зафиксируем точку $\xi_i \in (x_{i - 1}, x_i)$. $\displaystyle S_n = \sum^n_{i = 1} f(\xi_i)(x_i - x_{i - 1})$ — сумма Римана. $\displaystyle \tau = \max_{1 \leq i \leq n} (x_i - x_{i-1})$ — меткость разбиения. Тогда:
\[
    \int_a^b f(x)dx = \lim_{
        \substack{
            n \rightarrow \infty \\
            \tau \rightarrow 0
        }} S_n
\]

\[
    \int_a^b f(x)dx = - \int_b^a f(x)dx
\]

\section{Свойства определённого интеграла}
\begin{enumerate}
    \item Если функция терпит конечное количество разрывов первого рода в точках $x_1 < x_2 < x_3 < ... < x_{n - 1} < x_n$, лежащих на отрезке $[a, b]$ то:
          \[
              \int_a^b f(x)dx = \int_a^{x_1} f(x)dx + \sum^{n}_{i = 2} \int_{x_{i - 1}}^{x_i}f(x)dx
          \]
    \item $\forall c \in [a, b]$
          \[
              \int_{a}^{b}f(x)dx = \int_a^cf(x)dx + \int_c^bf(x)dx
          \]
    \item $\forall c \in D(f)$
          \[
              \int_{a}^{b}f(x)dx = \int_a^cf(x)dx + \int_c^bf(x)dx
          \]
    \item $f(x), g(x)$ — непрерывны на $[a, b]$. Тогда:
          \[
              \int_a^b (f(x) \pm g(x))dx = \int_a^b f(x)dx \pm \int_a^b g(x)dx
          \]
    \item $A$ — const
          \[
              \int_a^b Af(x)dx = A\int_a^b f(x)dx
          \]
    \item Пусть $f(x) \geq 0$, $x \in [a, b]$. Тогда:
          \[
              \int_a^b f(x)dx \geq 0
          \]
    \item Теорема о среднем:
          \[
              \left\{\begin{array}{l}
                  \displaystyle f(x) \text{ непрерывна на } [a, b] \\
                  \displaystyle m = \min_{a \leq x \leq b} f(x)    \\
                  \displaystyle M = \max_{a \leq x \leq b} f(x)    \\
              \end{array}\right|
              \Rightarrow
              \begin{array}{l}
                  \displaystyle m(b - a) \leq \int_a^b f(x)dx \leq M(b - a) \\
                  \displaystyle \exists \xi \in [a, b] : f(\xi)(b - a) = \int_a^b f(x)dx
              \end{array}
          \]
    \item $f(x)$ — непрерывна на $[a, b]$
          \[
              \left|\int_a^b f(x)dx \right| \leq \int_a^b f(x)dx
          \]
\end{enumerate}

\section{Формула Ньютона-Лейбница (строгая формулировка)}
\[
    \Phi(x) = \int_a^x f(t)dt
\]
\[
    \begin{array}{ll}
        \Phi'(x) & = \lim_{\Delta x \rightarrow 0} \dfrac{\Phi(x + \Delta x) - \Phi(x)}{\Delta x}
        = \lim_{\Delta x \rightarrow 0} \dfrac{\int_a^{x + \Delta x} f(t)dt - \int_a^x f(t)dt}{\Delta x} = \\
                 & = \lim_{\Delta x \rightarrow 0} \dfrac{\int_x^{x + \Delta x}f(t) dt}{\Delta x} = f(x)
    \end{array}
\]
Следовательно, $\Phi(x)$ — первообразная $f(x)$.
\[
    \Phi(x) + C = \int f(x)dx
\]
\[
    \int_a^xf(t)dt + C = \int f(x)dx
\]
\[
    \Phi(a) = \int_a^a f(t)dt = 0
\]
\[
    \Phi(b) = \int_a^b f(t)dt
\]
Отсюда:
\[
    \int_a^b f(x)dx = \Phi(b) - \Phi(a)
\]

\section{Обоснование существования первообразной у некоторого класса функций}
Все непрерывные функции имеют первообразные.

Определенный интеграл с переменным верхним пределом является одной из первообразных для непрерывной подынтегральной функции.
\[
    \Phi(x) = \int_a^x f(t)dt
\]
\[
    \begin{array}{ll}
        \Phi'(x) & = \lim_{\Delta x \rightarrow 0} \dfrac{\Phi(x + \Delta x) - \Phi(x)}{\Delta x}
        = \lim_{\Delta x \rightarrow 0} \dfrac{\int_a^{x + \Delta x} f(t)dt - \int_a^x f(t)dt}{\Delta x} = \\
                 & = \lim_{\Delta x \rightarrow 0} \dfrac{\int_x^{x + \Delta x}f(t) dt}{\Delta x} = f(x)
    \end{array}
\]
Что и требовалось доказать.

\section{Формула замены переменной в определённом интеграле (строгая формулировка)}
Пусть $\phi(t)$ — непрерывна и дифференцируема на $[a,b]$. $\phi([\alpha, \beta]) = [a, b]$. $F(x)$ — первообразная $f(x)$. Очевидно, что $F(\phi(t))$ является первообразной для выражения $f(\phi(t))\phi'(t)$.

\[
    \int_\alpha^\beta f(\phi(t))d(\phi(t)) = \int_\alpha^\beta f(\phi(t))\phi'(t)dt = F(\phi(\beta)) - F(\phi(\alpha)) = F(b) - F(a) = \int_a^b f(x)dx
\]

Отсюда:
\[
    \int_a^b f(x)dx = \int_{\alpha}^{\beta} f(\phi(t))\phi'(t)dt, \text{ где } \alpha
    = \phi^{-1}(a), \beta = \phi^{-1}(b)
\]

\section{Формула интегрирования по частям в определенном интеграле (строгая формулировка)}
Пусть функции $U(x), V(x)$ непрерывны и дифференцируемы на $[a, b], u = U(x), v = V(x)$. Тогда:
\[
    \int_a^b udv = uv\bigg|_a^b - \int_a^b vdu, \text{ где } uv|_a^b
    = U(b)V(b) - U(a)V(a)
\]

Доказательство:
\[
    d(U(x)V(x)) = U(x)d(V(x)) + V(x)d(U(x)) = U(x)V'(x)dx + V(x)U'(x)dx
\]
\[
    \int_a^b d(U(x)V(x)) = \int_a^b (U(x)V'(x) + V(x)U'(x))dx
\]
\[
    \int_a^b d(U(x)V(x)) = U(x)V(x)\bigg\rvert_a^b = uv\bigg\rvert_a^b
\]
\[
    \int_a^b udv = uv\bigg\rvert_a^b - \int_a^b vdu
\]
Что и требовалось доказать.

\section{Геометрические приложения определённого интеграла (вычисление площадей в декартовых и полярных координатах)}
В декартовых координатах ($S$ — площадь между графиком функции $f(x)$ и осью $O_x$ на промежутке $[a, b]$):
\[
    S = \int_a^b |f(x)|dx
\]

Вычисление площади в декартовых координатах, заданных параметрически:
\[
    S = \int_{t_1}^{t_2}y(t)x'(t)dt
\]

В полярных координатах ($S$ — площадь криволинейного сектора между лучами $\alpha$ и $\beta$, точкой $(0, 0)$ и графиком $r(\phi)$):
\[
    S = \dfrac{1}{2}\int_\alpha^\beta r^2(\phi)d\phi
\]

\section{Геометрические приложения определённого интеграла (вычисление длин дуг кривых в декартовых и полярных координатах)}
Длина кривой в декартовых координатах, заданных параметрически. Пусть задано $n$-мерное декартово пространство, в которой задана кривая:
\[
    \left\{\begin{array}{rcl}
        x_1 & =      & x_1(t) \\
        x_2 & =      & x_2(t) \\
            & \vdots &        \\
        x_n & =      & x_n(t)
    \end{array}\right.
\]

Тогда длина кривой:
\[
    l = \int_{t_1}^{t_2}\sqrt{\sum_{i=1}^n \left(\left(x_i\right)'_t\right)^2}dt
\]

В полярных координатах (легко выводится через составление уравнений, выражающих полярные в декартовых):
\[
    l = \int_{\phi_1}^{\phi_2} \sqrt{(r'(\phi))^2 + r^2(\phi)}d\phi
\]

На плоскости при $y = f(x)$:
\[
    l = \int_a^b \sqrt{1 + (f'(x))^2}dx
\]

\section{Геометрические приложения определённого интеграла (вычисление объемов тел вращения)}
Вращение вокруг оси $O_x$ при $y = f(x)$:
\[
    V_x = \pi\int_a^bf^2(x)dx
\]

\section{Несобственные интегралы от неограниченных функций. Определение, способы вычисления}
Пусть функция $f(x)$ определена на $(a, b]$ и терпит бесконечный разрыв в точке $x = a$. Тогда $\int_a^b f(x)dx$ называется несобственным интегралом и вычисляется как:
\[
    \int_a^b f(x)dx = \lim_{c \rightarrow a + 0} \int_c^b f(x)dx
\]

Пусть функция $f(x)$ определена на $[a, b)$ и терпит бесконечный разрыв в точке $x = b$. Тогда $\int_a^b f(x)dx$ называется несобственным интегралом и вычисляется как:

                \[
                    \int_a^b f(x)dx = \lim_{c \rightarrow b - 0} \int_a^c f(x)dx
                \]

                Пусть функция $f(x)$ терпит бесконечный разрыв в точке $x = c$, $c \in (a, b)$ тогда $\int_a^b f(x)dx$ называется несобственным интегралом и вычисляется как:
                \[
                    \int_a^b f(x)dx = \int_a^c f(x)dx + \int_c^b f(x)dx
                \]

                \section{Несобственные интегралы от неограниченных функций. Признаки сходимости}
                Если результат вычисления несобственного интеграла является конечным, то про такой несобственный интеграл говорят, что он сходится. Иначе говорят, что он расходится.

                Пусть $f(x)$ определена на $(a, b]$ и терпит бесконечный разрыв в точке $x = a$. Тогда $\displaystyle \int_a^b f(x)dx$ сходится, если $\displaystyle \lim_{c \rightarrow a + 0} \int_c^b f(x)dx$ существует и конечен.

Пусть $f(x)$ определена на $[a, b)$ и терпит бесконечный разрыв в точке $x = b$. Тогда $\displaystyle \int_{a}^{b} f(x)dx$ сходится, если $\displaystyle \lim_{c \rightarrow b - 0} \int_{a}^{c} f(x)dx$ существует и конечен.

                Пусть $f(x)$ терпит бесконечный разрыв в точке $c \in [a, b]$ и определена в остальных точках $[a, b]$. Тогда $\displaystyle \int_{a}^{b} f(x)dx$ сходится, если $\displaystyle \int_{a}^{c} f(x)dx$ и $\displaystyle \int_c^b f(x)dx$ сходятся.

                \subsection{Критерий Коши}
                Пусть $f(x)$ определена на $(a, b]$ и терпит бесконечный разрыв в точке $x = a$. Тогда $\int_a^b f(x)dx$ сходится, если:
\[
    \forall \varepsilon > 0 \Rightarrow \exists \delta(\varepsilon) \Rightarrow \forall(0 < \delta_1 < \delta_2 < \delta) \Rightarrow \left|\int_{a + \delta_1}^{a + \delta_2}f(x)dx\right| < \varepsilon
\]

\section{Интегралы с бесконечными пределами интегрирования. Определение, способы вычисления}
Интегралы вида $\displaystyle \int_a^{\infty}f(x)dx, \int_{-\infty}^bf(x)dx, \int_{-\infty}^\infty f(x)dx$ называются интегралами с бесконечными пределами интегрирования. Вычисление таких интегралов сводится к вычислению пределов:
\begin{align*}
     & \int_a^{\infty}f(x)dx = \lim_{b \rightarrow \infty} \int_a^bf(x)dx   \\
     & \int_{-\infty}^bf(x)dx = \lim_{a \rightarrow \infty} \int_a^bf(x)dx  \\
     & \int_{-\infty}^{\infty}f(x)dx = \lim_{\substack{a\rightarrow -\infty \\ b \rightarrow \infty}} \int_a^bf(x)dx  \\
\end{align*}

\section{Интегралы с бесконечными пределами интегрирования. Признаки сходимости}

\section{Абсолютная сходимость несобственных интегралов. Условная сходимость несобственных интегралов}

\section{Дифференциальные уравнения. Определение. Основные типы ДУ}

\section{Решение ДУ. Общее решение, частное решение}

\section{Задача Коши}

\section{ДУ с разделяющимися переменными. Общий вид, метод решения}

\section{Однородные ДУ. Общий вид, метод решения}

\section{Линейные ДУ первого порядка. Общий вид. Метод решения}

\section{Уравнение Бернулли. Общий вид, метод решения}

\section{Линейные дифференциальные уравнения с постоянными коэффициентами (ЛДУ).  Общий вид}

\section{Решение ЛДУ с постоянными коэффициентами и специальной правой частью}

\section{Числовые ряды. Вычисление суммы. Необходимое условие сходимости}

\section{Признаки сходимости положительных числовых рядов}

\section{Знакочередующиеся ряды. Признак Лейбница}

\section{Абсолютная и условная сходимость числовых рядов}
\end{document}
