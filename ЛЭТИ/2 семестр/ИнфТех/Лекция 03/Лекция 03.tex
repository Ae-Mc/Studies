\documentclass{article}
\usepackage[a4paper, portrait, margin=1cm, right=1cm]{geometry}
\usepackage{fontspec}
\usepackage[fleqn]{amsmath}
\usepackage{setspace}
\usepackage{graphicx}

\graphicspath{./graphics/}
\setmainfont[Ligatures=TeX]{Linux Libertine}

\title{Информационные технологии. Лекция 02. Свойства КФС. Основные компоненты КФС.}
\author{Студент группы 2305 Макурин Александр}
\date{13 февраля 2023}

\begin{document}
\maketitle
\begin{sloppypar}
    \setstretch{1.8}
    \section{Схема организации КФС}
    \includegraphics*[width=0.8\textwidth]{./graphics/Схема организации КФС.jpg}

    Механическая часть делится на автопилот (он же простейший вычислитель) с приводами и целевую нагрузку. И то, и другое представляет собой вычислительные устройства:

    \setstretch{1}
    \begin{itemize}
        \item Механика
              \begin{itemize}
                  \item Автопилот + приводы
                  \item Целевая нагрузка
              \end{itemize}
    \end{itemize}

    Система: $E = <Phys, Inf, Act, Sens, Humans>$. При этом люди ($Humans$) делятся на:
    \begin{itemize}
        \item ЛПР — лица, принимающие решения
        \item Stakeholder — лица, заинтересованные в результате работы системы,
              выготопреобретатели
        \item $Env$ — окружающая среда. Люди, влияющие на работу системы, но
              не связанные с ней напрямую
    \end{itemize}

    \setstretch{1.8}

    Функциональная единица: $F=<Ph, Inf>$. Это то, за счёт чего производится воздействие.

    Тензор: $Tr = <Act, Sens, Humans>$. Это то, что воздействует на систему.

    Отсюда, система: $E = <F, Tr>$.

    $func: E \rightarrow Service$, где $Service$ — это пространство услуг.

    $user: Service \rightarrow Consumer$, где $Consumer$ — потребитель услуги.

    БАС (беспилотно-авиационная система) делится на:
    \begin{itemize}
        \item НПУ (Наземный центр управления)
        \item БВС (беспилотное воздушное судно). После повышения уровня автономности, БВС называют БПЛА (беспилотным летательным аппаратом).
    \end{itemize}

    $\Delta \rho^{phy} \rightarrow 0$ — пространственно нераспределённые КФС

    $\Delta \rho^{phy} \rightarrow \infty$ — пространственно распределённые КФС

    Централизованные системы управления постепенно сменяются децентрализованными.


    Устройство централизованной СУ:
    \begin{itemize}
        \item ГСУ (глобальная СУ)
              \begin{itemize}
                  \item УСУ (узел субсидиарного управления). Одна ГСУ может быть связана с несколькими УСУ.
                        \begin{itemize}
                            \item ЛВУ (локальные вычислительные устройства). Один УСУ может быть связан с несколькими ЛВУ. Представляет собой объединение сенсоров и/или механики.
                        \end{itemize}
              \end{itemize}
    \end{itemize}

    Устройство децентрализованной СУ:
    \begin{itemize}
        \item НПУ $\Leftrightarrow$ НПУ — множество НПУ, коммуницирующих между собой.
              \begin{itemize}
                  \item ЛВУ. Один НПУ может быть связан с несколькими ЛВУ. При этом к ЛВУ может быть привязано одно БВС.
              \end{itemize}
    \end{itemize}

    \section{Навигация}

    Навигация — определение местоположения, ориентации и скорости движения объекта. \\
    Positioning — позиционирование, местоопределение, определение местоположения.

    Постановка задачи:
    \begin{itemize}
        \item Робот находится в неизвестном месте
        \item Необходимо установить его местоположение
    \end{itemize}

    Решения:
    \begin{itemize}
        \item Глобальная навигация
        \item Локальная навигация
    \end{itemize}

    \subsection{Глобальная навигация}

    \subsubsection{Система наземного базирования}
    До спутниковой системы навигации была попытка создать радионавигационную систему наземного базирования — Loran (LOng RAnge Navigation). Принци работы — импульсно-фазовый. Имела погрешность определения положения в несколько десятков километров у первой версии (использовалась во время Второй мировой войны). У системы Loran-C погрешность удалось снизить до 150-300 метров.

    \subsubsection{Спутниковая навигация}

    Спутниковая система навигации — комплексная электронно-техническая система, состоящая из совокупности наземного и космического оборудования. Предназначена для определения местоположения (географически координат и высоты), а также параметров движения (скорости и направления) для наземных, водных и воздушных объектов.

    Первая спутниковая система навигации в СССР называлась Циклон и была предназначена для военных (гражданский вариант — Цикада). Основные параметры:
    \begin{itemize}
        \item Погрешность 90м
        \item Принцип работы — эффект Доплера
        \item Система позволяла получать данные о местоположении лишь 6 минут раз в полтора часа
    \end{itemize}

    Для работы спутниковой навигации необходимо:
    \begin{itemize}
        \item Видимость нескольких спутников из любой точки земли
        \item Контроль с земли
        \item Атомные часы на каждом спутнике
        \item Координаты спутников должны быть известны в любой момент времени
    \end{itemize}

    Методы исключения ошибок:
    \begin{itemize}
        \item Методы моделирования
        \item Двухчастотный приёмник
        \item Разностные Методы
        \item Методы высокой точности
    \end{itemize}

    Недостатки:
    \begin{itemize}
        \item Точность
        \item Потеря связи
        \item Недостаточное количество спутников
        \item Ошибки из-за инфраструктурных объектов
    \end{itemize}

    \subsection{Локальное позиционирование}
    \begin{itemize}
        \item Lidar
        \item Видеокамера
        \item Датчики расстояния
        \item и т. д.
    \end{itemize}

    \subsubsection{Одометрия}
    Одометрия — использование данных о движении приводов для оценки перемещения.

    Расстояние = Скорость $\cdot$ Поправка $\cdot$ Время

    Проблемы:
    \begin{itemize}
        \item Измерение движение относительно идеала и, как следствие низкая точность
    \end{itemize}

    Причины:
    \begin{itemize}
        \item Неверные настройки
        \item Различные покрытия (одни колёса ездят по-разному по разным материалам, например по асфальту и гравию)
        \item Различия моторов (даже моторы в одной партии с одного завода не будут идеально идентичны, а современем их различия лишь возрастут)
    \end{itemize}

    Несколько уменьшить расхождение с моделью и повысить точность могут помочь энкодеры — датчики угла поворота, которые позволяют точно знать число совершённых колесом оборотов.

    Таким образом, для определения положения в пространстве методом одометрии будет требоваться изначальная калибровка модели. При этом со временем реальное положение робота будет всё больше отклонятся от модели из-за накапливающихся ошибок и требовать повторного проведения калибровки.
\end{sloppypar}
\end{document}