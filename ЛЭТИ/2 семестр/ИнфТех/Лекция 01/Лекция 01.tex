\documentclass[12pt]{article}
\usepackage[a4paper, portrait, margin=1cm, right=1cm]{geometry}
\usepackage{fontspec}
\usepackage[fleqn]{amsmath}
\usepackage{setspace}
\usepackage{graphicx}

\graphicspath{./graphics/}
\setmainfont[Ligatures=TeX]{Linux Libertine}

\title{Информационные технологии. Лекция 01. КФС. Основные свойства. БТС}
\author{Студент группы 2305 Макурин Александр}
\date{07 февраля 2023}

\begin{document}

\maketitle
\begin{sloppypar}
    \setstretch{1.8}

    \section*{Организационные вопросы}
    \subsection*{Список лабораторных (каждая даёт 10 баллов)}
    \begin{itemize}
        \item Начало работ с Gazebo
        \item Создание модели ТС (БПЛА, БТС)
        \item Автономное ТС
        \item Реализация протоколов связи
        \item Роевой интеллект на группе ТС
        \item Стратегическое планирование
    \end{itemize}
    \subsection*{Оценки}
    \begin{itemize}
        \item 95\%+ (57+ баллов) = 5
        \item 90\%+ (54+ баллов) = 4
        \item 80\%+ (48+ баллов) = 3
    \end{itemize}

    \section*{Индустрия 4.0 - замещение людей роботами}

    БТС - беспилотное транспортное средство

    \includegraphics[width=0.5\textwidth]{graphics/СУ_ОУ.png}

    Кибер-физическая система (КФС) - система, интегрирующая способности к вычислениям,
    связи и хранению информации с мониторингом и/или управлением объектами
    физического мира и должна делать это надёжно, безопасно, эффективно и в
    реальном времени.

    \includegraphics[width=0.5\textwidth]{graphics/Мехатроника.png}

    Информационные технологии охватывают такие аспекты мехатроники, как управление, электроника и программное обеспечение.

    АСУ ТП становятся всё менее распространёнными по причинам плохой
    расширяемости (одна система управления на множество датчиков и механизмов) и
    коррупции в научной среде.

    \section{История робототехники}
    \begin{itemize}
        \item Движущиеся статуи (I век до нашей эры)
        \item Механические устройства (Леонардо да Винчи)
        \item Автоматоны (Пьер Жаке-Дро)
        \item Разностная машина (Чарльз Бэббидж)
        \item Boilerplate (Арчи Кемпион)
    \end{itemize}

    \section{Промышленные роботы}
    \begin{itemize}
        \item Манипуляторы
        \item Johns Hopkins Beast (1960) — робот, решающий главную задачу всех
              роботов (найти поесть) Он умел искать розетки, от которых
              подзаряжался, в белой комнате с чёрными розетками.
              \includegraphics[width=0.7\textwidth]{graphics/johns_beast.jpg}
        \item Shakey (1970)
        \item Луноход
        \item Марсоход
    \end{itemize}

    Задача грузчика — как двум роботам перенести пианино. Не решённая задача.
    Для решения требуется найти алгоритм нахождения баланса между двумя роботами
    и пианино.

    Робо-рука для сбора помидоров — требует контроля силы сжатия/удержания
    помидора.

    \section{Тенденции развития}
    \begin{itemize}
        \item Разработка стандартов
        \item Уменьшение размеров
        \item Удешевление стоимости комплектующих
        \item Развитие систем управления:
              \begin{itemize}
                  \item ИИ
                  \item Стайное управление
                  \item Функционирование в условиях неопределённости
              \end{itemize}
    \end{itemize}
    \subsection*{Три уровня планирования:}
    \begin{itemize}
        \item Оперативный — копипаст со Stack Overflow - решение текущей задачи
        \item Тактический — Junior $\rightarrow$ Middle $\rightarrow$ Senior
        \item Стратегический — главная цель, на которую направлены задачи всех
              остальных уровней, например, увеличение прибыли
    \end{itemize}

    Разработка стандартов — Разработка правил, по которым можно было бы создать
    ИИ, который точно не сойдёт с ума — не восстанет против человека, будет
    выполнять свою задачу.

    \setstretch{1}
    \[
        \begin{array}{ccccc}
            E              & = & E^{\textit{inf}}               & \cup & E^{phy}                    \\
            \uparrow       &   & \uparrow                       &      & \uparrow                   \\
            \text{система} &   & \text{информационные элементы} &      & \text{физические элементы}
        \end{array}
    \]
    \begin{center}
        \includegraphics[width=0.5\textwidth]{graphics/Информационные_и_физические_элементы.png}
    \end{center}
    \[
        \begin{array}{ccccc}
            S_E                      & = & f( & E,             & U)                         \\
            \uparrow                 &   &    & \uparrow       & \uparrow                   \\
            \text{состояние системы} &   &    & \text{система} & \text{входные воздействия}
        \end{array}
    \]
    \setstretch{1.8}

    \subsection*{Система}

    \begin{center}
        \includegraphics[width=0.7\textwidth]{graphics/Система.png}
    \end{center}

    \[
        U = U_{out} \cup U_{in}
    \]

    \includegraphics[width=0.9\textwidth]{graphics/Состояние системы.png}

    $y$ — выходные параметры

    $|y| = |U|$. Или, другими словами, размерность $y$ = размерность $U$

    $\dfrac{\delta S}{\delta t} = F(E^t, U^t)$

    $S_E = y + e$, где $e$ - погрешность системы и обычно опускается

    При стремлении длины временного отрезка к 0, изменение системы тоже стремится к 0:
    \[
        \Delta r \rightarrow 0 \Leftrightarrow \Delta S \rightarrow 0
    \]

    Когда состояние системы близко к оптимальному, значения выходных параметров стремятся к нормальному распределению.

    \includegraphics[width=0.5\textwidth]{graphics/Состояния системы и нормальное распределение.png}

    \[
        f: S_E^{\text{норм}} \rightarrow S_E^{\text{плохо}}
    \]

    \section{Виды архитектур интеллектуальных агентов:}
    \subsection{Реактивные}
    Пример - Johns Hopkins Beast.

    \begin{picture}(200,200)
        \thicklines
        \put(100,10){\line(0,1){180}}
        \put(10,100){\line(1,0){180}}
        \put(100,190){\makebox(0,0)[b]{\textsc{Сенсоры}}}
        \put(190,100){\makebox(0,0)[l]{\rotatebox{-90}{\textsc{Реакция}}}}
        \put(100,10){\makebox(0,0)[t]{\textsc{Механика}}}
        \put(10,100){\makebox(0,0)[r]{\rotatebox{90}{\textsc{Словарь действий}}}}
    \end{picture}

    \subsection{Делиберативные}
    Пример - Яндекс Навигатор

    \begin{picture}(200,200)
        \thicklines
        \put(100,10){\line(0,1){180}}
        \put(10,100){\line(1,0){180}}
        \put(100,190){\makebox(0,0)[b]{\textsc{Сенсоры}}}
        \put(190,100){\makebox(0,0)[l]{\rotatebox{-90}{\textsc{Критерий}}}}
        \put(100,10){\makebox(0,0)[t]{\textsc{Механика}}}
        \put(10,100){\makebox(0,0)[r]{\rotatebox{90}{\textsc{Априорная информация}}}}
    \end{picture}

    \subsection{Гибридные}

    Пример - автопилот Tesla (он знает, что нужно делать (априорная информация) (ехать) и реагирует на изменения)

    \section{Сопутствующие задачи для беспилотника:}
    \subsection{Стабилизация}
    Система должна стремиться находится максимально близко к идеальному состоянию.

    \includegraphics[width=0.5\textwidth]{graphics/Стабилизация.png}
    \[
        y(t) \rightarrow y^* \text{ или } \lim{|y^* - y(t)|} \leq \nu
    \]
    где $\nu$ - допустимое отклонение, а $y^*$ - идеальное состояние.
    \[
        \lim{M(y^* - y(t))} \leq \nu
    \]
    где $M$ - математическое ожидание.

    \subsection{Слежение}
    Система должна сохранять хоть какую-то работоспособность.
    Например - при потере управляющего сигнала дроном.
    \[
        \lim |U_{in}^{{act}^*} - U_{in}^{act}| = 0
    \]

    \subsection{Возбуждение (разгон)}
    \[
        \lim |U^* - U^*(E)| = 0
    \]

    \subsection{Синхронизация}
    Система должна быть наблюдаема и повторяема.
    \[
        e_i^1 - e_j^2 \rightarrow 0
    \]

    \section{Рекомендуемая литература}
    \begin{itemize}
        \item Martsenyuk V. P. et al. Software Complex in the Study of the Mathematical Model of Cyber-Physical Systems //ICTES. – 2020. – С. 87-97.
        \item Legatiuk D. et al. A categorical approach towards metamodeling cyber-physical systems //The 11th International Workshop on Structural Health Monitoring (IWSHM). Stanford, CA, USA. – 2017. – Т. 12. – С. 2017.
        \item Platzer A. Logic \& proofs for cyber-physical systems //Automated Reasoning: 8th International Joint Conference, IJCAR 2016, Coimbra, Portugal, June 27–July 2, 2016, Proceedings 8. – Springer International Publishing, 2016. – С. 15-21.
        \item Letichevsky A. A. et al. Cyber-physical systems //Cybernetics and Systems Analysis. – 2017. – Т. 53. – С. 821-834.
        \item Wan J. et al. From machine-to-machine communications towards cyber-physical systems //Computer Science and Information Systems. – 2013. – Т. 10. – №. 3. – С. 1105-1128.
        \item Фрадков А. Л. Кибернетическая физика: принципы и примеры. – 2003.
    \end{itemize}


\end{sloppypar}
\end{document}