\documentclass{article}
\usepackage[a4paper, portrait, margin=1cm, right=1cm]{geometry}
\usepackage{fontspec}
\usepackage[fleqn]{amsmath}
\usepackage{setspace}
\usepackage{graphicx}

\graphicspath{./graphics/}
\setmainfont[Ligatures=TeX]{Linux Libertine}

\title{Информационные технологии. Лекция 04. Сенсорные системы}
\author{Студент группы 2305 Макурин Александр}
\date{06 марта 2023}

\begin{document}
\maketitle
\begin{sloppypar}
    \setstretch{1.8}
    Датчик — устройство, преобразающее контролируемую величину в удобную для обработки форму (из внешней среды в данные).

    Характеристики измерений:
    \begin{itemize}
        \item Чистота
        \item Качество
        \item Точность
    \end{itemize}

    Виды lfnxbrjd:
    \begin{enumerate}
        \item Направленные на измерение внетреннего состояния
        \item Физико-химический анализ окружающей среды
        \item Общая картина окружающей среды
    \end{enumerate}

    Проблемы датчиков:
    \begin{itemize}
        \item Шумы
        \item Объединение нескольких измерений
    \end{itemize}

    \subsection*{Энкодеры}
    \begin{itemize}
        \item Абсолютные
        \item Накапливающик
    \end{itemize}

    Проблемы (из-за окружающей среды) — локальный разброс.

    %TODO%
\end{sloppypar}
\end{document}