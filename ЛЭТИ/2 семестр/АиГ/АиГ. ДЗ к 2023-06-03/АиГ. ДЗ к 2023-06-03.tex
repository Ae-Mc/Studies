\documentclass[12pt]{article}
\usepackage[a4paper, portrait, margin=1cm, bottom=2cm]{geometry}
\usepackage{fontspec}
\usepackage[fleqn]{amsmath}
\usepackage{amssymb}
\usepackage{indentfirst}
\usepackage{graphicx}
\usepackage{framed}

\setmainfont[Ligatures=TeX]{Linux Libertine}
\graphicspath{./graphics/}

\title{АиГ. ДЗ к 2023-06-03. Группы 1. Вариант №14}
\author{Студент группы 2305 Александр Макурин}
\date{30 мая 2023}

\begin{document}
\maketitle
\section{Пусть $\phi : \mathbb{C}^*\rightarrow \mathbb{C}^*$ задано формулой $\phi(z) = \dfrac{z}{|z|}$. Докажите, что $\phi$ — гомоморфизм групп. Найдите его ядро и образ. Является ли $\phi$ мономорфизмом, эпиморфизмом, изоморфизмом?}
Пусть $z_1 = a + bi$, $z_2 = c + di$. Тогда, если $\phi$ является гомоморфизмом должно соблюдаться следующее равенство:
\[
    \phi(z_1 * z_2) = \phi(z_1) * \phi(z_2)
\]
\[
    \dfrac{ac - bd + (ad + bc)i}{\sqrt{(ac - bd)^2 + (ad + bc)^2}} = \dfrac{a + bi}{\sqrt{a^2 + b^2}} \dfrac{c + di}{\sqrt{c^2 + d^2}}
\]
Упростим левую и правую часть по отдельности:
\[
    \dfrac{ac - bd + (ad + bc)i}{\sqrt{(ac - bd)^2 + (ad + bc)^2}} = \dfrac{ac - bd + (ad + bc)i}{\sqrt{a^2c^2 + b^2d^2 + a^2d^2 + b^2c^2}}
\]
\[
    \dfrac{a + bi}{\sqrt{a^2 + b^2}} \dfrac{c + di}{\sqrt{c^2 + d^2}} = \dfrac{ac - bd + (ad + bc)i}{\sqrt{a^2c^2 + b^2d^2 + a^2d^2 + b^2c^2}}
\]
Левая и правая части равны, следовательно $\phi$ — гомоморфизм групп $\mathbb{C}^*$ и $\mathbb{C}^*$.

Нейтральным элементов в $\mathbb{C}^*$ является $1$. Отсюда, ядро:
\[
    \dfrac{a + bi}{\sqrt{a^2 + b^2}} = 1 \Rightarrow \dfrac{a}{|a|} = 1 \text{ — верно  при } \forall a \in \mathbb{R}_+ \backslash \{0\}
\]

Ядро — $\mathbb{R}_+\backslash\{0\}$.

Так как мы делим комплексные числа на длину их векторов, мы получаем комплексные числа, длина которых всегда равна единице. Тогда образ: $\left\{z \in \mathbb{C} : |z| = 1\right\}$.

$\phi$ не ивляется ни эпиморфизмом, ни мономорфизмом, ни изоморфизмом, т. к.:
\begin{itemize}
    \item Образ не совпадает с множеством $\mathbb{C}$.
    \item Бесконечное количество векторов множества комплексных чисел переходит в один и тот же вектор единичной длины.
    \item Невозможно создать морфизм, обратный данному. Т. е. $\nexists \phi^{-1} : \phi^{-1}(\phi(z)) = z$.
\end{itemize}

\begin{framed}
    Ответ:\\
    Ядро: $\mathbb{R}_+\backslash\{0\}$ \\
    Образ: $\left\{z \in \mathbb{C} : |z| = 1 \right\}$ \\
    $\phi$ не является мономорфизмом, изоморфизмом и эпиморфизмом.
\end{framed}

\section{Пусть $\mathbb{C}[x]$ обозначает аддитивную группу многочленов от \\
  переменной $x$ с коэффициентами из $\mathbb{C}$, а $H$ — подмножество \\
  многочленов, имеющих корень 3. Докажите, что $\mathbb{C}[x]/H \cong \mathbb{C}$.}

Нейтральный элемент группы $\mathbb{C}[x]$ $e_{\mathbb{C}[x]}$ равен $0$ (многочлен нулевой степени).

Нейтральный элемент группы $\mathbb{C}^+$ $e_{\mathbb{C^+}}$ равен $0$.

Возьмём морфизм $f: \mathbb{C}[x] \rightarrow \mathbb{C}^+$, такой, что $f(c) = c(3)$.

Докажем, что это гомоморфизм:
\[
    f(P + Q) = f(P) + f(Q)
\]

Т. к. значение многочлена в точке это подстановка чисел во все его члены, то значение суммы многочленов равно сумме значений многочленов. Следовательно, $f$ — это гомоморфизм.

$P_n = \sum_{k=0}^n{a_kx^k}, Q_l = \sum_{k=0}^l{b_kx^k}$ — многочлены степеней $n$ и $l$ соответственно на комплексном множестве ($a_k, b_k \in \mathbb{C}$). $n \geq l$. $\forall k \in \{l + 1, l + 2, ..., n\}, b_k = 0$. Так как сложение как многочленов, так и комплексных чисел коммутативно, можно поменять местами $P$ и $Q$. Следовательно, под ограничение $n \geq l$ подходят любые два многочлена, если расположить их в нужном порядке. На результат это не повлияет.
\[
    P + Q = \sum_{k=0}^n{a_kx^k} + \sum_{k=0}^n{b_kx^k} = \sum_{k=0}^n(a_k + b_k)x^k
\]
\[
    f(P + Q) = \sum_{k = 0}^n(a_k + b_k)3^k
\]
\[
    f(P) = \sum_{k = 0}^n{a_k3^k}
\]
\[
    f(Q) = \sum_{k = 0}^n{b_k3^k}
\]
\[
    f(P) + f(Q) = \sum_{k = 0}^n{a_k3^k} + \sum_{k = 0}^n{b_k3^k} = \sum_{k = 0}^n(a_k + b_k)3^k = f(P + Q) \text{ — следовательно, гомоморфизм}
\]

Ядром этого гомоморфизма $f$ будет множество $H$ (элементы $\mathbb{C}[x]$, такие, что $f(c) = e_{\mathbb{C}^+} = 0$, а это и есть множество $H$):
\[
    \text{Ker}(f) = H
\]

Образом будет всё множество комплексных чисел:
\[
    \text{Im}(f) = \mathbb{C}
\]

По теореме о том, что факторгруппа по ядру гомоморфизма изоморфна его образу:
\[
    \mathbb{C}[x] / H \cong \mathbb{C}^+
\]
Что и требовалось доказать.

\fbox{Ответ: $f : \mathbb{C}[x] \rightarrow \mathbb{C}^+$, $f(c) = c(3)$ — изоморфизм}
\end{document}