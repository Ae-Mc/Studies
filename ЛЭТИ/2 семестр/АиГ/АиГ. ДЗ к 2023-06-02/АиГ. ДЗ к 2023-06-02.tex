\documentclass[12pt]{article}
\usepackage[a4paper, portrait, margin=1cm, bottom=2cm]{geometry}
\usepackage{fontspec}
\usepackage[fleqn]{amsmath}
\usepackage{amssymb}
\usepackage{indentfirst}
\usepackage{graphicx}
\usepackage{framed}
\usepackage{enumitem}

\setmainfont[Ligatures=TeX]{Linux Libertine}
\graphicspath{./graphics/}

\title{АиГ. ДЗ к 2023-06-02. Группы 0. Вариант №14}
\author{Студент группы 2305 Александр Макурин}
\date{29 мая 2023}

\begin{document}
\maketitle
\section{}
\begin{enumerate}[label=\alph*)]
    \item Является ли группой $(\mathbb{N} \backslash \{0\}, +)$?

          Нет, т. к. нет нейтрального и обратных элементов.
    \item Является ли группой множество невырожденных нижнетреугольных матриц размера $n \times n$ над $\mathbb{R}$ с операцией сложения?

          Нет. Контрпример на матрицах:
          \[
              E_n + (-E_n) = 0 \text{ — вырожденная}
          \]
    \item Является ли группой множество биективных функций из $\mathbb{R}$ в $\mathbb{R}$ с операцией умножения?

          Нет. Контрпример — произведение двух одинаковых функций $f(x) = x$. Они дадут $x^2$, который не является биективной функцией.
\end{enumerate}
\end{document}