\documentclass[12pt]{article}
\usepackage[a4paper, portrait, margin=1cm, bottom=2cm]{geometry}
\usepackage{fontspec}
\usepackage[fleqn]{amsmath}
\usepackage{amssymb}
\usepackage{indentfirst}
\usepackage{polyglossia}

\setmainfont[Ligatures=TeX]{Linux Libertine}
\setdefaultlanguage{russian}
\setotherlanguages{english}

\begin{document}

\section{5. Ядро и образ линейного оператора, их размерности.}
\subsection{Определение}
Пусть A - линейный оператор векторного пространства V.

Образ: множество $\forall y \in V : \exists\ x \in V : A(x) = y$. Обозначение: \emph{Im A}.

Ядро: множество $\forall x \in V : A(x) = 0$. Обозначение: \emph{Ker A}.

\subsection{Размерности Образа и Ядра}
A - линейный оператор в век. пространстве размерности n.

Размерность образа соответствует рангу матрицы оператора (Размерность $Im(A) = rk(A)$)

Размерность ядра соответствует дополнению ранга матрицы до n (Размерность $Ker(A) = n - rk(A)$)

\subsection{Доказательство подпространства}

\subsubsection{Образ}
Пусть $y_1, y_2 \in \emph{Im A}$, t - произвольное вещественное число $\Rightarrow \exists\ x_1, x_2 \in A : \begin{cases}
        A(x_1) = y_1 \\
        A(x_2) = y_2 \\
    \end{cases}
    \Rightarrow y_1 + y_2 = A(x_1)\ + A(x_2) = A(x_1 + x_2)$ и $ty_1 = tA(x_1) = A(tx_1)$ $\Rightarrow y_1 + y_2$ и ty - элементы образа $\Rightarrow$ образ - подпространство.

\subsubsection{Ядро}
Пусть $x_1, x_2 \in \emph{Ker A}$, t - произвольное вещественное число $\Rightarrow \begin{cases}
        A(x_1) = 0 \\
        A(x_2) = 0 \\
    \end{cases}
    \Rightarrow A(x_1+x_2) = A(x_1) + A(x_2) = 0 + 0 = 0, A(tx_1) = tA(x_1) = 0 \Rightarrow x_1 + x_2$ и tx - элементы ядра $\Rightarrow$ ядро - подпространство.

\subsection{Примеры}
Образ проекции пространства на плоскость $XOY$ – плоскость $XOY$, ядро этого оператора – все векторы, параллельные оси $OZ$.

Образ оператора дифференцирования на пространстве всех многочленов – это же пространство многочленов. Ядро этого оператора – константы.

\section{6. Алгоритм Чуркина}
\subsection{Алгоритм}
Имеется матрица оператора A размерности n. Составим матрицу B порядка n x 2n, где первые n столбцов занимает транспонированная матрица A, следующие n столбцов единичная матрица.

Элементарными преобразованиями приводим левую часть к ступенчатому виду.

Ненулевые строки левой части - базис образа оператора A.

Продолжение нулевых строк в правой части - базис ядра.

\subsection{Обоснование}
Строки $A^T$ соответствуют значению оператора на базисе.

Записав значения и приведя к ступенчатому виду, получаем базис (базис Образа).

Левая часть матрицы B соответствует значению оператора на векторе, тогда как правая - координатной записи вектора.После преобразований закономерность сохраняется и в правой части: строки соответствующие нулевым строкам в левой являются базисом ядра. (P.S. Вспомни определение ядра)

\subsection{Пример}
Пусть $A = \left(\begin{array}{cccc}
            2  & 0 & 1 & -3 \\
            1  & 0 & 3 & -4 \\
            -1 & 0 & 2 & -1 \\
            1  & 0 & 1 & -2
        \end{array} \right)$
По алгоритму:

$\left(\begin{array}{cccc|cccc}
            2  & -1 & -1 & 1  & 1 & 0 & 0 & 0 \\
            0  & 0  & 0  & 0  & 0 & 1 & 0 & 0 \\
            1  & 3  & 2  & 1  & 0 & 0 & 1 & 0 \\
            -3 & -4 & -1 & -2 & 0 & 0 & 0 & 1
        \end{array} \right)$ перемещаем строку $\left(\begin{array}{cccc|cccc}
            2  & -1 & -1 & 1  & 1 & 0 & 0 & 0 \\
            1  & 3  & 2  & 1  & 0 & 0 & 1 & 0 \\
            -3 & -4 & -1 & -2 & 0 & 0 & 0 & 1 \\
            0  & 0  & 0  & 0  & 0 & 1 & 0 & 0
        \end{array} \right)$

$l_3 \rightarrow l_3 + l_2 + l_1$ получаем $\left(\begin{array}{cccc|cccc}
            2 & -1 & -1 & 1 & 1 & 0 & 0 & 0 \\
            1 & 3  & 2  & 1 & 0 & 0 & 1 & 0 \\
            0 & 0  & 0  & 0 & 1 & 0 & 1 & 1 \\
            0 & 0  & 0  & 0 & 0 & 1 & 0 & 0
        \end{array} \right)$

Тогда: $(2,-1,-1,1), (1,3,2,1)$ - базис образа, $(1,0,1,1),(0,1,0,0)$ -базис ядра.

\section{7. Проверка линейного оператора на вырожденность и невырожденность}
\subsection{Невырожденная матрица оператора}
\subsection{Оператор вырожденный, если $\exists\ x_1, x_2 : A(x_1) = A(x_2) = y$. Тогда невозможно однозначно определить обратный оператор от y}
\subsection{Оператор невырожденный только тогда, когда ядро состоит только из нулевого элемента}
Оператор невырожденный $\Rightarrow$ ядро не может состоять больше чем из одного элемента (пункт 2) $\Rightarrow$ ядро состоит только из 0.

Ядро состоит из 0 $\Rightarrow A(x_1) = A(x_2)$ будет:

$A(x_1) - A(x_2) = 0 \Rightarrow A(x_1 - x_2) = 0 \Rightarrow x_1 - x_2 = 0 \Rightarrow x_1 = x_2$

Совпадение значений оператора на двух элемента $\Rightarrow$ совпадают элементы $\Rightarrow$ возможно корректное определение обратного оператора.

\subsection{Геометрические соображения}
Пример:

Вырождена ли симметрия относительно плоскости? Легко заметить, что двойная симметрия вернёт точку в исходное положение. Таким образом обратный к симметрии - это сама симметрия, а значит она невырождена.

\subsection{Примеры}
\begin{enumerate}
    \item $A(x) = a * x$, где $a \in R$.

          Ядро состоит только из нулевого элемента тогда и только тогда, когда $a \neq 0$.

          Если $a = 0$, оператор вырожден, поскольку прообраз 0 неоднозначно определён.

    \item $A(x) = (x,e)e$, при этом $|e| = 1$

          Оператор вырожден, поскольку ядро состоит не только из 0.
\end{enumerate}

\section{8. Собственные числа}
\subsection{Определение}
Число $\lambda$ называется собственным числом оператора $L$, если существует такой ненулевой вектор $x$, что $L(x) = \lambda x$. При этом вектор $x$ называется собственным вектором оператора $L$, отвечающим собственному числу $\lambda$.

\subsection{Вычисление}
\subsubsection{Геометрический}
Пример: Найти с.ч. и с.в. проекции пространства на плоскость $XOY$.

Найдём все векторы || своей проекции - это все вектора || $XOY$ (проекция равна вектору) и все вектора перпендикулярные плоскости (проекция равна нулю). $\Rightarrow$

$\lambda_1 = 1, X_{\lambda_1} = \left(\begin{array}{c}
            x \\
            y \\
            0
        \end{array} \right); \lambda_2 = 0, X_{\lambda_2} = \left(\begin{array}{c}
            0 \\
            0 \\
            z
        \end{array} \right)$, где $x, y, z \in R$.
\subsubsection{Аналитический}
Пример: Найти с.ч. и с.в. оператора дифференцирования на множестве всех многочленов.

Рассмотрим $p^\prime(x) = \lambda p(x)$. Если $\lambda \neq 0$, то степени правой и левой частей не совпадут. Если $\lambda = 0$, то p(x) - константа.

Ответ: $\lambda = 0, X_\lambda = C = const$.

\subsubsection{С помощью матрицы}
Пусть оператор задан матрицей $\Rightarrow A(x) = Ax \Rightarrow$

\begin{center}
    $Ax = \lambda x \Leftrightarrow Ax - \lambda x = 0 \Leftrightarrow Ax - \lambda Ex = 0 \Leftrightarrow (A - \lambda E)x = 0$
\end{center}

$\lambda$ - с.ч. только тогда, когда система уравнений на координаты x $(A - \lambda E)x = 0$ имеет ненулевое решение.

Выполняется только тогда, когда определитель $(A - \lambda E)$ равен 0.

Алгоритм: \begin{enumerate}
    \item Найти корни уравнения $det(A - \lambda E) = 0$
    \item Для каждого из корней решить СЛУ $(A - \lambda E)x = 0$. Решения будут задавать координаты множества с.в., соответствующих с.ч. (может иметь размерность 1 и более)
\end{enumerate}
\subsection{Диагонализуемость}

Если удаётся найти базис из с.в. для линейного оператора, то его матрица в этом базисе будет диагональна, а на диагонали будут с.ч.

Пусть матрица диагональна, значит элементы диагонали обозначим через $\lambda_k$ из условия $A(e_k) = \lambda_k e_k$ получим, что по определению все элементы базиса - с.в. данного линейного оператора.

Итог: матрица линейного оператора диагонализуема тогда только тогда, когда для этого линейного оператора найдётся базис из собственных векторов.

\section{9. Евклидовые и Унитарные пространства}
Вещественное линейное пространство $\mathbb{E}$ называется евклидовым, если каждой паре элементов $\mathbf{u},\,\mathbf{v}$ этого пространства поставлено в соответствие действительное число $\langle \mathbf{u},\mathbf{v} \rangle$, называемое скалярным произведением, причем это соответствие удовлетворяет следующим условиям:
\begin{enumerate}
    \item $(\mathbf{u},\mathbf{v})=(\mathbf{v},\mathbf{u})\ \forall\ \mathbf{u},\mathbf{v}\in \mathbb{E}$
    \item $(\mathbf{u} + \mathbf{v},\mathbf{w})=(\mathbf{u},\mathbf{w})+(\mathbf{v},\mathbf{w})\ \forall\ \mathbf{u},\mathbf{v},\mathbf{w}\in \mathbb{E}$
    \item $(k\mathbf{u},\mathbf{v})=k(\mathbf{u},\mathbf{v})\ \forall\ \mathbf{u},\mathbf{v}\in \mathbb{E},\forall k\in \mathbb{R}$
    \item $(\mathbf{u},\mathbf{u})\geq 0$ и $(\mathbf{u},\mathbf{u})=0$ только если $\mathbf{u}=\mathbf{0}$
\end{enumerate}
В скалярном произведении $\langle\mathbf{u}, \mathbf{v}\rangle$ вектор $\mathbf{u}$ — первый, а вектор $\mathbf{v}$ — второй сомножители. Скалярное произведение $\langle\mathbf{v},\mathbf{v}\rangle$ вектора $\mathbf{v}$ на себя называется скалярным квадратом. Условия 1–4 называются аксиомами скалярного произведения. Аксиома 1 определяет симметричность скалярного произведения, аксиомы 2 и 3 — аддитивность и однородность по первому сомножителю, аксиома 4 — неотрицательность скалярного квадрата $\langle\mathbf{v}, \mathbf{v}\rangle$.

\subsection{Геометрические примеры для трёхмерного евклидова пространства}

Является ли скалярным произведением:
\begin{enumerate}
    \item Смешанное произведение $(n, x, y)$, где $n$ – фиксированный вектор.

          Ответ: Нет, поскольку если $n = 0$, то $(n, x, x) = 0$ при каждом $x$, а если $n\neq 0$, то $(n, x, x) = 0$ при $x = n$.

    \item Скалярное произведение $(x + n, y + n)$.

          Ответ: Если $n = 0$, то является, поскольку совпадает с обычным скалярным произведением. Если $n \neq 0$, то нет,
          поскольку $(x + n, x + n) = 0$ при $x = –n$.

    \item Произведение двух скалярных произведений $(n, x)(n, y)$.

          Ответ: Нет, поскольку $(n, x)(n, x) = 0$ при $x$, перпендикулярном $n$.

    \item Произведение модуля $|n|$ и скалярного произведения $|(n, x)|$.

          Ответ: Если $n = 0$, то не является, поскольку $|n|(x, x) = 0$ при каждом $x$. Если $n \neq 0$, то является, поскольку
          совпадает с обычным скалярным произведением, умноженным на ненулевую константу.

    \item Произведение модулей $|x|\cdot |y|$.

          Ответ: Нет, поскольку $(x, y - y) = 0$ не всегда равно $(x, y) + (x, -y) = 2(x, y)$. Например, это неверно для двух
          равных между собой ненулевых векторов.
\end{enumerate}

\subsection{Пример вычисления скалярного произведения, модулей и углов для многочленов}
Рассмотрим многочлены $t$ и $t^3$ и скалярное произведение $(f(t),g(t))=\int_{-1}^{1} f(t)g(t)\,dt$.
Найдем модуль вектора $t$:
$$(t,t)=\int_{-1}^{1} t^2\,dt =\frac{2}{3}$$
$$|t|=\sqrt{(t,t)}=\sqrt{\frac{2}{3}}$$
Затем найдем модуль вектора $t^3$:
$$(t^3,t^3)=\int_{-1}^{1} t^6\,dt=\frac{2}{7}$$
$$|t^3|=\sqrt{(t^3,t^3)}=\sqrt{\frac{2}{7}}$$
И с помощью скалярного произведения найдем угол между ними:
$$(t,t^3)=\int_{-1}^{1} t^4\,dt =\frac{2}{5}$$
$$\measuredangle(t,t^3)=\arccos \frac{(t,t^3)}{|t||t^3|}=\arccos \frac{0.4}{\sqrt{\frac{4}{21}}}=\arccos \frac{\sqrt{21}}{5} $$

\section{10. Неравенство Коши-Буняковского-Шварца. Неравенство треугольника}
\subsection{Неравенство Коши-Буняковского-Шварца}
В евклидовом пространстве модуль скалярного произведения двух векторов не превосходит произведения их модулей, то есть  $|(x,y)| \leq |x| \cdot |y|$.  Это неравенство обеспечивает условие $\left|\dfrac{(x,y)}{|x||y|}\right| \leq 1$.

\subsection{Доказательство неравенства Коши-Буняковского-Шварца}
Пусть x,y - векторы в линейном пространстве, $\lambda$  - произвольное вещественное число. Рассмотрим фукнцию от $\lambda: f(\lambda)=(\lambda x+y,\lambda x+y)$
\begin{align*}
    0 & \leq (\lambda x + y, \lambda x + y)= \lambda^2 (x,x) + 2\lambda (x,y) + (y,y)
\end{align*}
Этот квадратный трехчлен всюду неотрицательный, следовательно, его дискриминант неположительный, то есть
$$D=4(x,y)^2-4(x,x)(y,y)\leq0;$$
Следовательно, $(x,y)^2\leq(x,x)(y,y)$.
Поэтому $|(x,y)| \leq |x||y|$.\\
Примечание \\
Если определитель скалярного произведения в виде $(f(t),g(t))=\int_{-1}^{1} f(t)g(t) \,dt$, неравенство Коши-Неравенство Коши-Буняковского-Шварца принимает вид:
$$|(f(t),g(t))|=|\int_{-1}^{1} f(t)g(t)\,dt| \leq \sqrt{\int_{-1}^{1} f^2(t)\,dt} \sqrt{\int_{-1}^{1} g^2(t)\,dt}$$
для любых двух функций $f$ и $g$. непрерывных функций\\
\\
\textbf{Неравенство треугольника}\\
\\
В евклидовом пространстве выполнено неравенство треугольника: модуль суммы двух векторов не
превосходит суммы их модулей.
Пусть $l_1$ и $l_2$ – векторы.\\
\textbf{Докажем}, что $|l_1 + l_2| \leq |l_1|+|l_2|$.

$$(l_1+l_2)^2=l_1^2+2(l_1,l_2)+l_2^2 \leq l_1^2+2|l_1||l_2|+l_2^2=(|l_1|+|l_2|)^2$$

Извлекая из неравенства квадратный корень, получим$|l_1+l_2|\leq|l_1|+|l_2|$.
\section{11. Матрица Грама.Свойства матрицы Грама}
Рассмотрим скалярное произведение в трёхмерном евклидовом пространстве (базис не обязательно
ортогональный).\\
$$x=x_1e_1+x_2e_3+x_3e_3$$
$$y=y_1e_1+y_2e_2+y_3e_3$$
Тогда $(x,y)=((x_1e_1+x_2e_3+x_3e_3),(y_1e_1+y_2e_2+y_3e_3))=$
$$x_1y_1e^2_1+x_1y_2(e_1,e_2)+x_1y_3(e_1,e_3)+x_2y_1(e_2,e_1)+x_2y_2e^2_2+x_2y_3(e_2,e_3)+x_3y_1(e_3,e_1)+x_3y_2(e_3,e_2) + x_3y_3e^2_3$$
Записывать выражение в такой форме не очень удобно, поэтому используется знак суммирования:
$$(x,y)=\sum\limits_{i,j=1}^3 x_iy_j(e_i,e_j).$$

Множество коэффициентов можно записать в виде матрицы Грама:
$$\Gamma_{ij}=(e_i,e_j) \quad (i,j \text{ от 1 до 3}).$$
То есть
$$\Gamma = \begin{pmatrix}
        (e_1,e_1) & (e_1,e_2) & (e_1,e_3) \\
        (e_2,e_1) & (e_2,e_2) & (e_2,e_3) \\
        (e_3,e_1) & (e_3,e_2) & (e_3,e_3)
    \end{pmatrix}$$
или
$$\Gamma = \begin{pmatrix}
        e_1^2     & (e_1,e_2) & (e_1,e_3) \\
        (e_2,e_1) & e_2^2     & (e_2,e_3) \\
        (e_3,e_1) & (e_3,e_2) & e_3^2
    \end{pmatrix}.$$
Поэтому для случая $n$-мерного евклидова пространства мы можем записать скалярное произведение векторов в виде $(x,y)=\sum\limits_{i,j=1}^n x_iy_j\Gamma_{ij}$.

Для ортогонального базиса матрица Грама будет диагональной, для ортонормированного базиса будет единичной.

\subsection{Свойства матрицы Грама:}
\begin{enumerate}
    \item Симметричность: $(e_i,e_j)=(e_j,e_i)$.
    \item Все элементы на диагонали положительны. В самом деле, это скалярные квадраты базисных векторов, а векторы базиса ненулевые.
    \item  Наибольший элемент (или один из наибольших) находится на диагонали.
          Возьмём элемент базиса с наибольшим модулем. Тогда его скалярный квадрат будет больше произведения
          модулей каждых двух векторов и больше их скалярного произведения.
\end{enumerate}

\subsection{Примеры вычисления углов}
\begin{enumerate}
    \item Найти угол между векторами $x = \begin{pmatrix}
                  3 \\
                  1 \\
              \end{pmatrix}$ и $y= \begin{pmatrix}
                  1 \\
                  1 \\
              \end{pmatrix}$, если матрица Грама $\begin{pmatrix}
                  1  & -2 \\
                  -2 & 5  \\
              \end{pmatrix}$.

          $(x,y)=(x,y)=\sum\limits_{i,j=1}^2 x_iy_j\Gamma_{ij} = 3\cdot1 - 3\cdot2-1\cdot2+5=0$, следовательно векторы ортогональные и угол равен $\frac{\pi}{2}$.\\

          Ответ:$\frac{\pi}{2}$.\\
    \item Найти угол между векторами $x = \begin{pmatrix}
                  1 \\
                  1 \\
              \end{pmatrix}$ и $y= \begin{pmatrix}
                  2 \\
                  1 \\
              \end{pmatrix}$, если матрица Грама $\begin{pmatrix}
                  1  & -2 \\
                  -2 & 5  \\
              \end{pmatrix}$.\\
          $(x,y)=(x,y)=\sum\limits_{i,j=1}^2 x_iy_j\Gamma_{ij} = 2\cdot1 + 1\cdot(-2)+2\cdot(-2)+5=1$\\
          \begin{equation*}
              \begin{aligned}
                  |x|=\sqrt{(x,x)}    & =\sqrt{1-2-2+5}=\sqrt{2}                                               \\
                  |y|=\sqrt{(y,y)}    & =\sqrt{4-2\cdot2-2\cdot2+5}=1                                          \\
                  \measuredangle(x,y) & =\arccos \frac{(x,y)}{|x||y|}=\arccos \frac{1}{\sqrt{2}}=\frac{\pi}{4} \\
              \end{aligned}
          \end{equation*}

          Ответ: $\dfrac{\pi}{4}$
\end{enumerate}

\section{15. Инвариантные подпространства}
\subsection{Определение:}

Подпространство U линейного пространства L называется инвариантным относительно действия линейного оператора A, если $A(U)  \subset  L$.\\
Иными словами, действие оператора на инвариантном подпространстве не выводит за пределы этого подпространства.\\\\
\subsection{Примеры:}
\begin{itemize}
    \item ker $A$
    \item Im $A$
    \item $X(A,\lambda)$ - Множество всех собственных векторов оператора A для собственного числа $\lambda$
\end{itemize}
Рассмотрим проекцию трёхмерного пространства на плоскость YOZ.\\
Тогда инвариантными подпространствами являются:
\begin{itemize}
    \item плоскость YOZ
    \item каждая прямая в плоскости YOZ
    \item прямая OX
    \item нулевой вектор
    \item всё пространство
\end{itemize}
Конечно, нулевой вектор и всё пространство являются тривиальными примерами инвариантных подпространств.

\section{16. Сопряжённый оператор. Самосопряжённый оператор.}
\subsection{Сопряжённый оператор. Определение:}
Пусть A(x) - линейный оператор на евклидовом пространстве L.\\
Тогда оператор A* называют сопряжённым к A, если для каждой пары векторов выполнено равенство (A(x), y) = (x, A*(y))\\
Если А - матрица оператора A  в ортонормированном базисе, то матрица оператора A* равна матрице $A^T$, транспонированной к A.\\
В самом деле,(A$(x)^{T}$, y) = $x^T$$A^T$y = xA*y для всех столбцов координат x, y. \\
    Следовательно, $A* = A^T$

    \subsection{Самосопряжённый оператор. Определение:}
    Линейный оператор A на евклидовом пространстве называется самосопряжённым, если $<A(x), y)> = <x, A(y)>$

    \subsection{Примеры:}
    \begin{itemize}
        \item Тождественный оператор

              \quad Справа и слева - скалярное произведение векторов
        \item Гомотетия с коэффициентом t(вещественное число)

              \quad Справа и слева - скалярное произведение векторов, умноженное на t
        \item Проекция на плоскость в трёхмерном евклидовом пространстве с обычным скалярным произведением.

              \quad Пусть x = $x_1$ + $x_2$, $x_1$ - проекция на нормаль, $x_2$ - проекция на плоскость.

              \quad y = $y_1$ + $y_2$, $y_1$ - проекция на нормаль, $y_2$ - проекция на плоскость.

              \quad Тогда $<A(x), y> = <A(x_1 + x_2), y> = <x_2, y_1 + y_2> = <x_2, y_2>$

              \quad Тогда $<x, A(y)> = <(x_1 + x_2), A(y_1 + y_2)> = <x_1 + x_2, y_2> = <x_2, y_2>$

              \quad Следовательно, $<A(x), y> = <x, A(y)>$
    \end{itemize}

    \subsection{Свойство собственных векторов самосопряжённого оператора:}
    Собственные векторы самосопряжённого оператора, соответствующие разным собственным числам, ортогональны.

    \subsubsection{Доказательство:}

    Пусть $\lambda, \mu$ - собственные числа самосопряжённого оператора A.

    Тогда $A(x) = \lambda x, A(y) = \mu y$.

$\lambda \not \equiv \mu$.

$<A(x), y> = <x, A(y)>$

$\lambda <x, y> = \mu <x, y>$

$<\lambda - \mu><x, y> = 0$

    Поскольку ($\lambda - \mu \not\equiv 0$), получим <x, y> = 0

    Следовательно, x ортогонально y.

    \section{18. Метод Лагранжа (выделение полных квадратов)}
    Этот метод удобен для приведения к диагональному виду квадратичной формы, если собственные числа иррациональные. В общем случае:
    \[
        \sum_{i,j=1}^n a_{ij}x_ix_j = a_{11}x_1^2 + a_{12}x_1x_2 + ...
    \]
    Тогда алгоритм диагонализации:
    \begin{enumerate}
        \item Разделить на $a_{11}$ (или вынести за скобку)
        \item Рассмотреть все слагаемые, содержащие $x_1$
        \item Вместе с выражением $x_1^2$ выделить полный квадрат (возможно, прибавляя и вычитая квадраты остальных $x_i$), воспользовавшись формулой:
              \[
                  (\sum_{i = 1}^n a_i)^2 = \sum_{i = 1}^na_i^2 + 2\sum_{i,j=1, i<j}^na_ia_j
              \]
        \item Сделать замену переменной $\displaystyle y_1 = x_1 + \sum_{i=2}^nt_ix_i$
        \item В итоге квадратичная форма примет вид:
              \[
                  y_1^2 + \sum_{i,j=2}^na_{ij}x_ix_j
              \]
              где $\displaystyle \sum_{i,j=2}^na_{ij}x_ix_j$ — квадратичная форма с меньшим количеством слагаемых. После этого остаётся лишь повторять алгоритм до того момента, пока квадратичная форма не примет вид:
              \[
                  \sum_{i=1}^nb_iy_i^2
              \]
    \end{enumerate}

    \subsection{Пример}
    \[
        \begin{array}{l}
            4x_1^2 + x_2^2 + 3x_3^2 - 4x_1x_2 + 2x_1x_3 + 2x_2x_3                                                                                           \\
            \text{Поделим на } a_1 = 4                                                                                                                      \\
            x_1^2 + \dfrac{1}{4}x_2^2 + \dfrac{3}{4}x_3^2 - x_1x_2 + \dfrac{1}{2}x_1x_3 + \dfrac{1}{2}x_2x_3                                                \\
            \text{Выделим полный квадрат}                                                                                                                   \\
            (x_1 - \dfrac{1}{2}x_2 + \dfrac{1}{4}x_3)^2 = x_1^2 + \dfrac{1}{4}x_2^2 + \dfrac{1}{16}x_3^2 - x_1x_2 + \dfrac{1}{2}x_1x_3 - \dfrac{1}{4}x_2x_3 \\
            \text{Сделаем замену } y_1 = x_1 - \dfrac{1}{2}x_2 + \dfrac{1}{4}x_3                                                                            \\
            y_1^2 + \dfrac{11}{16}x_3^2 + \dfrac{3}{4}x_2x_3                                                                                                \\
            \text{Поделим на } \dfrac{11}{16}                                                                                                               \\
            \dfrac{16}{11}y_1^2 + x_3^2 + \dfrac{12}{11}x_2x_3                                                                                              \\
            \text{Выделим полный квадрат}                                                                                                                   \\
            (x_3 + \dfrac{6}{11}x_2)^2 = x_3^2 + \dfrac{36}{121}x_2^2 + \dfrac{12}{11}x_2x_3                                                                \\
            \text{Сделаем замену } y_2 = x_3 + \dfrac{6}{11}x_2                                                                                             \\
            \dfrac{16}{11}y_1^2 + y_2^2 - \dfrac{36}{121}x_2^2                                                                                              \\
            \text{Сделаем замену } y_3 = x_2                                                                                                                \\
            \dfrac{16}{11}y_1^2 + y_2^2 - \dfrac{36}{121}y_3^2 \text{ — диагональный вид}                                                                   \\
            \text{Сделав ещё одну замену } z_1 = \dfrac{4y_1}{\sqrt{11}}, z_2 = y_2, z_3 = \dfrac{6y_3}{11} \text{ можно прийти к каноническому виду}       \\
            z_1^2 + z_2^2 - z_3^2 \text{ — канонический вид}
        \end{array}
    \]
    \section{19. Закон инерции квадратичной формы}
    Закон инерции квадратичных форм гласит: число положительных, отрицательных и нулевых диагональных коэфициентов квадратичной формы не зависит от невырожденного преобразования, с помощью которого квадатичная форма приводится к диагональному виду.

    Число положительных диагональных коэфициентов квадратичной формы называется \textbf{положительным индексом инерции} квадратичной формы. Число отрицательных диагональных коэфициентов квадратичной формы называется \textbf{отрицательным индексом инерции} квадратичной формы. Разность между положительным и отрицательным индексами квадратичной формы называется \textbf{сигнатурой} квадратичной формы. Число ненулевых диагональных коэффициентов называется \textbf{рангом} квадратичной формы.

    \subsection{Пример}
    Приведём квадратичную форму $x_1x_2 + 2x_1x_3 + 4x_2x_3$ к диагональному виду:
\[
    \begin{cases}
        y_1 = \dfrac{1}{2}x_1 + \dfrac{1}{2}x_2 + 3x_3 \\
        y_2 = -x_1 + x_2 - 2x_3                        \\
        y_3 = x_3                                      \\
    \end{cases} \Rightarrow
    \begin{cases}
        x_1 = y_1 - \dfrac{1}{2}y_2 - 4y_3 \\
        x_2 = y_1 + \dfrac{1}{2}y_2 - 2y_3 \\
        x_3 = y_3                          \\
    \end{cases}
\]
\[
    \begin{cases}
        z_1 = \dfrac{1}{2}x_1 + \dfrac{1}{2}x_2 + 3x_3 \\
        z_2 = -\dfrac{1}{2}x_1 + \dfrac{1}{2}x_2 - x_3 \\
        z_3 = 2x_3                                     \\
    \end{cases} \Rightarrow
    \begin{cases}
        x_1 = z_1 - z_2 - 2z_3 \\
        x_2 = z_1 + z_2 - z_3  \\
        x_3 = \dfrac{1}{2}z_3  \\
    \end{cases}
\]
Получим два диагональных вида:
\[
    y_1^2 - \dfrac{1}{4}y_2^2 - 8y_3^2 \\ z_1^2 - z_2^2 - 2z_3^2
\]
При этом можно заметить, что оба вида имеют одинаковое количество положительных и отрицательных коэффициентов.

\section{20. Примеры приведения уравнения кривой второго порядка и поверхности второго порядка к канонической форме}
Выбор метода зависит от поставленной задачи. Вычисление с помощью собственных векторов и ортонормированного базиса позволяет применить движение плоскости или пространства и решить вычислительные задачи.

Преобразование методом Лагранжа удобнее применить, если собственные числа иррациональны. При этом некоторые величины (например, координаты центра при его наличии) найти удастся, но некоторые другие (например, координаты фокусов) не удастся. С применением закона инерции квадратичной формы можно определить тип кривой или поверхности, но вычислительные возможности ниже. Однако вычислений может потребоваться меньше.

\subsection{Приведение уравнения кривой второго порядка $11x^2 + 19y^2 + 6xy - 28x - 44y = 14$ к каноническому виду}
\[
    A = \begin{pmatrix}
        11 & 3  \\
        3  & 19
    \end{pmatrix}
    \\
    B = \begin{pmatrix}
        11 - \lambda & 3            \\
        3            & 19 - \lambda
    \end{pmatrix}
\]
\[
    |B| = \lambda^2 - 30\lambda + 200 = 0
    \\ \lambda_{1, 2} = 10; 20
\]
\[
    \lambda = 20:
    \begin{pmatrix}
        -9 & 3  \\
        3  & -1
    \end{pmatrix}
    \Rightarrow
    3x_1 = x_2
    \Rightarrow
    e_1 = t\begin{pmatrix}1 \\ 3\end{pmatrix}
\]
\[
    \lambda = 10:
    \begin{pmatrix}
        1 & 3 \\
        3 & 9
    \end{pmatrix}
    \Rightarrow
    x_1 = -3x_2
    \Rightarrow
    e_2  = t\begin{pmatrix}3 \\ -1\end{pmatrix}
\]
\[
    C = \dfrac{1}{\sqrt{10}}\begin{pmatrix}
        1 & 3  \\
        3 & -1
    \end{pmatrix} \text{ — ортогональная}
    \Rightarrow
    C^{-1} = C^T
\]
\[
    C^{-1} = C^T, C^T = C \Rightarrow C^{-1} = C
\]
\[
    \begin{cases}
        x = \dfrac{x' + 3y'}{\sqrt{10}} \\
        y = \dfrac{3x' - y'}{\sqrt{10}}
    \end{cases} \\
    \begin{cases}
        x' = \dfrac{x + 3y}{\sqrt{10}} \\
        y' = \dfrac{3x - y}{\sqrt{10}}
    \end{cases}
\]
\[
    \dfrac{11(x' + 3y')^2 + 19(3x' - y')^2 + 6(x' + 3y')(3x' - y')}{10} - \dfrac{28(x' + 3y') + 44(3x' - y')}{\sqrt{10}} = 14
\]
\[
    20{x'}^2 + 10{y'}^2 - 16\sqrt{10}x' - 4\sqrt{10}y' = 14
\]
\[
    20(x' - \dfrac{4}{\sqrt{10}})^2 - 32 + 10{y'}^2 - 4\sqrt{10}y' = 14
\]
\[
    20(x' - \dfrac{4}{\sqrt{10}})^2 - 32 + 10(y' - \dfrac{2}{\sqrt{10}})^2 - 4 = 14
\]
\[
    20(x' - \dfrac{4}{\sqrt{10}})^2 + 10(y' - \dfrac{2}{\sqrt{10}})^2 = 50 \\ |:50
\]
\[
    \dfrac{x' - \dfrac{4}{\sqrt{10}}}{\sqrt{\dfrac{5}{2}}}^2 + \dfrac{y' - \dfrac{2}{\sqrt{10}}}{\sqrt{5}}^2 = 1
\]
\[
    \begin{cases}
        x'' = y' - \dfrac{2}{\sqrt{10}} \\
        y'' = x' - \dfrac{4}{\sqrt{10}} \\
    \end{cases}
    \\
    \begin{cases}
        x' = y'' + \dfrac{4}{\sqrt{10}} \\
        y' = x'' + \dfrac{2}{\sqrt{10}} \\
    \end{cases}
\]
\[
    \dfrac{x''}{\sqrt{5}}^2 + \dfrac{y''}{\sqrt{\dfrac{5}{2}}}^2 = 1
\]
\subsection{Приведение поверхности второго порядка $4x^2 + y^2 + 3z^2 - 4xy + 2xz + 2yz - 2x + 4y + 2z = 0$ к канонической форме}
\[
    (2x - y - z)^2 = 4x^2 + y^2 + z^2 - 4xy - 4xz + 2yz
\]
\[
    (2x - y + z)^2 + 2(z^2 + 3xz) - 2x + 4y + 2z = 0
\]
\[
    2(z + \dfrac{3}{2}x)^2 = 2z^2 + 6xz + \dfrac{9}{2}x^2
\]
\[
    (2x - y - z)^2 + 2(z + \dfrac{3}{2}x)^2 - \dfrac{9}{2}x^2 - 2x + 4y + 2z = 0
\]
\[
    \begin{cases}
        x' = 2x - y - z        \\
        y' = z + \dfrac{3}{2}x \\
        z' = x
    \end{cases}
    \\
    \begin{cases}
        x = z'                        \\
        y = -x' - y' + \dfrac{7}{2}z' \\
        z = y' - \dfrac{3}{2}z'
    \end{cases}
\]
\[
    {x'}^2 + 2{y'}^2 - \dfrac{9}{2}{z'}^2 - 2z' - 4x' - 4y' + 14z' + 2y' - 3z' = 0
\]
\[
    {x'}^2 + 2{y'}^2 - \dfrac{9}{2}{z'}^2 - 4x' - 2y' + 9z' = 0
\]
\[
    (x' - 2)^2 - 4 + 2\left(y' - \dfrac{1}{2}\right)^2 - \dfrac{1}{2} - \dfrac{9}{2}\left(z' - 1\right)^2 + \dfrac{9}{2} = 0
\]
\[
    (x' - 2)^2 + 2\left(y' - \dfrac{1}{2}\right)^2 - \dfrac{9}{2}\left(z' - 1\right)^2 = 0
\]
\[
    \begin{cases}
        x'' = x' - 2            \\
        y'' = y' - \dfrac{1}{2} \\
        z'' = z' - 1
    \end{cases}
    \\
    \begin{cases}
        x' = x'' + 2            \\
        y' = y'' + \dfrac{1}{2} \\
        z' = z'' + 1
    \end{cases}
\]
\[
    {x''}^2 + \left(\dfrac{y''}{\frac{1}{\sqrt{2}}}\right)^2 - \left(\dfrac{z''}{\frac{\sqrt{2}}{3}}\right)^2 = 0
\]
\end{document}