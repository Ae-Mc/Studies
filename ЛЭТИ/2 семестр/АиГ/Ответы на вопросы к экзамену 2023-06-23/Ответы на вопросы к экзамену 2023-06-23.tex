\documentclass[12pt]{article}
\usepackage[a4paper, portrait, margin=1cm, bottom=2cm]{geometry}
\usepackage{fontspec}
\usepackage[fleqn]{amsmath}
\usepackage{amssymb}
\usepackage{indentfirst}
\usepackage{polyglossia}

\setmainfont[Ligatures=TeX]{Linux Libertine}
\setdefaultlanguage{russian}
\setotherlanguages{english}

\begin{document}

\section{5. Ядро и образ линейного оператора, их размерности.}
\subsection{Определение}
Пусть A - линейный оператор векторного пространства V.

Образ: множество $\forall y \in V : \exists\ x \in V : A(x) = y$. Обозначение: \emph{Im A}.

Ядро: множество $\forall x \in V : A(x) = 0$. Обозначение: \emph{Ker A}.

\subsection{Размерности Образа и Ядра}
A - линейный оператор в век. пространстве размерности n.

Размерность образа соответствует рангу матрицы оператора (Размерность $Im(A) = rk(A)$)

Размерность ядра соответствует дополнению ранга матрицы до n (Размерность $Ker(A) = n - rk(A)$)

\subsection{Доказательство подпространства}

\subsubsection{Образ}
Пусть $y_1, y_2 \in \emph{Im A}$, t - произвольное вещественное число $\Rightarrow \exists\ x_1, x_2 \in A : \begin{cases}
        A(x_1) = y_1 \\
        A(x_2) = y_2 \\
    \end{cases}
    \Rightarrow y_1 + y_2 = A(x_1)\ + A(x_2) = A(x_1 + x_2)$ и $ty_1 = tA(x_1) = A(tx_1)$ $\Rightarrow y_1 + y_2$ и ty - элементы образа $\Rightarrow$ образ - подпространство.

\subsubsection{Ядро}
Пусть $x_1, x_2 \in \emph{Ker A}$, t - произвольное вещественное число $\Rightarrow \begin{cases}
        A(x_1) = 0 \\
        A(x_2) = 0 \\
    \end{cases}
    \Rightarrow A(x_1+x_2) = A(x_1) + A(x_2) = 0 + 0 = 0, A(tx_1) = tA(x_1) = 0 \Rightarrow x_1 + x_2$ и tx - элементы ядра $\Rightarrow$ ядро - подпространство.

\subsection{Примеры}
Образ проекции пространства на плоскость $XOY$ – плоскость $XOY$, ядро этого оператора – все векторы, параллельные оси $OZ$.

Образ оператора дифференцирования на пространстве всех многочленов – это же пространство многочленов. Ядро этого оператора – константы.

\section{6. Алгоритм Чуркина}
\subsection{Алгоритм}
Имеется матрица оператора A размерности n. Составим матрицу B порядка n x 2n, где первые n столбцов занимает транспонированная матрица A, следующие n столбцов единичная матрица.

Элементарными преобразованиями приводим левую часть к ступенчатому виду.

Ненулевые строки левой части - базис образа оператора A.

Продолжение нулевых строк в правой части - базис ядра.

\subsection{Обоснование}
Строки $A^T$ соответствуют значению оператора на базисе.

Записав значения и приведя к ступенчатому виду, получаем базис (базис Образа).

Левая часть матрицы B соответствует значению оператора на векторе, тогда как правая - координатной записи вектора.После преобразований закономерность сохраняется и в правой части: строки соответствующие нулевым строкам в левой являются базисом ядра. (P.S. Вспомни определение ядра)

\subsection{Пример}
Пусть $A = \left(\begin{array}{cccc}
            2  & 0 & 1 & -3 \\
            1  & 0 & 3 & -4 \\
            -1 & 0 & 2 & -1 \\
            1  & 0 & 1 & -2
        \end{array} \right)$
По алгоритму:

$\left(\begin{array}{cccc|cccc}
            2  & -1 & -1 & 1  & 1 & 0 & 0 & 0 \\
            0  & 0  & 0  & 0  & 0 & 1 & 0 & 0 \\
            1  & 3  & 2  & 1  & 0 & 0 & 1 & 0 \\
            -3 & -4 & -1 & -2 & 0 & 0 & 0 & 1
        \end{array} \right)$ перемещаем строку $\left(\begin{array}{cccc|cccc}
            2  & -1 & -1 & 1  & 1 & 0 & 0 & 0 \\
            1  & 3  & 2  & 1  & 0 & 0 & 1 & 0 \\
            -3 & -4 & -1 & -2 & 0 & 0 & 0 & 1 \\
            0  & 0  & 0  & 0  & 0 & 1 & 0 & 0
        \end{array} \right)$

$l_3 \rightarrow l_3 + l_2 + l_1$ получаем $\left(\begin{array}{cccc|cccc}
            2 & -1 & -1 & 1 & 1 & 0 & 0 & 0 \\
            1 & 3  & 2  & 1 & 0 & 0 & 1 & 0 \\
            0 & 0  & 0  & 0 & 1 & 0 & 1 & 1 \\
            0 & 0  & 0  & 0 & 0 & 1 & 0 & 0
        \end{array} \right)$

Тогда: $(2,-1,-1,1), (1,3,2,1)$ - базис образа, $(1,0,1,1),(0,1,0,0)$ -базис ядра.

\section{7. Проверка линейного оператора на вырожденность и невырожденность}
\subsection{Невырожденная матрица оператора}
\subsection{Оператор вырожденный, если $\exists\ x_1, x_2 : A(x_1) = A(x_2) = y$. Тогда невозможно однозначно определить обратный оператор от y}
\subsection{Оператор невырожденный только тогда, когда ядро состоит только из нулевого элемента}
Оператор невырожденный $\Rightarrow$ ядро не может состоять больше чем из одного элемента (пункт 2) $\Rightarrow$ ядро состоит только из 0.

Ядро состоит из 0 $\Rightarrow A(x_1) = A(x_2)$ будет:

$A(x_1) - A(x_2) = 0 \Rightarrow A(x_1 - x_2) = 0 \Rightarrow x_1 - x_2 = 0 \Rightarrow x_1 = x_2$

Совпадение значений оператора на двух элемента $\Rightarrow$ совпадают элементы $\Rightarrow$ возможно корректное определение обратного оператора.

\subsection{Геометрические соображения}
Пример:

Вырождена ли симметрия относительно плоскости? Легко заметить, что двойная симметрия вернёт точку в исходное положение. Таким образом обратный к симметрии - это сама симметрия, а значит она невырождена.

\subsection{Примеры}
\begin{enumerate}
    \item $A(x) = a * x$, где $a \in R$.

          Ядро состоит только из нулевого элемента тогда и только тогда, когда $a \neq 0$.

          Если $a = 0$, оператор вырожден, поскольку прообраз 0 неоднозначно определён.

    \item $A(x) = (x,e)e$, при этом $|e| = 1$

          Оператор вырожден, поскольку ядро состоит не только из 0.
\end{enumerate}

\section{8. Собственные числа}
\subsection{Определение}
Число $\lambda$ называется собственным числом оператора $L$, если существует такой ненулевой вектор $x$, что $L(x) = \lambda x$. При этом вектор $x$ называется собственным вектором оператора $L$, отвечающим собственному числу $\lambda$.

\subsection{Вычисление}
\subsubsection{Геометрический}
Пример: Найти с.ч. и с.в. проекции пространства на плоскость $XOY$.

Найдём все векторы || своей проекции - это все вектора || $XOY$ (проекция равна вектору) и все вектора перпендикулярные плоскости (проекция равна нулю). $\Rightarrow$

$\lambda_1 = 1, X_{\lambda_1} = \left(\begin{array}{c}
            x \\
            y \\
            0
        \end{array} \right); \lambda_2 = 0, X_{\lambda_2} = \left(\begin{array}{c}
            0 \\
            0 \\
            z
        \end{array} \right)$, где $x, y, z \in R$.
\subsubsection{Аналитический}
Пример: Найти с.ч. и с.в. оператора дифференцирования на множестве всех многочленов.

Рассмотрим $p^\prime(x) = \lambda p(x)$. Если $\lambda \neq 0$, то степени правой и левой частей не совпадут. Если $\lambda = 0$, то p(x) - константа.

Ответ: $\lambda = 0, X_\lambda = C = const$.

\subsubsection{С помощью матрицы}
Пусть оператор задан матрицей $\Rightarrow A(x) = Ax \Rightarrow$

\begin{center}
    $Ax = \lambda x \Leftrightarrow Ax - \lambda x = 0 \Leftrightarrow Ax - \lambda Ex = 0 \Leftrightarrow (A - \lambda E)x = 0$
\end{center}

$\lambda$ - с.ч. только тогда, когда система уравнений на координаты x $(A - \lambda E)x = 0$ имеет ненулевое решение.

Выполняется только тогда, когда определитель $(A - \lambda E)$ равен 0.

Алгоритм: \begin{enumerate}
    \item Найти корни уравнения $det(A - \lambda E) = 0$
    \item Для каждого из корней решить СЛУ $(A - \lambda E)x = 0$. Решения будут задавать координаты множества с.в., соответствующих с.ч. (может иметь размерность 1 и более)
\end{enumerate}
\subsection{Диагонализуемость}

Если удаётся найти базис из с.в. для линейного оператора, то его матрица в этом базисе будет диагональна, а на диагонали будут с.ч.

Пусть матрица диагональна, значит элементы диагонали обозначим через $\lambda_k$ из условия $A(e_k) = \lambda_k e_k$ получим, что по определению все элементы базиса - с.в. данного линейного оператора.

Итог: матрица линейного оператора диагонализуема тогда только тогда, когда для этого линейного оператора найдётся базис из собственных векторов.

\end{document}