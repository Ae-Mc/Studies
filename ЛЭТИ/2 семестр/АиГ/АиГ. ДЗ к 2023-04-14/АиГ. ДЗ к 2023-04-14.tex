\documentclass[12pt]{article}
\usepackage[a4paper, portrait, margin=1cm, right=1cm]{geometry}
\usepackage{fontspec}
\usepackage[fleqn]{amsmath}
\usepackage{setspace}

\setmainfont[Ligatures=TeX]{Linux Libertine}

\title{АиГ. ДЗ к 2023-04-14. Вариант №14}
\author{Студент группы 2305 Александр Макурин}
\date{07 апреля 2023}

\begin{document}

\maketitle

\begin{sloppypar}
    \section{Найдите собственные числа и собственные вектора матриц}
    \subsection{$\begin{pmatrix}
                -7 & -2 & 2 \\
                0  & -6 & 1 \\
                -4 & -5 & 0
            \end{pmatrix}$}

    \begin{align*}
        \begin{vmatrix}
            -7 - \lambda & -2           & 2           \\
            0            & -6 - \lambda & 1           \\
            -4           & -5           & 0 - \lambda
        \end{vmatrix}
         & = -(7 + \lambda)(\lambda(\lambda + 6) + 5) -4(-2 + 12 + 2\lambda) =                   \\
         & =-(7\lambda^2 + 42\lambda + 35 + \lambda^3 + 6\lambda^2 + 5\lambda) - 8\lambda - 40 = \\
         & = -\lambda^3 - 13\lambda^2 - 55\lambda - 75
    \end{align*}

    \begin{align*}
        -\lambda^3 - 13\lambda^2 - 55\lambda - 75 = 0 \\
        \lambda^3 + 13\lambda^2 + 55\lambda + 75 = 0  \\
        \lambda_1 = -3                                \\
        (\lambda + 3)(\lambda^2 + 10\lambda + 25) = 0 \\
        D = 10^2 - 4 \cdot 1 \cdot 25 = 0             \\
        \lambda_{2,3} = \dfrac{-10}{2} = -5           \\
    \end{align*}

    Нахождение $u_1$ — первого собственного вектора
    \begin{align*}
         & \begin{pmatrix}
               -4 & -2 & 2 \\
               0  & -3 & 1 \\
               -4 & -5 & 3
           \end{pmatrix}
        \sim
        \begin{pmatrix}
            -4 & -2 & 2 \\
            0  & -3 & 1 \\
            0  & -3 & 1
        \end{pmatrix}
        \sim
        \begin{pmatrix}
            -4 & 1 & 1  \\
            0  & 3 & -1 \\
            0  & 0 & 0
        \end{pmatrix}
        \Rightarrow       \\
         & \Rightarrow
        \left.\begin{cases}
                  -4x_1 + x_2 + x_3 = 0 \\
                  3x_2 - x_3 = 0        \\
              \end{cases}\right|
        \Rightarrow
        \begin{cases}
            x_3 = 3x_2 \\
            x_1 = x_2
        \end{cases}
        \Rightarrow
        u_1 = \begin{pmatrix}
                  1 \\ 1 \\ 3
              \end{pmatrix}
    \end{align*}

    Нахождение $u_{2,3}$
    \begin{align*}
         & \begin{pmatrix}
               -2 & -2 & 2 \\
               0  & -1 & 1 \\
               -4 & -5 & 5
           \end{pmatrix}
        \sim
        \begin{pmatrix}
            -2 & -2 & 2 \\
            0  & -1 & 1 \\
            0  & -1 & 1
        \end{pmatrix}
        \sim
        \begin{pmatrix}
            -4 & 0  & 0 \\
            0  & -1 & 1 \\
            0  & 0  & 0
        \end{pmatrix}
        \Rightarrow       \\
         & \Rightarrow
        \left.\begin{cases}
                  -4x_1 = 0      \\
                  -x_2 + x_3 = 0 \\
              \end{cases}\right|
        \Rightarrow
        \begin{cases}
            x_3 = x_2 \\
            x_1 = 0
        \end{cases}
        \Rightarrow
        u_{2,3} = \begin{pmatrix}
                      0 \\ 1 \\ 1
                  \end{pmatrix}
    \end{align*}

    \fbox{Ответ: $\lambda_1 = -3$; $\lambda_2 = -5$; $u_1 = \begin{pmatrix}1 \\ 1 \\ 3 \end{pmatrix}$; $u_2 = \begin{pmatrix} 0 \\ 1 \\ 1 \end{pmatrix}$}

    \subsection{$\begin{pmatrix}
                1  & -7  & -3 \\
                2  & 10  & 3  \\
                -4 & -10 & -1
            \end{pmatrix}$}

    \begin{align*}
         & \begin{vmatrix}
               1 - \lambda & - 7          & -3           \\
               2           & 10 - \lambda & 3            \\
               -4          & -10          & -1 - \lambda
           \end{vmatrix} =                                                             \\
         & = (1 - \lambda)((10 - \lambda)(-1 - \lambda) + 30) + 7(-2 - 2\lambda + 12) - 3(-20 + 40 - 4\lambda) = \\
         & = (1 - \lambda)(-10 - 10\lambda + \lambda + \lambda^2 + 30) + 70 - 14\lambda - 60 + 12\lambda =       \\
         & = 20 - 9\lambda + \lambda^2 - 20\lambda + 9\lambda^2 - \lambda^3 + 10 - 2\lambda =                    \\
         & = -\lambda^3 + 10\lambda^2 - 31\lambda + 30
    \end{align*}

    \begin{align*}
        -\lambda^3 + 10\lambda^2 - 31\lambda + 30 = 0 \\
        \lambda^3 - 10\lambda^2 + 31\lambda - 30 = 0  \\
        \lambda_1 = 2                                 \\
        (\lambda - 2)(\lambda^2 - 8\lambda + 15) = 0  \\
        D = 8^2 - 4 \cdot 1 \cdot 1 = 4               \\
        \lambda_{2,3} = \dfrac{8 \pm 2}{2} = 3; 5
    \end{align*}

    Нахождение $u_1$
    \begin{align*}
         & \begin{pmatrix}
               -1 & -7  & -3 \\
               2  & 8   & 3  \\
               -4 & -10 & -3
           \end{pmatrix}
        \sim
        \begin{pmatrix}
            -1 & -7 & -3 \\
            0  & -6 & -3 \\
            0  & 18 & 9
        \end{pmatrix}
        \sim
        \begin{pmatrix}
            1 & 1 & 0 \\
            0 & 2 & 1 \\
            0 & 0 & 0
        \end{pmatrix} \Rightarrow \\
         & \Rightarrow
        \left.\begin{cases}
                  x_1 + x_2 = 0 \\
                  2x_2 + x_3 = 0
              \end{cases}\right|
        \Rightarrow
        \left.\begin{cases}
                  x_1 = -x_2 \\
                  x_3 = -2x_2
              \end{cases}\right|
        \Rightarrow
        u_1 = \begin{pmatrix}
                  1 \\ -1 \\ 2
              \end{pmatrix}
    \end{align*}

    Нахождение $u_2$
    \begin{align*}
         & \begin{pmatrix}
               -2 & -7  & -3 \\
               2  & 7   & 3  \\
               -4 & -10 & -4
           \end{pmatrix}
        \sim
        \begin{pmatrix}
            -2 & -7 & -3 \\
            0  & 0  & 0  \\
            0  & 4  & 2
        \end{pmatrix}
        \sim
        \begin{pmatrix}
            2 & 1 & 0 \\
            0 & 0 & 0 \\
            0 & 2 & 1
        \end{pmatrix}
        \Rightarrow       \\
         & \Rightarrow
        \left.\begin{cases}
                  2x_1 = -x_2 \\
                  2x_2 = -x_3
              \end{cases}\right|
        \Rightarrow
        u_2 = \begin{pmatrix}
                  1 \\ -2 \\ 4
              \end{pmatrix}
    \end{align*}

    Нахождение $u_3$

    \begin{align*}
         & \begin{pmatrix}
               -4 & -7  & -3 \\
               2  & 5   & 3  \\
               -4 & -10 & -6
           \end{pmatrix}
        \sim
        \begin{pmatrix}
            2 & 5  & 3  \\
            0 & 3  & 3  \\
            0 & -3 & -3
        \end{pmatrix}
        \sim
        \begin{pmatrix}
            1 & 1 & 0 \\
            0 & 1 & 1 \\
            0 & 0 & 0
        \end{pmatrix}
        \Rightarrow       \\
         & \Rightarrow
        \left.\begin{cases}
                  x_1 = -x_2 \\
                  x_2 = -x_3
              \end{cases}\right|
        \Rightarrow
        u_3 = \begin{pmatrix}
                  1 \\ -1 \\ 1
              \end{pmatrix}
    \end{align*}

    \fbox{Ответ: $\lambda_1 = 2$;
        $\lambda_2 = 3$;
        $\lambda_3 = 5$;
        $u_1 = \begin{pmatrix}1 \\ -1 \\ 2 \end{pmatrix}$;
        $u_2 = \begin{pmatrix} 1 \\ -2 \\ 4 \end{pmatrix}$;
        $u_3 = \begin{pmatrix} 1 \\ -1 \\ 1 \end{pmatrix}$}



\end{sloppypar}
\end{document}