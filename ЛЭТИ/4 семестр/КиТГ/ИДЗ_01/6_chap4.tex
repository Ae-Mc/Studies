\vspace{20pt}

{\let\clearpage\relax
    \chapter{}}
% Ответьте, применим ли к этому отношению алгоритм топологической сортировки; если алгоритм применим, примените его; приведите протокол работы алгоритма, интерпретируя его на графе и матрице смежности (для определенности при проверке, при наличии нескольких минимальных элементов договоримся выбирать первый в лексикографическом порядке); дайте объяснение смыслу алгоритма топологической сортировки. В качестве ответа привести линейно упорядоченные элементы множества.

\section{Ответьте, применим ли к этому отношению алгоритм топологической сортировки}
Неприменим, т. к. граф не ациклический.
\section{Примените его}
Алгоритм неприменим.
\section{Приведите протокол работы алгоритма, интерпретируя его на графе и матрице смежности}
Протокол отсутствует, т. к. алгоритм неприменим.
\section{Дайте объяснение смыслу алгоритма топологической сортировки. В качестве ответа привести линейно упорядоченные элементы множества}
Алгоритм топологической сортировки используется для превращения частично строго упорядоченного множества в линейно упорядоченное. Линейно упорядоченные элементы привести нельзя, т. к. алгоритм неприменим.

\endinput