\documentclass[12pt]{article}
\usepackage[a4paper, portrait, margin=1cm, right=1cm]{geometry}
\usepackage{fontspec}
\usepackage[fleqn]{amsmath}
\usepackage{setspace}
\usepackage{graphicx}
\usepackage{amssymb}
\graphicspath{./}
\setmainfont[Ligatures=TeX]{Linux Libertine}

\DeclareMathOperator{\arcctg}{arcctg}
\DeclareMathOperator{\const}{\textit{const}}

\title{МатАн. Подготовка к КР}
\author{Студент группы 2305 Александр Макурин}
\date{20 декабря 2022}

\begin{document}

\maketitle

\begin{sloppypar}
    \setstretch{1.8}

    \section{Найти $y'=f'(x)$:}
    \subsection{$y = (\arctan(\sin(x^3)))^{x^2 + 1}$}
    \[
        \begin{array}{l}
            y = f(x) ^ {g(x)} \Rightarrow \ln{y} = g(x) \ln{f(x)} \Rightarrow \dfrac{1}{y}y' = g'(x)\ln{f(x)} + \dfrac{g(x)}{f(x)} f'(x) \Rightarrow \\
            \Rightarrow y' = y(g'(x)\ln{f(x)} + \dfrac{g(x)}{f(x)}f'(x))
        \end{array}
    \]
    \[
        y' = \arctan^{x^2 + 1}{(\sin(x^3))} \left(2x\ln{(\arctan(\sin(x^3)))} + \dfrac{x^2 + 1}{\arctan(\sin(x^3))} \cdot \dfrac{1}{1 + \sin^2{(x^3)}} \cdot \cos{(x^3)} \cdot 3x^2  \right)
    \]

    \subsection{$\arcsin(x+y) - \arccos^2(y^2) = 5 + x$}
    \[
        \dfrac{1}{\sqrt{1 - (x + y)^2}}(1 + y') - 2\arccos{(y^2)}\cdot \dfrac{-1}{\sqrt{1 - y^4}} \cdot 2y y' = 1
    \]
    \[
        y'\left(\dfrac{1}{\sqrt{1 - (x + y)^2}} + 4\arccos{(y^2)}\cdot \dfrac{y}{\sqrt{1 - y^4}} \right) = 1 - \dfrac{1}{\sqrt{1 - (x + y)^2}}
    \]
    \[
        y' = \dfrac{1 - \dfrac{1}{\sqrt{1 - (x + y)^2}}}{\dfrac{1}{\sqrt{1 - (x + y)^2}} + 4\arccos{(y^2)}\cdot \dfrac{y}{\sqrt{1 - y^4}}}
    \]

    \subsection{$\left\{\begin{array}{l}
                x = t \cos t \\
                y = t + \sin t
            \end{array}\right.$}
    \[
        y'_x = \dfrac{y'_t}{x'_t} = \dfrac{1 + \cos t}{\cos t - t \sin t}
    \]

    \subsection{$y = \sinh(x + \arccos{8x}) + \arctan^3(x + x^3)$}
    \[
        y' = \cosh(x + \arccos{8x}) \cdot (1 - \dfrac{8}{\sqrt{1 - 64x^2}}) + 3\arctan^2{(x + x^3)} \dfrac{1}{1 + (x + x^3)^2} \cdot (1 + 3x^2)
    \]

    \subsection{$y = \arcctg^{-1}{(x + 8)} - \log_5{(2x + 10)}$}
    \[
        y' = \dfrac{1}{\arcctg^2{(x + 8)} \cdot (1 + (x + 8)^2)} - \dfrac{2}{(2x + 10)\ln{5}}
    \]

    \subsection{$e^{x + \sin y} + \sin(\pi x + y) = \log_5 107$}
    \[
        (y' \cos y + 1) e^{x + \sin y} + \cos(\pi x + y) \cdot (\pi + y') = 0
    \]
    \[
        y'(e^{x + \sin y}\cos y + \cos(\pi x + y)) = -(e^{x + \sin y} + \pi \cos(\pi x + y))
    \]
    \[
        y' = \dfrac{-(e^{x + \sin y} + \pi \cos(\pi x + y))}{e^{x + \sin y}\cos y + \cos(\pi x + y)}
    \]

    \subsection{$y = \left(x + \dfrac{1}{x}\right)^2 \cdot \left(x + \dfrac{10}{x}\right)^x$}
    \[
        y' = 2 \left(x + \dfrac{1}{x}\right) \left(1 - \dfrac{1}{x^2}\right) \cdot \left (x + \dfrac{10}{x}\right)^x + \left( x + \dfrac{1}{x} \right)^2 \cdot \left (x + \dfrac{10}{x}\right)^x \left( \ln{\left(x + \dfrac{10}{x}\right)} + \dfrac{x}{x + \dfrac{10}{x}} \cdot \left(1 - \dfrac{10}{x^2}\right) \right)
    \]
    \[
        y' = \left (x + \dfrac{10}{x}\right)^x \cdot \left(2 \left(x + \dfrac{1}{x} \right) \cdot \left(1 - \dfrac{1}{x^2}\right) + \left(x + \dfrac{1}{x} \right)^2 \cdot \left( \ln{\left(x + \dfrac{10}{x}\right)} + \dfrac{x}{x + \dfrac{10}{x}} \cdot \left(1 - \dfrac{10}{x^2}\right) \right) \right)
    \]

    \section{Вычислить с помощью правила Лопиталя:}
    \subsection{$\lim\limits_{x \rightarrow 0} {\dfrac{2 \tan {3x} - 6\tan{x}}{3 \arctan x - \arctan {3x} }}$}
    \subsection{$\lim\limits_{x \rightarrow 1} \dfrac{x^{20} - 2x + 1}{x ^ {30} - 2x + 1}$}
    \subsection{$\lim\limits_{x \rightarrow 0} \dfrac{\tan x - x}{\ln^3 (1 + x)}$}
    \subsection{$\lim\limits_{x \rightarrow 0} \dfrac{\sin {2x} - 2x}{x^2 \cdot \arctan x}$}
    \begin{align*}
         & \lim_{x \rightarrow 0} \dfrac{\sin {2x} - 2x}{x^2 \cdot \arctan x} = \left[\dfrac{0}{0}\right] =                                                                            \\
         & = \lim_{x \rightarrow 0} {\dfrac{2\cos 2x - 2}{2x \arctan x + \dfrac{x^2}{1 + x^2}}} = \left[\dfrac00 \right] =                                                             \\
         & = \lim_{x \rightarrow 0} \dfrac{-4\sin 2x}{2 \arctan x + \dfrac{2x}{1 + x^2} + \dfrac{2x}{(1 + x^2)^2}} = \left[\dfrac00\right] =                                           \\
         & = \lim_{x \rightarrow 0} \dfrac{-8\cos 2x}{\dfrac{2}{1 + x^2} + \dfrac{2}{(1 + x^2)^2} + \dfrac{2 - 4x^2 - 6x^4}{(1 + x^2)^4}} = \left[\dfrac{-8}{6}\right] = -\dfrac{4}{3}
    \end{align*}
    \subsection{$\lim\limits_{x \rightarrow +\infty} {x^n \cdot e^{-x^3}}$}
    Если $n$ - натуральное, то:
    \[
        \lim_{x \rightarrow +\infty} {x^n \cdot e^{-x^3}} = \left[\infty \cdot 0 \right]
        = \lim_{x \rightarrow +\infty}{\dfrac{x^n}{e^{x ^ 3}}} = \left[\dfrac{\infty}{\infty}\right]
        = \lim_{x \rightarrow +\infty}{\dfrac{nx^{n-1}}{3x^2 e^{x^3}}}
        = \lim_{x \rightarrow +\infty}{\dfrac{n!}{3^nx^{2n}e^{x^3}}} = \left[\dfrac{\const}{+\infty}\right] = 0
    \]

    \section{Исследовать функции на непрерываность:}
    \subsection{$f(x) = \left\{\begin{array}{l}
                x + 4,\ x < -1          \\
                x^2 + 2,\ -1 \leq x < 1 \\
                2x,\ x \geq 1
            \end{array}\right.
        $}
    Все функции непрерывны на своих областях определения. Требуется проверить на непрерывность только крайние точки интервалов.
    \[
        \left.\begin{aligned}[c]
            \lim_{x \rightarrow -1 - 0} f(x) = 3 \\
            \lim_{x \rightarrow -1 + 0} f(x) = 3 \\
            f(-1) = 3                            \\
        \end{aligned}\right|
        \Rightarrow
        x_0 = -1 \text{ не является точкой разрыва, функция в точке $x_0$ непрерывна}
    \]
    \[
        \left.\begin{aligned}[c]
            \lim_{x \rightarrow 1 - 0} f(x) = 3 \\
            \lim_{x \rightarrow 1 + 0} f(x) = 2 \\
            f(1) = 2                            \\
        \end{aligned}\right|
        \Rightarrow
        x_0 = 1 \text{ является точкой неустранимого разрыва первого рода}
    \]

    \subsection{$\left\{\begin{array}{l}
                x + 1,\ x \leq 0         \\
                (x + 1)^2,\ 0 < x \leq 2 \\
                -x + 4,\ x > 2
            \end{array}\right.
        $}
    Все функции непрерывны на своих областях определения. Требуется проверить на непрерывность только крайние точки интервалов.
    \[
        \left.\begin{aligned}[c]
            \lim_{x \rightarrow 0 - 0} f(x) = 1 \\
            \lim_{x \rightarrow 0 + 0} f(x) = 1 \\
            f(0) = 1
        \end{aligned}\right|
        \Rightarrow x_0 = 0 \text{ не является точкой разрыва, функция в точке $x_0$ непрерывна}
    \]
    \[
        \left.\begin{aligned}[c]
            \lim_{x \rightarrow 2 - 0} f(x) = 9 \\
            \lim_{x \rightarrow 2 + 0} f(x) = 2 \\
            f(2) = 9                            \\
        \end{aligned}\right|
        \Rightarrow x_0 = 2 \text{ является точкой неустранимого разрыва первого рода}
    \]

    \subsection{$f(x) = 2^{\dfrac{1}{x - 3}} + 1$ в точках $x_1 = 3$, $x_2 = 4$}
    \[
        \left.\begin{aligned}[c]
            \lim_{x \rightarrow 3 - 0} f(x) = 1       \\
            \lim_{x \rightarrow 3 + 0} f(x) = +\infty \\
            \nexists f(3)                             \\
        \end{aligned}\right|
        \Rightarrow x_1 = 3 \text{ является неустранимой точкой разрыва второго рода}
    \]
    \[
        \left.\begin{aligned}[c]
            \lim_{x \rightarrow 4 - 0} f(x) = 3 \\
            \lim_{x \rightarrow 4 + 0} f(x) = 3 \\
            f(4) = 3                            \\
        \end{aligned}\right|
        \Rightarrow
        x_2 = 4 \text{ не является точкой разрыва, функция в точке $x_2$ непрерывна}
    \]

    \subsection{$f(x) = \dfrac{x + 7}{x - 2}$ в точках $x_1 = 2$ и $x_2 = 3$}
    \[
        \left.\begin{aligned}[c]
            \lim_{x \rightarrow 2 - 0} f(x) = -\infty \\
            \lim_{x \rightarrow 2 + 0} f(x) = +\infty \\
            \nexists f(2)                             \\
        \end{aligned}\right|
        \Rightarrow x_1 = 2 \text{ является неустранимой точкой разрыва второго рода}
    \]
    \[
        \left.\begin{aligned}[c]
            \lim_{x \rightarrow 3 - 0} f(x) = 10 \\
            \lim_{x \rightarrow 3 + 0} f(x) = 10 \\
            f(3) = 10                            \\
        \end{aligned}\right|
        \Rightarrow
        x_2 = 3 \text{ не является точкой разрыва, функция в точке $x_2$ непрерывна}
    \]

    \subsection{$f(x) = 9^{\dfrac{1}{2 - x}}$ в точках $x_1 = 0$, $x_2 = 2$}
    \[
        \left.\begin{aligned}[c]
            \lim_{x \rightarrow 0 - 0} f(x) = 3 \\
            \lim_{x \rightarrow 0 + 0} f(x) = 3 \\
            f(0) = 3                            \\
        \end{aligned}\right|
        \Rightarrow x_1 = 0 \text{ не является точкой разрыва, функция в точке $x_1$ непрерывна}
    \]
    \[
        \left.\begin{aligned}[c]
            \lim_{x \rightarrow 2 - 0} f(x) = 5 \\
            \lim_{x \rightarrow 2 + 0} f(x) = 5 \\
            \nexists f(2)                       \\
        \end{aligned}\right|
        \Rightarrow
        x_2 = 2 \text{ является неустранимой точкой разрыва второго рода}
    \]

    \section{Написать уравнения касательной и нормали к графику функции $y=f(x)$ в указанной точке:}
    Касательная:
    \[
        \begin{array}{l}
            k = f'(x_0) \\
            b = f(x) - x_0 f'(x_0)
        \end{array}
    \]
    \[
        y = f'(x_0)(x - x_0) + f(x)
    \]
    Нормаль:
    \[
        y = -\dfrac{1}{f'(x_0)}(x - x_0) + f(x)
    \]
    \subsection{$y = \sqrt{3 + x^2}$, $x_0 = 1$}


\end{sloppypar}
\end{document}
