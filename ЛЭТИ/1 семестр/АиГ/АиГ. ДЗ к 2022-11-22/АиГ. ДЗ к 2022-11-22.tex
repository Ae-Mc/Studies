\documentclass[12pt]{article}
\usepackage[a4paper, portrait, margin=1cm, right=1cm]{geometry}
\usepackage{fontspec}
\usepackage[fleqn]{amsmath}
\usepackage{setspace}

\setmainfont[Ligatures=TeX]{Linux Libertine}

\title{АиГ. ДЗ к 22.11.2022. Вариант №13}
\author{Студент группы 2305 Александр Макурин}
\date{20 ноября 2022}

\begin{document}

\maketitle

\begin{sloppypar}
    \setstretch{1.8}

    \section{Вычислить произведение матриц}
    \[
        \begin{pmatrix}
            -1 & 1 & -2 \\
            1  & 3 & 4
        \end{pmatrix}
        \cdot
        \begin{pmatrix}
            3  & -1 \\
            3  & -1 \\
            -1 & -3
        \end{pmatrix}
        =
        \begin{pmatrix}
            -1 \cdot 3 + 1 \cdot 3 - 2 \cdot -1 & -1 \cdot -1 + 1 \cdot -1 - 2 \cdot -3 \\
            1 \cdot 3 + 3 \cdot 3 + 4 \cdot -1  & 1 \cdot -1 + 3 \cdot -1 + 4 \cdot -3  \\
        \end{pmatrix}
        =
        \begin{pmatrix}
            2 & 6   \\
            8 & -16
        \end{pmatrix}
    \]

    \framebox{Ответ:
        $\begin{pmatrix}
                2 & 6   \\
                8 & -16
            \end{pmatrix}$
    }

    \section{Вычислить определитель:}
    \[
        \begin{vmatrix}
            -2 & 1  & -1 & -1 \\
            1  & 2  & -1 & 0  \\
            -2 & 2  & 2  & -1 \\
            1  & -2 & -1 & -1
        \end{vmatrix}
        =
        -(-1) \begin{vmatrix}
            1  & 2  & -1 \\
            -2 & 2  & 2  \\
            1  & -2 & -1
        \end{vmatrix}
        -(-1) \begin{vmatrix}
            -2 & 1  & -1 \\
            1  & 2  & -1 \\
            1  & -2 & -1
        \end{vmatrix}
        +(-1) \begin{vmatrix}
            -2 & 1 & -1 \\
            1  & 2 & -1 \\
            -2 & 2 & 2
        \end{vmatrix}
    \]
    \[
        = (-2 + 4 -2(2-2) - (4-2)) + (-2(-2-2)-(-1+1)-(-2-2)) - (-2(4+2) - (2 - 2) - (2 + 4)) =
    \]
    \[
        = 0 + 12 + 18 = 30
    \]
    \framebox{Ответ: $30$}

    \section{Вычислить $A^{-1}B^{-1}AB$, если}

    \[
        A = \begin{pmatrix}
            1 & 0 & 3 \\
            0 & 1 & 0 \\
            0 & 0 & 1
        \end{pmatrix}
        B = \begin{pmatrix}
            1 & 0  & 0 \\
            0 & 1  & 0 \\
            0 & -5 & 1
        \end{pmatrix}
    \]
    \[
        \left(
        \begin{array}{ccc|ccc}
                1 & 0 & 3 & 1 & 0 & 0 \\
                0 & 1 & 0 & 0 & 1 & 0 \\
                0 & 0 & 1 & 0 & 0 & 1
            \end{array}
        \right)
        \sim
        \left(
        \begin{array}{ccc|ccc}
                1 & 0 & 0 & 1 & 0 & -3 \\
                0 & 1 & 0 & 0 & 1 & 0  \\
                0 & 0 & 1 & 0 & 0 & 1
            \end{array}
        \right)
        \Rightarrow
        A^{-1} = \begin{pmatrix}
            1 & 0 & -3 \\
            0 & 1 & 0  \\
            0 & 0 & 1
        \end{pmatrix}
    \]
    \[
        \left(
        \begin{array}{ccc|ccc}
                1 & 0  & 0 & 1 & 0 & 0 \\
                0 & 1  & 0 & 0 & 1 & 0 \\
                0 & -5 & 1 & 0 & 0 & 1
            \end{array}
        \right)
        \sim
        \left(
        \begin{array}{ccc|ccc}
                1 & 0 & 0 & 1 & 0 & 0 \\
                0 & 1 & 0 & 0 & 1 & 0 \\
                0 & 0 & 1 & 0 & 5 & 1
            \end{array}
        \right)
        \Rightarrow
        B^{-1} = \begin{pmatrix}
            1 & 0 & 0 \\
            0 & 1 & 0 \\
            0 & 5 & 1
        \end{pmatrix}
    \]
    \[
        A^{-1}B^{-1}AB =
        \begin{pmatrix}
            1 & 0 & -3 \\
            0 & 1 & 0  \\
            0 & 0 & 1
        \end{pmatrix}
        \begin{pmatrix}
            1 & 0 & 0 \\
            0 & 1 & 0 \\
            0 & 5 & 1
        \end{pmatrix}
        \begin{pmatrix}
            1 & 0 & 3 \\
            0 & 1 & 0 \\
            0 & 0 & 1
        \end{pmatrix}
        \begin{pmatrix}
            1 & 0  & 0 \\
            0 & 1  & 0 \\
            0 & -5 & 1
        \end{pmatrix}
        =
    \]
    \[
        =
        \begin{pmatrix}
            1 & -15 & -3 \\
            0 & 1   & 0  \\
            0 & 5   & 1
        \end{pmatrix}
        \begin{pmatrix}
            1 & 0 & 3 \\
            0 & 1 & 0 \\
            0 & 0 & 1
        \end{pmatrix}
        \begin{pmatrix}
            1 & 0  & 0 \\
            0 & 1  & 0 \\
            0 & -5 & 1
        \end{pmatrix}
        =
        \begin{pmatrix}
            1 & -15 & 0 \\
            0 & 1   & 0 \\
            0 & 5   & 1
        \end{pmatrix}
        \begin{pmatrix}
            1 & 0  & 0 \\
            0 & 1  & 0 \\
            0 & -5 & 1
        \end{pmatrix}
        =
        \begin{pmatrix}
            1 & -15 & 0 \\
            0 & 1   & 0 \\
            0 & 0   & 1
        \end{pmatrix}
    \]

    \framebox{Ответ:
        $\begin{pmatrix}
                1 & -15 & 0 \\
                0 & 1   & 0 \\
                0 & 0   & 1
            \end{pmatrix}$
    }

    \section{С помощью алгебраических дополнений найти $A^{-1}$, если}
    \[
        A = \begin{pmatrix}
            1 & -15 & 0 \\
            0 & 1   & 0 \\
            0 & 0   & 1
        \end{pmatrix}
    \]
    \[
        A^{-1} =
        \begin{pmatrix}
            \begin{vmatrix}
                1 & 0 \\
                0 & 1
            \end{vmatrix}
             &
            -\begin{vmatrix}
                 -15 & 0 \\
                 0   & 1
             \end{vmatrix}
             &
            \begin{vmatrix}
                -15 & 0 \\
                1   & 0
            \end{vmatrix}
            \\
            -\begin{vmatrix}
                 0 & 0 \\
                 0 & 1
             \end{vmatrix}
             &
            \begin{vmatrix}
                1 & 0 \\
                0 & 1
            \end{vmatrix}
             &
            -\begin{vmatrix}
                 1 & 0 \\
                 0 & 0
             \end{vmatrix}
            \\
            \begin{vmatrix}
                0 & 1 \\
                0 & 0
            \end{vmatrix}
             &
            -\begin{vmatrix}
                 1 & -15 \\
                 0 & 0
             \end{vmatrix}
             &
            \begin{vmatrix}
                1 & -15 \\
                0 & 1
            \end{vmatrix}
            \\
        \end{pmatrix}
        \cdot
        \dfrac{1}{\begin{vmatrix}
                1 & -15 & 0 \\
                0 & 1   & 0 \\
                0 & 0   & 1
            \end{vmatrix}}
        =
        \begin{pmatrix}
            1 & 15 & 0 \\
            0 & 1  & 0 \\
            0 & 0  & 1
        \end{pmatrix}
        \cdot
        \dfrac{1}{1}
        =
        \begin{pmatrix}
            1 & 15 & 0 \\
            0 & 1  & 0 \\
            0 & 0  & 1
        \end{pmatrix}
    \]
    \fbox{Ответ:
        $\begin{pmatrix}
                1 & 15 & 0 \\
                0 & 1  & 0 \\
                0 & 0  & 1
            \end{pmatrix}$
    }

\end{sloppypar}
\end{document}
